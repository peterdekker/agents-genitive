\documentclass{article}
\usepackage[T1]{fontenc}
\usepackage[utf8x]{inputenc}
\usepackage{longtable}
\usepackage[left=3cm, right=3cm]{geometry}\title{interesting\_possessee}
\date{}
\begin{document}
\maketitle
\noindent
\begin{longtable}{p{1cm}|p{1cm}|p{1cm}|p{13cm}}
ID&Lemma&Tag&Sentence\\
\hline
2&bak&nheþ&Stigu þeir af baki og drápu þegar á dur og gekk maður til \textbf{baki} \\
\hline
3&bak&nheo&» Hrappur segir mart en spurði fás en þó varð hann brátt var að þeir ætluðu að stefna að Helga og lét hann vel yfir því og sagði að hans skal eigi á bak að \textbf{bak} \\
\hline
4&bak&nheþ&Hann hljóp að baki Kára og lagði til \textbf{baki} spjóti\\
\hline
5&bak&nheo&Bergur lagði til Kols í gegnum skjöldinn og fyrir brjóst Kol og féll hann á bak aftur og í því sló Finnbogi hann með steini í höfuðið svo að í smán mola lamdist hausinn og fékk hann bráðan \textbf{bak} \\
\hline
6&bak&nheþ&Hann steig þá af baki og gekk ofan Indriðastíg hjá Þyrli og beið þar uns Þorsteinn fór til blóthúss síns sem hann var \textbf{baki} \\
\hline
7&bak&nheo&Og í því heyrði Þórður brest á bak sér aftur og sér hann að Klyppur bróðir hans var högginn \textbf{bak} \\
\hline
8&bak&nheþ&En einn var sá maður er þeim fannst einna mest um er á baki \textbf{baki} \\
\hline
9&bak&nheo&En annan veg á þingið sátu þeir á einum stóli Rögnvaldur jarl og Þorgnýr og sat þar fyrir þeim hirð jarls og húskarlasveit Þorgnýs en á bak stólinum stóð bóndamúgurinn og allt umhverfis í \textbf{bak} \\
\hline
10&bak&nheþ&spjóti yfir ána til Bjarnar og kom í lær honum en Björn tók spjótið og skaut yfir ána til þeirra og varð maður fyrir og flaug í gegnum hann og tók Kolbein Þórðarson er sat að baki honum og höfðu báðir \textbf{baki} \\
\hline
11&bak&nheþ&Hann reið í gryfju nakkverja og féll hesturinn undir honum en hann af baki og varð honum meint \textbf{baki} \\
\hline
12&bak&nheo&Þá drógu þeir hann milli sín til skógar og bundu hendur hans á bak \textbf{bak} \\
\hline
13&bróðir&nken&Þar var þá og Halldór sonur Guðmundar á Möðruvöllum og Kolbeinn sonur Þórðar Freysgoða bróðir \textbf{bróðir} \\
\hline
14&bróðir&nken&Þorvaldur tinteinn bjó norður í Svínadal en Þorvarður bróðir \textbf{bróðir} í Fljótum\\
\hline
15&bróðir&nkeo&» Eftir það nefnir konungur til Eirík bróður Grana og með honum sex tigu manna að fara suður til Danmerkur að sitja um líf \textbf{bróður} \\
\hline
16&bróðir&nken&Bróðir hans hét Þorvarður er bjó norður í \textbf{Bróðir} \\
\hline
17&bróðir&nken&Maður hét Hafliði Höskuldsson bróðir Sighvats \textbf{bróðir} \\
\hline
18&bróðir&nkfn&Konungur bað Þórólf og þá bræður að þeir skyldu láta prímsignast því að það var þá mikill siður bæði með kaupmönnum og þeim mönnum er á mála gengu með kristnum \textbf{bræður} \\
\hline
19&bróðir&nkfn&Þorkell Skíðason fer þá heim til föðurs síns og vaxa þeir þar upp allir \textbf{bræður} \\
\hline
20&bróðir&nkfn&Þeir bræður gættu fjár föður \textbf{bræður} \\
\hline
21&bróðir&nkfn&Og er þetta spurðist þá ráku þeir bræður á burt Þóreyju systur \textbf{bræður} úr sveit\\
\hline
22&bróðir&nkfþ&Hann sendi orð bræðrum sínum og komu þeir til \textbf{bræðrum} \\
\hline
23&bróðir&nken&Svarthöfði sagði sér það þá í hug er Sturla bróðir þinn var drepinn frá oss en allir vér hraktir og skemmdir að vér mundum fegnir verða ef nokkur vildi þess réttar \textbf{bróðir} \\
\hline
24&bróðir&nken&Ormur Jónsson bjó á Breiðabólstað í Fljótshlíð bróðir \textbf{bróðir} \\
\hline
25&bróðir&nkfþ&Þegar skildi með þeim bræðrum er þeir komu til þingsins og verður Hreiðar skauttogaður mjög og færður í \textbf{bræðrum} \\
\hline
26&bróðir&nken&bróðir \textbf{bróðir} \\
\hline
27&bróðir&nken&Einar bróðir hans fór með \textbf{bróðir} \\
\hline
28&bróðir&nkeo&Síðan fór hann suður í land og ætlaði austur til Túnbergs að finna Þorstein drómund bróður \textbf{bróður} \\
\hline
29&bróðir&nken&Guðmundur bróðir Þorgils hafði verið á Miklabæ með honum um \textbf{bróðir} \\
\hline
30&bróðir&nkfn&Komu þá til fundar við konung þeir bræður Þórólfur og \textbf{bræður} \\
\hline
31&bróðir&nken&Ketill Sigfússon og Mörður bróðir \textbf{bróðir} \\
\hline
32&bróðir&nkeo&Hermundur Illugason undi lítt eftir Gunnlaug bróður sinn og þótti ekki hans hefnt að heldur þótt þetta væri að \textbf{bróður} \\
\hline
33&bróðir&nken&Mörukári og Valþjófur jarl bróðir \textbf{bróðir} \\
\hline
34&bróðir&nken&Brúsi Ljótsson og Ingimundur bróðir \textbf{bróðir} \\
\hline
35&bróðir&nkeþ&« Eg er kominn hingað með Glúmi bróður mínum þess erindis að biðja Hallgerðar dóttur \textbf{bróður} \\
\hline
36&bróðir&nken&Heyrir Ögmundur á stefnu og Guðmundur bróðir \textbf{bróðir} \\
\hline
37&bróðir&nken&« Ef eg em ráðinn til að fylgja þér Þorbergur bróðir þóttú viljir berjast við konung þá skal eg eigi við þig skiljast ef þú tekur betra ráð og mun eg fylgja ykkur Finni og taka þann kost sem þið sjáið ykkur til \textbf{bróðir} \\
\hline
38&bróðir&nken&Guðmundur hafði goðorð að meðför er átt hafði Ásgrímur bróðir hans og Þorvarður \textbf{bróðir} \\
\hline
39&bróðir&nkfn&Þeir bræður komu heim um kvöldið í Brekku og spurði faðir þeirra þá hvað þeir hefðu \textbf{bræður} \\
\hline
40&bróðir&nkfn&hittir þá bræður og segir hver efni hann ætlar í vera um samband þeirra \textbf{bræður} \\
\hline
41&bróðir&nken&Jón Ófeigsson bróðir \textbf{bróðir} \\
\hline
42&bróðir&nkfn&Þú munt og taka siðaskipti og er sá siður miklu betri þeim sem hann mega hljóta en hinum erfiðara um sem eigi eru til þess skapaðir og slíkir eru sem eg því að við bræður vorum \textbf{bræður} \\
\hline
43&bróðir&nkeo&Hann kemur til Laugarhúsa og hittir Bjarna bróður sinn og segir honum þessi tíðindi og biður að hann muni nokkurn hlut að eiga um eftirmál \textbf{bróður} \\
\hline
44&bróðir&nkfn&Og er þeir sáu er í virkinu voru flokk Þorvalds og svo skipamennina þá ganga þeir bræður \textbf{bræður} \\
\hline
45&bróðir&nken&» Þá mælti Hringur bróðir \textbf{bróðir} \\
\hline
46&bróðir&nkfn&Á því sumri komu heim bræður \textbf{bræður} \\
\hline
47&bróðir&nkfþ&» Þau fóru og komu til valsins og var þeim bræðrum velt í sleða og Þorkatli með þeim en þeir fóru á hestum er sárir \textbf{bræðrum} \\
\hline
48&bróðir&nken&Helgi var bróðir Þorkels er bjó í Hvammi í \textbf{bróðir} \\
\hline
49&bróðir&nken&Þá var og höfðingi yfir Jómsvíkingum Búi digri af Borgundarhólmi og Sigurður bróðir \textbf{bróðir} \\
\hline
50&bróðir&nkeo&Tók Þorvarður þá að leita liðveislu við Þorgils að hann veitti honum til hefndar eftir Odd bróður sinn við þá Hrafn og \textbf{bróður} \\
\hline
51&bróðir&nken&Snorri bróðir \textbf{bróðir} \\
\hline
52&bróðir&nkfn&Og um morguninn er verkmenn voru farnir til starfs fóru þeir bræður heim í bæinn og námu staðar við skáladyr og töluðust með um hríð hversu þeir skyldu með \textbf{bræður} \\
\hline
53&bróðir&nken&En þeir lögðu fetil á hönd Þórði og ætluðu af að höggva áður Þorbjörn Sælendingur og Þorsteinn lýsuknappur bróðir hans fengu grið til handa öllum þeim því að þar voru tengdir í \textbf{bróðir} \\
\hline
54&bróðir&nken&ung að aldri og var hún þó ekkja og bjó í Fljótsdal þar sem nú heitir að Þorgerðarstöðum og var að umsýslu með henni bróðir hennar er Kolfinnur \textbf{bróðir} \\
\hline
55&bróðir&nken&« Saman er bræðra eign best að sjá bróðir og sé eg til aufúsuorð að við skiptum \textbf{bróðir} \\
\hline
56&bróðir&nkeþ&Sé eg þá er þú hefir mitt traust og tvo hluti landa að þú mátt vel halda þínu að réttu fyrir Þorfinni bróður \textbf{bróður} \\
\hline
57&bróðir&nkfn&Arf og bætur eftir þá bræður átti að taka Einar bróðir þeirra en dótturson Snorra en Illugi var \textbf{bræður} \\
\hline
58&bróðir&nkeþ&Synir Vésteins fara til Gests frænda síns og skora á hann að hann komi þeim utan með ráðum sínum og Gunnhildi móður þeirra og Auði er Gísli hafði átta og Guðríði Ingjaldsdóttur og Geirmundi bróður \textbf{bróður} \\
\hline
59&bróðir&nken&« Þórir Helgason býður gerð sína á þessu máli og veit eg bróðir að þér mun þykja mart til þess \textbf{bróðir} \\
\hline
60&bróðir&nkfn&bræður því að synir Yngva voru þá \textbf{bræður} \\
\hline
61&bróðir&nkeo&Skal og það vera og vil eg leysa undan Þorstein bróður minn því að það er skaði að honum verði nokkuð til meins en það er þó eigi örvænt ef þeir Finnbogi berjast því að hann er hinn mesti ofurhugi en hér er hvoriga að spara sem vér \textbf{bróður} \\
\hline
62&bróðir&nkfn&« Hverja ráðagerð hafið þér nú fyrir yður bræður og \textbf{bræður} \\
\hline
63&bróðir&nkfn&Eftir það segir hann þeim viðskipti þeirra Hrafnkels og beiðir þá bræður fulltings og liðsinnis enn sem \textbf{bræður} \\
\hline
64&bróðir&nken&Önundur var bróðir \textbf{bróðir} \\
\hline
65&bróðir&nken&Synir Halls voru ungir og var Klængur bróðir \textbf{bróðir} vígssakaraðili\\
\hline
66&bróðir&nken&» Þorgerður var ekkja og hafði átt hana Halldór bróðir \textbf{bróðir} \\
\hline
67&bróðir&nken&Bróðir hennar hét \textbf{Bróðir} \\
\hline
68&bróðir&nken&« Undarlegur maður ertu bróðir því að Skeggi mun skjótt hér koma og hefna frænda síns með fjölmenni og hefir þú ekki megn að standa í stríði við hann þó að þú sért garpur \textbf{bróðir} \\
\hline
69&bróðir&nkeo&Eitthvert sinn gekk Arinbjörn til hans og spurði hvað ógleði hans ylli « nú þó að þú hafir fengið skaða mikinn um bróður þinn þá er það karlmannlegt að bera það \textbf{bróður} \\
\hline
70&bróðir&nkeo&Og nú fór hann enn til fundar við Þorkel bróður sinn og kemur hann þar í \textbf{bróður} \\
\hline
71&bróðir&nkfþ&Einar Skúlason var með þeim bræðrum Sigurði og Eysteini og var Eysteinn konungur mikill vin \textbf{bræðrum} \\
\hline
72&bróðir&nken&bróðir þeirra \textbf{bróðir} \\
\hline
73&bróðir&nkfn&» Síðan skiljast þeir bræður og fer Þorkell til fundar við þá \textbf{bræður} \\
\hline
74&bróðir&nkfþ&Eftir það sendir hann orð bræðrum Hálfdanar að þeir skyldu standa upp með honum ella kveðst hann mundu fara um allt héraðið og hrekja fyrir \textbf{bræðrum} \\
\hline
75&bróðir&nken&Börkur hinn digri og Þorgrímur bróðir \textbf{bróðir} \\
\hline
76&bróðir&nken&Nú er þér það að segja ef þú selur fram manninn undir öx Bjarna að jafnskjótt skaltu í burtu verða og norður til Hofs og hafa slíka sæmd sem Bjarni bróðir þinn hyggur þér og aldrei skaltu í sama húsi mér vera meðan þú \textbf{bróðir} \\
\hline
77&bær&nkeþg&Það var á einni hverri nótt að Steinn hljóp í brott úr bænum og skósveinn \textbf{bænum} með honum\\
\hline
78&bær&nkeo&Og þá er hann fór utan með Grunnafirði um bæ Hávars þá ræddu förunautar hans um við hann að þeir mundu koma þar til \textbf{bæ} og láta þar eftir hestinn\\
\hline
79&bær&nkeo&flaut fyrir bæ hans tjölduð og \textbf{bæ} \\
\hline
80&bær&nkeog&Og þegar er hann kom á bæinn gekk hann til föður síns og fagnaði honum og settist niður að fótum \textbf{bæinn} \\
\hline
81&bær&nkeþ-s&Kom Þorkell hrútur þar til móts við þá og hafði spurt að að nóni dags var Þorleifur í Bæ og hann mundi þar vera um \textbf{Bæ} \\
\hline
82&bær&nkeþg&koma við Björgyn og þegar spyr Þórður eftir konungi og var honum sagt að Magnús konungur var í bænum og hafði skömmu áður komið og vildi eigi láta kæja sig \textbf{bænum} \\
\hline
83&bær&nken&Var þá rannsakaður bær allur í Þingnesi og tekið fé það sem laust var innan gátta en rænt hjá fram hrossum \textbf{bær} \\
\hline
84&bær&nkfþ&Var það flest bónda að burtu voru af bæjum sínum og höfðu \textbf{bæjum} \\
\hline
85&bær&nkeþg&« Maður rennur þar út frá bænum á Uppsölum og mun sá sendur til Hofstaða til \textbf{bænum} \\
\hline
86&bær&nkeþ&Það er sagt þá er biskup reið frá skipi og menn hans að þeir áðu á bæ nokkurum í Landeyjum og sátu \textbf{bæ} \\
\hline
87&bær&nkfþ&sumir að Gríms Einarssonar skeggs en sumir á öðrum \textbf{bæjum} \\
\hline
88&bær&nkeog&» Komumaðurinn leiddi hann einshvers staðar út að túngarðinum er gert var um bæinn og mælti til \textbf{bæinn} \\
\hline
89&bær&nkeog&Mörður sagði að þeir mundu eigi koma á óvart Gunnari nema þeir tækju bónda af næsta bæ er Þorkell hét og létu hann fara nauðgan með sér að taka hundinn Sám og færi hann einn heim á \textbf{bæinn} \\
\hline
90&bær&nkfþ&Og nú er svo er komið bardaganum þá sáu menn af bæjum fund þeirra og fóru til að skilja \textbf{bæjum} \\
\hline
91&bær&nkeþ&Bersi fór byggðum á Laugaból í Laugadal því að Vermundur vildi eigi svo nær bæ sínum láta vera hráskinn þeirra Þorgeirs og \textbf{bæ} \\
\hline
92&bær&nkeþ&kom um nótt á bæ Haralds \textbf{bæ} \\
\hline
93&bær&nkeo&Þetta veitir faðir hans honum og hann gerir sér bæ í dal þeim og kallar á \textbf{bæ} \\
\hline
94&bær&nkfo&Voru þá sótt naut og sauðir á bæi til \textbf{bæi} mönnum\\
\hline
95&bær&nkeog&Eftir þessi tíðindi ríður Þorsteinn heim og er hann nálgaðist bæinn sá hann marga menn ríða í mót sér og kenndi þar föður sinn og marga kunningja og fóru allir hans að \textbf{bæinn} \\
\hline
96&bær&nkeþg&Og er hann kom til Túnsbergs þá gengu þeir Ólafur og Sigröður með lið sitt austur úr bænum á brekkuna og fylktu \textbf{bænum} \\
\hline
97&bóndi&nkfo&Eftir um haustið fóru þeir Haraldur konungur og Sigurður bróðir hans upp á Vörs og stefndu þar þing við \textbf{bændur} \\
\hline
98&bóndi&nkfn&Um veturinn eftir jól fóru menn þeirra Sighvats og Sturlu vestur til Víðidals og var erindi að bændur skyldu járna hesta sína og vera búnir þann tíma er þeir væru upp \textbf{bændur} \\
\hline
99&bóndi&nkfn&Voru þá upp gengin föngin og ætluðu bændur að þá mundi sest á þeirra kostnað og kurruðu þeir \textbf{bændur} \\
\hline
100&bóndi&nkfo&um lenda menn og ríka bændur og alla þá er honum var grunur á að nokkurrar uppreistar var af \textbf{bændur} \\
\hline
101&bóndi&nken-s&» Bóndi gengur að Þormóði og vildi sjá sár \textbf{Bóndi} \\
\hline
102&bóndi&nken&Tók Þorkell bóndi í hönd vísindakonunni og leiddi hana til þess sætis er henni var \textbf{bóndi} \\
\hline
103&bóndi&nken&Þorgils bóndi sagði þeim fóstbræðrum allt slíkt sem hann sagði Gretti en þeir gerðu svo mikil metorð hans að hvorigir lögðu öðrum öfugt orð en þó fóru ekki þykkjur þeirra \textbf{bóndi} \\
\hline
104&bóndi&nkfn&Fóru nú bændur heim síðan og kurraði það hver í sínum híbýlum að veturgestur Brynjólfs mundi eigi vera hvers drengs \textbf{bændur} \\
\hline
105&bóndi&nkfn&Þegar hann kom í veldi Noregskonungs þá höfðu bændur fyrir safnað og múg \textbf{bændur} \\
\hline
106&bóndi&nken&Menn komu brátt til þeirra og bóndi sá er þar bjó næst hét \textbf{bóndi} \\
\hline
107&bóndi&nkfn&En bændur kjósa hinn kost heldur að búa konungi veislur þá stund alla er hann þurfti til þess og tók konungur þann kost að hann fór um land að veislum með sumt lið sitt en sumt gætti skipa \textbf{bændur} \\
\hline
108&bóndi&nken&Þorsteinn bóndi gekk mjög að að vinna mönnum beinleika og taka við klæðum manna því að hann var hverjum manni \textbf{bóndi} \\
\hline
109&bóndi&nkfo&Erlingur átti jafnan þing við bændur og var þar talað oft um óspektir Sigurðar \textbf{bændur} \\
\hline
110&bóndi&nken&Eyjólfur bóndi og fylgdarmenn hans og Þuríður \textbf{bóndi} \\
\hline
111&bóndi&nkfn&Bændur sóttu þingið með her \textbf{Bændur} alvopnaðan\\
\hline
112&bú&nheo&Son hennar hét Eyjólfur er átti bú með \textbf{bú} \\
\hline
113&bú&nheo&Fór Kolbeinn norður til föður síns og var þar um veturinn en um vorið tóku þeir feðgar heimildum á Grenjaðarstöðum af Jóni Eyjólfssyni í Möðrufelli og gerði Kolbeinn þar bú \textbf{bú} \\
\hline
114&bú&nhen&Reið Þórður þá austur á Völlu til bús síns og var það þó lítið bú að því sem fyrr hafði á \textbf{bú} \\
\hline
115&bú&nheþ&Hafði hann aðdrátt að búi \textbf{búi} \\
\hline
116&bú&nheþ&Forni hét maður er farinn var til rétta en sá maður var fyrir búi hans er Þorgeir hét er bjó að Þverá í \textbf{búi} \\
\hline
117&bú&nheo&Börkur gengur þar í bú með Þórdísi og fær \textbf{bú} \\
\hline
118&bú&nheo&Hann bað Þuríðar dóttur Hallsteins goða til handa Ketilbirni fóstbróður sínum og fékk þann kost og gerði bú í Tungu í Króksfirði en stundum var hann með \textbf{bú} \\
\hline
119&bú&nheo&og réðu það með sér að hverfa að báðir samt og fóru norður til Reykjadals og höfðu nær fimm tigu manna og settust í bú á \textbf{bú} \\
\hline
120&bú&nheo&Átti hann nú bú að Staðarhóli en Hrafn gerði bú í Stafaholti með ráði \textbf{bú} \\
\hline
121&bú&nheþ&Hann hafði þá forráð öll fyrir búi þeirra feðga og tilöflun alla en þó var Kveld-Úlfur hress maður og vel \textbf{búi} \\
\hline
122&bú&nhen&Bú hans hafði þar staðið meðan hann var \textbf{Bú} \\
\hline
123&bú&nheo&En Gísli hafði bú eftir og saknar engis í að nú sé búið verra en \textbf{bú} \\
\hline
124&bú&nheo&Bergþóra sagði honum að hann skyldi til þings ríða « en nú skalt þú fyrst fara upp í Þórólfsfell og sjá þar um bú og vera þar eigi lengur en eina nótt eða \textbf{bú} \\
\hline
125&bú&nheo&Eftir þetta setur Helgi Ásbjarnarson bú á Oddsstöðum og var vel um samfarir \textbf{bú} \\
\hline
126&dagur&nkeog&Um daginn eftir gekk Hrútur fyrir konung við þrjá tigu manna og kvaddi \textbf{daginn} \\
\hline
127&dagur&nkeo&Og einn dag sjá þeir að maður reið af landi ofan til \textbf{dag} \\
\hline
128&dagur&nkeþ&Þorkell var ofláti mikill og vann ekki fyrir búi þeirra en Gísli vann nótt með \textbf{degi} \\
\hline
129&dagur&nkeo&Og nú kom Einar til konungs og setur konungur hann hið næsta sér og var það eftir hinn átta dag er Einar \textbf{dag} \\
\hline
130&dagur&nkeo&Það var einn dag að Þorsteinn talar við föður sinn að hann mundi fara austur á fund Ingimundar jarls sem hann hefði heitið \textbf{dag} \\
\hline
131&dagur&nkeo&Sighvatur gekk einn dag um þorp nokkuð og heyrði að einnhver húsbóndi veinaði mjög er hann hafði misst konu \textbf{dag} \\
\hline
132&dagur&nkeog&Og er á leið daginn urðu þeir Egill varir við að þar voru sex menn á skóginum og þóttust vita að þar mundu vera húskarlar \textbf{daginn} \\
\hline
133&dagur&nkeo&Það var einn dag að Þorsteinn spurði hirðmann einn hvað til bæri um ógleði \textbf{dag} \\
\hline
134&dagur&nkeo&En Þorgrímur Þórisson fór þann dag til laugar er þeirra var utan von með maltið og voru að Hrafnagilslaugu og sex húskarlar \textbf{dag} með honum\\
\hline
135&dagur&nkeo&« Það höfum vér heyrt að þú hafir lítt verið leiðitamur þínum óvinum og er nú vel að þú kennir þess á þér í \textbf{dag} \\
\hline
136&dagur&nkeo&Lét hann þá setja borð sitt og senda menn víða um byggðina og bauð til sín mörgum mönnum og var þar þann dag veisla mikil og prýðilega \textbf{dag} \\
\hline
137&dagur&nkeo&Einn dag um þingið er á leið gengur Ófeigur frá búð og kemur til Mýramannabúðar og var hann Egill úti í virkinu og talar við mann \textbf{dag} \\
\hline
138&dagur&nkeo&þá vinda þeir fyrst klæði sín og búa sig til göngu og ganga þann dag \textbf{dag} \\
\hline
139&dagur&nkeo&Annan dag fór Gautur og hans förunautar að fá eldibranda en Þorgeir var heima og bjó \textbf{dag} \\
\hline
140&dagur&nkeo&Og þá er Ívars var heim von þá tók Eyjólfur loðkápu og hafði hvern \textbf{dag} \\
\hline
141&dagur&nken&Þar kom og Bárður vin Gunnars því að hann hafði frétt að glíman var lögð og vildi hann finna Gunnar áður sá dagur \textbf{dagur} \\
\hline
142&dagur&nkeog&» Þá mælti Gunnar að þeir mundu til \textbf{daginn} um daginn\\
\hline
143&dagur&nkeog&Þóttist Sturla þá sjá að allt skap konungs var þá léttara en hinn fyrra \textbf{daginn} \\
\hline
144&dagur&nkeo&Og nú um sumarið fjölmenntu hvorirtveggju til alþings eftir föngum þá er þar kom og riðu menn á þing hinn næsta dag fyrir Jóns messu \textbf{dag} \\
\hline
145&dagur&nkfo&Sigldu þeir þá bæði nætur og daga allt til þess er þeir Karli lögðu aftan dags að eyjum \textbf{daga} \\
\hline
146&dagur&nkeog&Þeir gengu til \textbf{daginn} eftir um daginn\\
\hline
147&dagur&nkfo&Sendimenn biðja enn fresta um þrjá daga og þess með að Ólafur konungur sendi þá menn sína að heyra orð Aðalsteins konungs hvort hann vill eða eigi þenna \textbf{daga} \\
\hline
148&dagur&nkeo&Segið honum það til jartegna að hann fái Vandráði hest þann er eg gaf Karli fyrra dag og söðul sinn og son sinn til \textbf{dag} \\
\hline
149&dagur&nkeog&En um daginn er þeir riðu ofan eftir Jökulsárbökkum talast þeir við Þorvarður og Þorgils um liðveislu þá er Þorvarður hafði heitið Þorgilsi til \textbf{daginn} í Skagafirði\\
\hline
150&dagur&nkeo&Annan dag viku var \textbf{dag} \\
\hline
151&dagur&nkeo&Það er sagt einn dag er þeir feðgar Höskuldur og Ólafur gengu frá búð og til \textbf{dag} við Egil\\
\hline
152&dagur&nken&« að hlaða hér vörðu á hóli þessum er nú stöndum við á og mun þá finnast er ljós dagur er og sér héðan frá vörðunni til kleifanna er skammt er að \textbf{dagur} \\
\hline
153&dagur&nkeo&Af því tali gekk Hákon hvern dag til máls við hana meðan þau voru \textbf{dag} \\
\hline
154&dagur&nkfþ&Hrafnkels saga Freysgoða Það var á dögum Haralds kóngs hins hárfagra Hálfdanarsonar hins \textbf{dögum} \\
\hline
155&dagur&nkeo&Sáu þeir það sem var að þau Búi áttu hvern pening eftir \textbf{dag} dag\\
\hline
156&dagur&nkeog&Hann fer til nausts Þórdísar og leggst þá niður og felst því að hann ætlaði að Böðvar mundi róa á sjá um daginn sem hann var \textbf{daginn} \\
\hline
157&dagur&nkeog&Hann hafði sótt korn um daginn og kom heim til \textbf{daginn} \\
\hline
158&dagur&nkeo&En er það sáu Væringjar þá gengu þeir einn dag svo til leiksins að þeir höfðu sverð undir möttlum en \textbf{dag} undir höttum\\
\hline
159&dagur&nkeo&Það var einnhvern dag er Steinn Skaftason var fyrir konungi og spurði hann máls ef hann vildi hlýða drápu þeirri er Skafti faðir hans hafði ort um \textbf{dag} \\
\hline
160&dagur&nkeo&» Annan dag eftir er sagt víg \textbf{dag} frá Stokkahlöðu\\
\hline
161&dagur&nken-s&Létti Þórir eigi fyrr ferðinni en hann kom um nóttina á Súlu og spurði hann þar þau tíðindi að um kveldið hafði þar komið Dagur Hringsson og margar aðrar sveitir af \textbf{Dagur} mönnum\\
\hline
162&dagur&nkeo&Og á þinginu gekk hann einn dag til Einars Þveræings og heimti hann á tal við sig og sagði \textbf{dag} \\
\hline
163&dagur&nkeo&« Er Þórður gamall maður og barnlaus og ætla eg Ólafi allt fé eftir hans dag en þú mátt hitta hann ávallt er þú \textbf{dag} \\
\hline
164&dagur&nkeo&Og um vorið einn dag gekk Gunnlaugur úti og Þorkell frændi \textbf{dag} með honum\\
\hline
165&dagur&nkeo&Svo bar hér til að það var einn dag á jólunum að komu til Einars bónda illvirkjar margir \textbf{dag} \\
\hline
166&dóttir&nven&Dóttir Snorra Karlsefnissonar hét \textbf{Dóttir} \\
\hline
167&dóttir&nven&dóttir Ketils \textbf{dóttir} úr Hrafnistu\\
\hline
168&dóttir&nveo&» Var nú svo að Hólmkell gifti Víglundi Ketilríði dóttur sína en Þorgrímur Sigurði spaka Helgu dóttur sína en Helgi Gunnlaugi ofláta Ragnhildi dóttur sína og var nú setið að þessum brúðlaupum öllum \textbf{dóttur} \\
\hline
169&dóttir&nveþ&dóttur Ásmundar \textbf{dóttur} \\
\hline
170&dóttir&nven&Þar var og í ferð með honum Þórný dóttir hans og ætlaði hann að hún skyldi \textbf{dóttir} \\
\hline
171&dóttir&nveo&» Fer það fram að Kolbjörn fastnar Þórði dóttur sína Solrúnu með þeim skilmála að á hálfsmánaðar fresti skal hann sækja brullaupið heim til \textbf{dóttur} \\
\hline
172&dóttir&nven&Ólöf hét dóttir \textbf{dóttir} \\
\hline
173&dóttir&nveþ&dóttur Þorfinns \textbf{dóttur} \\
\hline
174&dóttir&nven&Dóttir \textbf{Dóttir} \\
\hline
175&dóttir&nveþ&dóttur Eyvindar \textbf{dóttur} \\
\hline
176&dóttir&nveo&» Var nú svo að Hólmkell gifti Víglundi Ketilríði dóttur sína en Þorgrímur Sigurði spaka Helgu dóttur sína en Helgi Gunnlaugi ofláta Ragnhildi dóttur sína og var nú setið að þessum brúðlaupum öllum \textbf{dóttur} \\
\hline
177&dóttir&nven&Þau Þórir og Þorgerður voru ásamt til þess að þau gátu son og \textbf{dóttir} \\
\hline
178&dóttir&nven&Halldóra hét dóttir \textbf{dóttir} \\
\hline
179&dóttir&nven&en Arndís hin auðga var dóttir \textbf{dóttir} \\
\hline
180&dóttir&nvfn&Dætur Álfs í Dölum voru þær \textbf{Dætur} \\
\hline
181&dóttir&nveþ&Páll var son Þorvalds og Halldóru dóttur Sveins \textbf{dóttur} \\
\hline
182&dóttir&nveþ&Þorkell spyr nú brátt hvað um er að vera og þykir sér horfa til óvirðingar og dóttur sinni ef Kormákur vill þetta eigi meir \textbf{dóttur} \\
\hline
183&dóttir&nveþ&jarl átti Sigríði dóttur \textbf{dóttur} \\
\hline
184&dóttir&nveo&Hann orti mansöngsdrápu um Ástríði dóttur Ólafs \textbf{dóttur} \\
\hline
185&dóttir&nven&dóttir \textbf{dóttir} af Lundum\\
\hline
186&dóttir&nven&Dóttir Styrkárs hét Kerling og heldur \textbf{Dóttir} \\
\hline
187&dóttir&nven&Hallveig var dóttir \textbf{dóttir} \\
\hline
188&dóttir&nven&Hún var dóttir \textbf{dóttir} úr Vatnsfirði\\
\hline
189&dóttir&nveþ&dóttur Ketils \textbf{dóttur} \\
\hline
190&dóttir&nvfn&Dætur Eyvindar voru þær Þorbjörg er átti Þormóður úr Laxárdal og Fjörleif er átti Þórir leðurháls son Þorsteins \textbf{Dætur} \\
\hline
191&dóttir&nveþ&dóttur \textbf{dóttur} háls\\
\hline
192&dóttir&nveþ&dóttur Gils \textbf{dóttur} \\
\hline
193&dóttir&nven&Dóttir Þórðar hét \textbf{Dóttir} \\
\hline
194&dóttir&nven&svo að Gunnhildur var móðir Eyvindar dóttir Hálfdanar jarls en móðir hennar var Ingibjörg dóttir Haralds konungs hins \textbf{dóttir} \\
\hline
195&dóttir&nveþ&dóttur \textbf{dóttur} af Hvanneyri\\
\hline
196&dóttir&nven&dóttir \textbf{dóttir} \\
\hline
197&dóttir&nveþ&dóttur \textbf{dóttur} á Kambsnesi\\
\hline
198&dóttir&nveþ&Þar kemur að þessi tíðindi koma fyrir Hlégunni dóttur Hjörvarðar \textbf{dóttur} \\
\hline
199&dóttir&nven&Dóttir Holbarka var \textbf{Dóttir} \\
\hline
200&dóttir&nveþ&Tómas er átti Höllu dóttur Þórðar \textbf{dóttur} \\
\hline
201&dóttir&nven&Guðrún hét dóttir \textbf{dóttir} \\
\hline
202&dóttir&nven&Hét sonur þeirra Einar en Ásvör \textbf{dóttir} \\
\hline
203&dóttir&nven&Steinn og Helgi voru synir þeirra en Arndís dóttir og Þuríður er Sleitu-Björn \textbf{dóttir} \\
\hline
204&dóttir&nven&þeirra dóttir \textbf{dóttir} \\
\hline
205&dóttir&nvfn&Á skipi með Bárði var Herþrúður kona hans og dætur \textbf{dætur} allar\\
\hline
206&dóttir&nven&hans dóttir var \textbf{dóttir} \\
\hline
207&dóttir&nveþ&Var það í ráðagerðum að Mörður skyldi biðja Þorkötlu dóttur Gissurar hvíta og skyldi Þorgeir þegar ríða vestur um ár með þeim Valgarði og \textbf{dóttur} \\
\hline
208&dóttir&nven&Jóreiður hét dóttir þeirra er gefin var \textbf{dóttir} \\
\hline
209&dóttir&nveo&Hann átti Herdísi dóttur \textbf{dóttur} frá Höfða\\
\hline
210&dóttir&nven&Svo segja sumir menn að Þorlaug kona Egils færi með honum og Þorgerður dóttir þeirra og væri hún átta \textbf{dóttir} gömul\\
\hline
211&dóttir&nven&þeirra dóttir var \textbf{dóttir} \\
\hline
212&dóttir&nveo&hann átti Steinvöru dóttur Þorsteins \textbf{dóttur} \\
\hline
213&dóttir&nven&Ragnhildur dóttir Erlings \textbf{dóttir} \\
\hline
214&dóttir&nveo&« Eigi er eg einhlítur um svör þessa máls og vil eg ráðast um við móður hennar og svo við dóttur mína og einkum við Þórð gelli frænda \textbf{dóttur} \\
\hline
215&dóttir&nveþ&dóttur Hrólfs \textbf{dóttur} \\
\hline
216&dóttir&nveo&Ásbjörn Arnórsson átti Ingunni dóttur Þorsteins Snorrasonar \textbf{dóttur} \\
\hline
217&dóttir&nveo&Hann átti Herdísi dóttur Halldórs \textbf{dóttur} \\
\hline
218&dóttir&nveþ&dóttur Hróðgeirs hins \textbf{dóttur} \\
\hline
219&dóttir&nven&Þar var brúðurin í för með þeim og Þorgerður dóttir hennar og var hún \textbf{dóttir} fríðust\\
\hline
220&dóttir&nveo&Hann átti Kolfinnu dóttur Gissurar \textbf{dóttur} \\
\hline
221&dóttir&nven&dóttir Sigurðar \textbf{dóttir} \\
\hline
222&dóttir&nven&Móðir hennar var Ingiríður dóttir Sigurðar konungs sýr og \textbf{dóttir} \\
\hline
223&dóttir&nven&Hún var dóttir Böðvars hersis \textbf{dóttir} \\
\hline
224&dóttir&nven&Arnfríður hét dóttir \textbf{dóttir} \\
\hline
225&dóttir&nveo&Hann átti Sigríði dóttur Þorgríms \textbf{dóttur} \\
\hline
226&dóttir&nveo&Þrándur mjóbeinn átti dóttur Gils \textbf{dóttur} \\
\hline
227&dóttir&nveþ&dóttur Geirmundar \textbf{dóttur} \\
\hline
228&dóttir&nven&hans dóttir var \textbf{dóttir} \\
\hline
229&dóttir&nveþ&dóttur Þorfinns stranga og \textbf{dóttur} \\
\hline
230&dóttir&nven&Ingibjörg var dóttir Snorra og Guðrúnar \textbf{dóttir} \\
\hline
231&dóttir&nven&stóðhross fimm saman og fingurgull og feld hlaðbúinn er honum hafði gefið Sigríður dóttir Eyjólfs Snorrasonar goða austan frá Höfðabrekku er átt hafði Jón \textbf{dóttir} \\
\hline
232&dóttir&nveo&Rögnvaldur átti Hildi dóttur Hrólfs \textbf{dóttur} \\
\hline
233&dóttir&nven&Er dóttir þín kona eigi fálynd og eigi einn líklegri en annar til \textbf{dóttir} með henni\\
\hline
234&dóttir&nven&dóttir \textbf{dóttir} úr Dölum\\
\hline
235&dóttir&nven&« Þú skalt ríða suður til fundar við Mörð og bið hann að þið skipið máldaga annan og sitji dóttir hans þrjá vetur í \textbf{dóttir} \\
\hline
236&dóttir&nveþ&En Arneiður réð sig í skip með Þorvaldi og vildi fylgja dóttur sinni til \textbf{dóttur} \\
\hline
237&dóttir&nveþ&dóttur Þóris \textbf{dóttur} \\
\hline
238&dóttir&nven&Þar var Kerling í för með þeim dóttir \textbf{dóttir} \\
\hline
239&dóttir&nveþ&dóttur Þórarins hins \textbf{dóttur} \\
\hline
240&dóttir&nven&Guðríður hét dóttir \textbf{dóttir} \\
\hline
241&dóttir&nveo&« Það er orðtak mjög margra manna Þormóður að þú fíflir Þórdísi dóttur mína og er mér það lítt að skapi að hún hljóti orð af \textbf{dóttur} \\
\hline
242&dóttir&nveo&áður hann fékk Þórhildar dóttur Þorsteins \textbf{dóttur} \\
\hline
243&dóttir&nven&En hún sagði til og kvaðst Droplaug heita og vera dóttir Björgólfs \textbf{dóttir} \\
\hline
244&dóttir&nven&dóttir Hrólfs Ingjaldssonar Fróðasonar \textbf{dóttir} \\
\hline
245&dóttir&nveo&Hann átti Þórdísi dóttur Gunnars \textbf{dóttur} \\
\hline
246&dóttir&nven&Úlfhildur dóttir \textbf{dóttir} \\
\hline
247&dóttir&nveo&Klængur átti Ástu dóttur Andrésar Sæmundarsonar systur \textbf{dóttur} \\
\hline
248&dóttir&nveo&Hann átti Þorgerði dóttur \textbf{dóttur} \\
\hline
249&dóttir&nvfn&Dætur Þorláks voru þær Ásbjörg \textbf{Dætur} \\
\hline
250&dóttir&nven&Ólafur fékk þeirrar konu er Sölveig hét eða Sölva dóttir Hálfdanar \textbf{dóttir} vestan af Sóleyjum\\
\hline
251&dóttir&nven&Ljótur sonur Halls átti Helgu dóttur Einars frá Þverá og var þeirra dóttir Guðrún er átti Ari Þorgilsson af \textbf{dóttir} \\
\hline
252&dóttir&nveo&» Hann átti Halldóru dóttur Arnórs \textbf{dóttur} \\
\hline
253&dóttir&nven&Þeirra dóttir var Þorgerður móðir Jóhanns biskups hins \textbf{dóttir} \\
\hline
254&faðir&nken&» Faðir hans fýsti hann mjög utan að \textbf{Faðir} \\
\hline
255&faðir&nkeþ&« Það vildi eg fóstri minn að vel svarir þú föður mínum um bónorðið og virðir mikils \textbf{föður} flutning\\
\hline
256&faðir&nken&« að þeir skulu af mér taka sæmdina ef þeim þykir það betra og segið þeim það að eg býð þeim til mín og skal faðir minn ekki mein gera \textbf{faðir} \\
\hline
257&faðir&nken&Faðir hans var Helgi hinn hvassi en móðir hans var Áslaug dóttir Sigurðar orms í auga Ragnarssonar \textbf{Faðir} \\
\hline
258&faðir&nken&hann var faðir Ljótólfs \textbf{faðir} í Svarfaðardal\\
\hline
259&faðir&nken&ef þú hefðir eigi fyrri valdið upphöfum og fullkomnum fjandskap við okkur bræður þar sem faðir okkar veitti þér það upphald að þú munt aldrei fá svo góðu launað þótt þú leitaðir við það sem þú skyldir en eigi með vélum og prettum sem þú ert nú sannprófaður \textbf{faðir} \\
\hline
260&faðir&nken&Guðbrandur kom til Þverár og sagði að faðir hans hafði hann þangað \textbf{faðir} \\
\hline
261&faðir&nken&Hálfdan var faðir Ívars hins \textbf{faðir} \\
\hline
262&faðir&nken&faðir \textbf{faðir} \\
\hline
263&faðir&nken&Eindriði faðir Styrkárs og Ásbjörn faðir Eindriða \textbf{faðir} \\
\hline
264&faðir&nken&Þorleifur hreimur var faðir \textbf{faðir} \\
\hline
265&faðir&nken&Jörundur var og faðir \textbf{faðir} \\
\hline
266&faðir&nkeþ&Ásdís húsfreyja sendir eftir mönnum og var búið um lík Atla og var hann jarðaður hjá föður \textbf{föður} \\
\hline
267&faðir&nken&Helgi var faðir \textbf{faðir} \\
\hline
268&faðir&nkeþ&Tak nú hér hundrað silfurs og lát hann á brott og skal eg svo til stilla að hann sé eigi hér tekinn á þínum varnaði svo að það sé þér lagið til ámælis en vér munum þó eftir honum leita þó að föður vors sé eigi að \textbf{föður} \\
\hline
269&faðir&nkeþ&Hún bað hann þá sitja fyrir Þórði Kolgrímssyni « því að faðir hans var mestur mótgöngumaður Herði föður \textbf{föður} \\
\hline
270&faðir&nken&Ketill var faðir \textbf{faðir} \\
\hline
271&faðir&nken&faðir Ósvífurs \textbf{faðir} spaka\\
\hline
272&faðir&nken&faðir \textbf{faðir} \\
\hline
273&faðir&nken&« Ger þú þér ekki angur að faðir um komur hans en eg mun ræða við hann að hann láti af komum \textbf{faðir} \\
\hline
274&faðir&nkeþ&Var hann því með Skúla frænda sínum meðan hann var ungur að hann þóttist þar betur kominn sakir áleitni Þórðar Kolbeinssonar en hjá föður \textbf{föður} \\
\hline
275&faðir&nkeþ&Þá spurðu hásetar hans hvað er hann bærist fyrir en hann svaraði að hann ætlaði að halda siðvenju sinni og þiggja að föður sínum veturvist « og vil eg halda skipinu til Grænlands ef þér viljið mér fylgd \textbf{föður} \\
\hline
276&faðir&nken&Var hann og kallaður vandræðaskáld sem faðir \textbf{faðir} \\
\hline
277&faðir&nken&faðir \textbf{faðir} \\
\hline
278&faðir&nken&faðir \textbf{faðir} í Reykjaholti\\
\hline
279&faðir&nken&En Sturla faðir hennar fékk henni sex tigi \textbf{faðir} \\
\hline
280&faðir&nken&« Ekki fer eg að því þó að þú hafir svelt þig til fjár og faðir \textbf{faðir} \\
\hline
281&faðir&nkeo&En nú skal fyrst segja hversu hann leiddi til sanns átrúnaðar föður sinn og \textbf{föður} heimamenn\\
\hline
282&faðir&nken&Það sama sumar kom Bjarni skipi sínu á Eyrar er faðir hans hafði brott siglt um \textbf{faðir} \\
\hline
283&faðir&nken&Og sem þeir komu heim um haustið var Örn faðir \textbf{faðir} andaður\\
\hline
284&faðir&nkfo&Eru hér margir ríkra manna synir af Íslandi og munu feður þeirra mikið liðsinni veita að þessu \textbf{feður} \\
\hline
285&faðir&nken&faðir hans « þætti mér og það kaup vel fara að þú héldir mig til konungs en eg ynni þér slíkrar sæmdar sem faðir yðvar hefir \textbf{faðir} \\
\hline
286&faðir&nken&faðir Hlenna hins \textbf{faðir} \\
\hline
287&faðir&nken&Böðvar var faðir \textbf{faðir} \\
\hline
288&faðir&nken&faðir \textbf{faðir} \\
\hline
289&faðir&nken&Höskuldur faðir hans hefir vakið bónorð fyrir hönd \textbf{faðir} og beðið þín\\
\hline
290&faðir&nken&« Þann einn spurdaga höfum vér til þín Höskuldur að vér viljum þessu vel svara því að vér hyggjum að fyrir þeirri konu sé vel séð er þér er gift en þó mun faðir minn mestu af ráða því að eg mun því samþykkjast hér um sem hann \textbf{faðir} \\
\hline
291&faðir&nken&Ófeigur faðir hans var fálátur við hann og unni honum \textbf{faðir} \\
\hline
292&faðir&nken&faðir \textbf{faðir} í Álftafirði\\
\hline
293&faðir&nkeþ&Og svo segir mér hugur um sem fleira muni til vandræða verða með okkur föður þínum og hans frændum en sjá megi fyrir hvern enda eiga mun og muntu jafnan með miklum vanda verða í millum að \textbf{föður} \\
\hline
294&faðir&nken&Bróðir Böðvars var Sigurður faðir Eiríks \textbf{faðir} \\
\hline
295&faðir&nken&hans fór þegar með hann vestur á Agðir og settist þar til ríkis á Ögðum þeim er átt hafði Haraldur faðir \textbf{faðir} \\
\hline
296&faðir&nken&Lét hann Þorstein ná eignum sínum og þar með gerðist hann lendur maður konungs svo sem faðir hans hafði \textbf{faðir} \\
\hline
297&faðir&nkeo&Snorri sonur Snorra goða bjó í Sælingsdalstungu eftir föður \textbf{föður} \\
\hline
298&faðir&nken&hann var faðir \textbf{faðir} \\
\hline
299&faðir&nken&Mun hann vera þrályndur í skapi sem faðir hans en hafa brjóst \textbf{faðir} \\
\hline
300&faðir&nkeo&Gaf Eyvindur Þrándi arf allan eftir föður þeirra ef Björn andaðist fyrr en \textbf{föður} \\
\hline
301&faðir&nken&Auðun skökull var faðir \textbf{faðir} mosháls\\
\hline
302&faðir&nken&Þórir jarl þegjandi hafði þá ríki þvílíkt sem haft hafði Rögnvaldur jarl faðir \textbf{faðir} \\
\hline
303&faðir&nken&faðir \textbf{faðir} \\
\hline
304&faðir&nken&faðir Ísleifs \textbf{faðir} \\
\hline
305&faðir&nken&Þórður Andrésson reið til þings með Gissurarsonum og veitti þeim að öllum málum á því þingi og Andrés faðir \textbf{faðir} \\
\hline
306&faðir&nken&faðir \textbf{faðir} \\
\hline
307&faðir&nken&faðir \textbf{faðir} \\
\hline
308&faðir&nken&« Mikils virðir faðir minn vingan þína er hann veitir þér slíkt og annað það er þú biður \textbf{faðir} \\
\hline
309&faðir&nken&Nú höfum vér þriðjung goðorðs en faðir vor \textbf{faðir} \\
\hline
310&faðir&nkeþ&Þess er getið að Þorsteinn fagri beiddist fjárlánstillaga af föður sínum og kvaðst vilja fara af landi á \textbf{föður} \\
\hline
311&ferð&nvfþ&Hann fann tvo menn um kveldið í skógi og sögðust vera heimamenn í Vatnsfirði og spurðu að ferðum \textbf{ferðum} \\
\hline
312&ferð&nveo&Hún sér ferð \textbf{ferð} \\
\hline
313&ferð&nveng&En er ferðin var búin þá var það af ráðið að Sigurður skyldi fara en Eysteinn skyldi hafa landráð af hendi beggja \textbf{ferðin} \\
\hline
314&ferð&nveþ&Hví skyldir þú eigi hyggja fyrir því áður þú hétir þeirri ferð að þú hefir ekki ríki til þess að mæla í mót Ólafi \textbf{ferð} \\
\hline
315&ferð&nveþg&Um vorið eftir þá er Þorkell hafði búið skip sitt fer hann suður um Breiðafjörð og fær sér þar hest og ríður einn saman og léttir eigi ferðinni fyrr en hann kemur í Ás til Eiðs frænda \textbf{ferðinni} \\
\hline
316&ferð&nvfo&Guðrún spurði vandlega um ferðir hans en því næst að \textbf{ferðir} \\
\hline
317&ferð&nveo&Lét Þrándur illa yfir ferð \textbf{ferð} \\
\hline
318&ferð&nveþ&Frá ferð Órækju er það að segja að hann kom til Laugarvatns um \textbf{ferð} skeið\\
\hline
319&ferð&nveo&Menn fara eftir honum og vildu sjá ferð hans og svo ef honum mætti nokkuð við \textbf{ferð} \\
\hline
320&ferð&nven&Hinn fyrsta aftan veislunnar var sén ferð berserkjanna og kvíddu menn mjög við \textbf{ferð} \\
\hline
321&ferð&nveþ&Létti hann eigi sinni ferð fyrr en hann kom heim til \textbf{ferð} \\
\hline
322&ferð&nveþg&Hafði Þórir einn forráð liðs þess og svo aflan þá alla er fengist í \textbf{ferðinni} \\
\hline
323&ferð&nveþ&En frá ferð hans er það að segja að skip það týndist og kom engi maður \textbf{ferð} \\
\hline
324&ferð&nveo&Nú er að segja frá Össuri að Þverá að hann heldur njósnum til um ferð Þórðar þá hann fer frá \textbf{ferð} \\
\hline
325&ferð&nveo&En þó höfðu þeir áður gert njósn ofan í Orkadal og svo í Skaun og létu segja um ferð Ólafs konungs allt það er þeir vissu \textbf{ferð} \\
\hline
326&ferð&nveþ&Tók sá fyrst til orða er kominn var af Raumaríki og segir frá ferð Ólafs digra og þeim ófriði er hann gerði bæði í manna aftökum og \textbf{ferð} meiðslum\\
\hline
327&ferð&nveo&Ekki vissu landsmenn til um ferð \textbf{ferð} \\
\hline
328&ferð&nveo&Og er hann hefur búið um varnað sinn þá býr hann ferð sína til \textbf{ferð} \\
\hline
329&ferð&nveng&kaupir skip og fer utan og kona hans með honum og er sagt að honum greiðist vel ferðin og kemur af hafi norður við Hálogaland og er þar um veturinn í Þjóttu með Sveini Hárekssyni og var vel virður og þóttust menn sjá á honum stórmennsku og virti hann þau bæði \textbf{ferðin} \\
\hline
330&ferð&nveþg&» Og nú kvað hann heldur mundu seinka ferðinni þeirra \textbf{ferðinni} \\
\hline
331&ferð&nveþ&Það er sagt frá ferð Þórðar að hann sækir fund Ólafs \textbf{ferð} \\
\hline
332&ferð&nveo&» Síðan lét konungur flytja ferð \textbf{ferð} upp á land\\
\hline
333&ferð&nven&Vildi eg að þú gerðir hans ferð hingað sæmilega og unna honum góðra sætta fyrir það er hann telur á \textbf{ferð} \\
\hline
334&ferð&nven&Síðan lét Finnbogi í haf og greiðist vel þeirra ferð og komu við \textbf{ferð} \\
\hline
335&ferð&nveþ&Nú er ekki að segja frá þeirra ferð fyrr en þeir koma á \textbf{ferð} \\
\hline
336&ferð&nveþ&Er það að segja skjótast af ferð Rauðs að hann sigldi heim í \textbf{ferð} \\
\hline
337&ferð&nven&Svo er sagt að nú líkar Skútu forkunnar vel og þykir nú gott hlut í að eiga með honum er hann vildi nokkuð að manna vera sjálfur og kallar hans ferð góða \textbf{ferð} \\
\hline
338&ferð&nvfþ&Hann kvongaðist þá er hann létti af ferðum og fékk Herdísar \textbf{ferðum} \\
\hline
339&ferð&nveo&Smalamaður Arnkels varð var við ferð þeirra og segir \textbf{ferð} \\
\hline
340&ferð&nvfo&Tökum vér það allt fyrir satt er í þeim kvæðum finnst um ferðir \textbf{ferðir} eða orustur\\
\hline
341&ferð&nven&fóru utan landveg og héldu heilu öllu þar til er þeir komu í Noreg og var þeirra ferð allmjög \textbf{ferð} \\
\hline
342&ferð&nveþ&Riðu þeir síðan heim til Staðar og lét Þorgils vel yfir \textbf{ferð} ferð\\
\hline
343&ferð&nveo&Bindur hún um sár hans og lætur eigi vel yfir \textbf{ferð} ferð\\
\hline
344&ferð&nveþ&Þeir fengu sér hesta og riðu austan úr Hornafirði og luku eigi ferð sinni fyrr en þeir komu í Fljótshlíð til \textbf{ferð} \\
\hline
345&frændi&nken&frændi \textbf{frændi} \\
\hline
346&frændi&nken&Hann var frændi þeirra \textbf{frændi} \\
\hline
347&frændi&nken&frændi \textbf{frændi} \\
\hline
348&frændi&nkfn&Þykir mér og á því líkindi að frændur yðrir og vinir muni mjög á það hlýða hvað þér talið fyrir þeim er þér komið út til \textbf{frændur} \\
\hline
349&frændi&nken&Þetta sumar sem nú var frá sagt riðu þeir Kormákur og Þorgils og Narfi frændi þeirra suður til Norðurárdals að erindum \textbf{frændi} \\
\hline
350&frændi&nken&Muntu svo eiga við að sjá að Össur frændi Orms mun sitja um líf þitt þá hann fréttir því að hann er höfðingi mikill og \textbf{frændi} \\
\hline
351&frændi&nken&« Svo líst mér frændi sem nú munum við hafa gert ráð okkað því að eg kann skapi \textbf{frændi} \\
\hline
352&frændi&nkeo&Vitið þið ekki um ráð Dálks frænda ykkars og vilduð við Björn enn \textbf{frænda} \\
\hline
353&frændi&nken&En Tanni hafði verið þingmaður Þorleifs \textbf{frændi} og frændi\\
\hline
354&frændi&nkfn&Fylgdi Eyjólfur Möðruvellingur þessum málum þeirra Brettings og Inga og margir menn með honum en Þorgeir goði mót honum og þeir allir frændur \textbf{frændur} \\
\hline
355&frændi&nkeo&Nú fer Vémundur að finna Áskel frænda sinn og ræðst um utanferðina \textbf{frænda} við hann\\
\hline
356&frændi&nkfn&Þessir voru allir frændur Gunnars og voru kappar \textbf{frændur} \\
\hline
357&frændi&nken&« Ill sending hefir komið til vor af þínu tilstilli þar sem er Hrolleifur frændi þinn og sitjum vér honum marga svívirðing og göngum því eigi frekt að að hann er þinn \textbf{frændi} \\
\hline
358&frændi&nken&Hann var frændi Þóris náinn og áþekkur honum í \textbf{frændi} \\
\hline
359&frændi&nken&Til ferðar réðust með honum þeir Úlfur hinn skjálgi frændi hans og Steinólfur hinn \textbf{frændi} \\
\hline
360&frændi&nkeo&« Far þú á fund Svarfdæla vina þinna og frænda eða hvað gafstu honum að \textbf{frænda} \\
\hline
361&frændi&nken&Var þar kominn Jökull Ingimundarson og Þórarinn frændi \textbf{frændi} \\
\hline
362&frændi&nkfo&Þá skyldi brenna alla dauða menn og Uppsölum þá gerðu margir höfðingjar eigi síður hauga en bautasteina til minningar um frændur \textbf{frændur} \\
\hline
363&frændi&nkfn&frændur \textbf{frændur} \\
\hline
364&frændi&nkfo&En þó gerðust deilur síðan í Færeyjum eftir víg Karls mærska og áttust þá við frændur Þrándar úr Götu og Leifur Össurarson og eru frá því stórar \textbf{frændur} \\
\hline
365&frændi&nken&« Svo þykir mér sem Hrolleifur láti eigi af sínum ferðum og þætti mér til þín koma Oddur frændi því að þú ert nú maður ungur og til alls vel fær en eg er örvasi fyrir \textbf{frændi} sakir\\
\hline
366&frændi&nken&frændi \textbf{frændi} \\
\hline
367&frændi&nken&að Magnús konungur frændi hans mundi bjóða honum sætt og félagsskap og Haraldur mundi hafa skulu hálfan Noreg við Magnús konung en hvor þeirra við annan hálft lausafé beggja \textbf{frændi} \\
\hline
368&frændi&nkfn&Urðu frændur hans honum fegnir þar sem þeir komu í \textbf{frændur} \\
\hline
369&frændi&nkfn&Hét Þorvaldur því að hann skyldi veita Sturlu við hvern sem hann ætti málum að skipta á Íslandi og skiljast aldrei við hann en Sturla hét í mót að veita Þorvaldi og setjast fyrir mál þau er Snorri og frændur hans höfðu á \textbf{frændur} \\
\hline
370&frændi&nkfn&« Eg firrði þig og næst á Hegranessþingi vandræðum sem von var að verða mundi ef þú sektir Þorvarð og frændur \textbf{frændur} \\
\hline
371&frændi&nkeo&Leifur spurði hvort það væri nokkuð vilji frænda \textbf{frænda} \\
\hline
372&frændi&nken&« Eg vil bjóðast til þess frændi að sætta ykkur Þóri og mæla til vinmæla í milli \textbf{frændi} \\
\hline
373&frændi&nkfo&Annan dag eftir gekk Ófeigur yfir brú og hittir frændur sína Skarðamenn og biður að þeir gangi með honum til Lögbergs og svo gera \textbf{frændur} \\
\hline
374&frændi&nken&Þorgeir hét frændi Þórdísar er hún hafði upp fætt og \textbf{frændi} \\
\hline
375&frændi&nken&Hann var frændi þeirra Vémundar og bjó í \textbf{frændi} \\
\hline
376&frændi&nken&« Hitt er ráð Kolbjörn frændi að liggja eigi lengur því að burt er Þórður farinn með Solrúnu og \textbf{frændi} félagar\\
\hline
377&frændi&nkfn&Finnbogi situr nú heima á Eyri og svo er sagt að menn verða nakkvað svo til áleitni við hann og eru mest að því synir Brettings og frændur \textbf{frændur} og vinir\\
\hline
378&frændi&nkfn&Bræður hans taka og með blíðu við honum og allir frændur \textbf{frændur} \\
\hline
379&frændi&nkeo&« og drepa Brynjólf frænda \textbf{frænda} \\
\hline
380&frændi&nken&« Svo er með vexti fóstbróðir að maður sá er kominn til mín er eigi þykir dæll í skaplyndi en hann er þó frændi minn og heitir \textbf{frændi} \\
\hline
381&frændi&nken&frændi Kolbeins \textbf{frændi} \\
\hline
382&fundur&nkfn&Nú kemur sumar og verða þá enn oftlega fundir \textbf{fundir} á laun\\
\hline
383&fundur&nkeo&Sigldi þá Hákon norður til Þrándheims og fór á fund Sigurðar Hlaðajarls er allra spekinga var mestur í Noregi og fékk þar góðar viðtökur og bundu þeir lag sitt \textbf{fund} \\
\hline
384&fundur&nkeo&Og er þeir komu á fund Marðar stóð hann upp í mót þeim og fagnaði þeim vel og bað þá \textbf{fund} \\
\hline
385&fundur&nkeo&» Eftir það reið Steinar upp í Reykjardal á fund Tungu-Odds og bað hann liðs og bauð honum fé \textbf{fund} \\
\hline
386&fundur&nkeo&Sótti Stúfur síðan norður til Kaupangs á konungs fund og tók konungur vel við honum og með samþykki hirðmanna gerðist Stúfur handgenginn konungi og var með honum nokkura \textbf{fund} \\
\hline
387&fundur&nkeo&Fór Hallvarður við það á fund jarls og sagði \textbf{fund} \\
\hline
388&fundur&nkeo&En er þeir bræður voru búnir þá fóru þeir norðan og komu á fund Haralds konungs er hann var í Þrándheimi og voru með honum um \textbf{fund} \\
\hline
389&fundur&nkeo&Dálkur fer á fund Þórðar Kolbeinssonar og sagði honum vígið og sakirnar og þótti Þórði mjög af sér hlotist hafa og bætti hann Dálki fébótum og tók við málinu til sóknar er eigi kæmu sættir á en Dálkur skyldi fylgja Þórði um eftirmál slíkt er hann \textbf{fund} \\
\hline
390&fundur&nkeo&Þeir Þrándur og Önundur komu á fund Eyvindar austmanns og tók hann vel við bróður \textbf{fund} \\
\hline
391&fundur&nkeo&« á fund Ölvis frænda \textbf{fund} \\
\hline
392&fundur&nkeo&Létti Þyri ferðinni eigi fyrr en þau koma á fund Ólafs \textbf{fund} \\
\hline
393&fundur&nkeo&Ella far þú aðra leið en þú fórst hingað þótt hún sé lengri nokkuru því að hinnar munu þeir gæta er þinn fund vilja hafa er skemmst er og alþýðuleið \textbf{fund} \\
\hline
394&fundur&nken&Eitt haust var fundur fjölmennur í Skörðum að tala um hreppaskil og ómegðir manna og var því skipt að \textbf{fundur} \\
\hline
395&fundur&nkeo&Hallkell tók við þessum mönnum og flutti með sér til Noregs og þegar á fund Sigurðar konungs með Harald og móður \textbf{fund} \\
\hline
396&fundur&nkeo&En um mál okkur Eiríks konungs er yður það að segja að eg var á hans fund og skildumst við svo að hann bað mig í friði fara hvert er eg \textbf{fund} \\
\hline
397&fundur&nkeo&« Þú skalt nú fara á fund konungs þess er þér gaf grið og tólf menn með \textbf{fund} \\
\hline
398&fundur&nkeo&Þórólfur fór um sumarið suður til Þrándheims á fund Haralds konungs og hafði þar með sér skatt allan og mikið fé annað og níu tigu manna og alla vel \textbf{fund} \\
\hline
399&fundur&nkeþ&að þú munt láta verða hérað þetta með litlum sóma eða sýna þig á einhverjum fundi ykkrum nokkuru óslæra en þú hefir fyrr \textbf{fundi} \\
\hline
400&fundur&nkeo&Þetta sumar hið næsta bjóst Eiríkur jarl Hákonarson úr landi vestur til Englands á fund Knúts konungs hins ríka mágs síns en hann setti eftir til ríkis í Noregi Hákon jarl son sinn og fékk hann í hendur Sveini jarli bróður sínum til forsjá og ríkisstjórnar því að Hákon var barn að \textbf{fund} \\
\hline
401&fundur&nkeo&» En er sendimenn komu á fund Önundar konungs þá báru þeir fram gjafir þær er Knútur konungur sendi honum og vináttu hans \textbf{fund} \\
\hline
402&fundur&nkeo&En er Kálfur kom á fund Knúts konungs þá fagnaði konungur honum forkunnar vel og hafði á tali við \textbf{fund} \\
\hline
403&fundur&nkeo&Þaðan fór hún á fund Leifs bróður síns og bað að hann gæfi henni hús þau er hann hafði gera látið á \textbf{fund} \\
\hline
404&fundur&nkeo&Kári fylgir þeim á fund jarls og sagði hverjir menn þeir \textbf{fund} \\
\hline
405&fundur&nkeo&Svall þetta sundurþykki svo að Heinrekur brá til utanferðar þetta sumar og kom á fund Hákonar \textbf{fund} \\
\hline
406&fundur&nkeo&En er Þorgils kom á fund konungs þá bar hann konungi kveðju Þórólfs og sagði að hann fór þar með finnskatt þann er Þórólfur sendi \textbf{fund} \\
\hline
407&fundur&nkeo&En Guðröður ljómi fór á fund Þjóðólfs hins hvinverska fósturföður síns og bað hann fara með sér til konungs því að Þjóðólfur var ástvinur \textbf{fund} \\
\hline
408&fundur&nkeo&Gissur reið vestur með flokkinn til móts við Kolbein og sendu þeir orð Böðvari til Staðar að hann færi á fund \textbf{fund} \\
\hline
409&fundur&nkeo&Voru margir menn í þeirri ferð gjarnir á líf Gissurar og ætluðu nú að eigi skyldi við bera að fund þeirra bæri \textbf{fund} \\
\hline
410&fundur&nken&Hélt Þórir eftir honum og varð fundur þeirra á hjalla \textbf{fundur} \\
\hline
411&fundur&nkeo&En allir fýstu Hörða-Knút að fara á fund föður \textbf{fund} \\
\hline
412&fundur&nkeo&« Sá er málavöxtur að eg ætla að fara á fund þeirra Gamla og \textbf{fund} \\
\hline
413&fundur&nkeo&Fór þá Björn með kaupmönnum austur í Garðaríki á fund Valdimars \textbf{fund} \\
\hline
414&fundur&nkeþ&Og var á þeim fundi ráðin utanferð hans og Hrafns \textbf{fundi} \\
\hline
415&fundur&nkeo&Þeir fóru á fund Eyjólfs og bjóða honum sátt fyrir þingmann \textbf{fund} \\
\hline
416&fundur&nkeo&Fær Ólafur sér hesta og sækir nú á fund Haralds konungs með sínu \textbf{fund} \\
\hline
417&fé&nheo&« Þenna grip vil eg hafa til míns bús en þú haf annað fé í \textbf{fé} \\
\hline
418&fé&nheþ&Hann var snauður að fé og eigi mjög vinsæll af alþýðu \textbf{fé} \\
\hline
419&fé&nheo&Var gefið fé til að þeir skyldu vera ferjandi en eiga eigi útkvæmt meðan nokkur Ólafssona væri á dögum eða Ásgeir \textbf{fé} \\
\hline
420&fé&nheo&Geitir spyr þetta og fer til fundar við Guðmund og býður honum að taka fé til \textbf{fé} \\
\hline
421&fé&nheo&« Það er mitt ráð að Gissur hinn hvíti og Hjalti Skeggjason varðveiti fé þetta til annars \textbf{fé} \\
\hline
422&fé&nheo&En er það kom upp allt saman þá lét Þórhallur kenna mannamunar og dró fé Þorsteins allt undir sig en hann var sjálfur lagður í \textbf{fé} \\
\hline
423&fé&nhen&Fé Hávarðar bónda var mjög óspakt um sumarið og einn morgun snemma kom smalamaður heim og spurði Ólafur hversu að \textbf{Fé} \\
\hline
424&fé&nheo&En Helgi kvaðst ekki eiga að gjalda honum og kvaðst eigi sjá að hann þyrfti fé að gefa í milli vinfengis \textbf{fé} \\
\hline
425&fé&nhen&« en það fé þeirra muni best komið er beinum þeirra fylgir og ef eg á á nokkuru ráð þá vil eg að svo \textbf{fé} \\
\hline
426&fé&nhen&Þetta fé sem nú var talið ráku þeir til Svínafells og sátu nú um kyrrt það sem eftir var sumars og höfðu þeir Ormssynir \textbf{fé} \\
\hline
427&fé&nhen&og var það ófa mikið fé svo að af því silfri lét Guttormur gera róðu eftir vexti sínum eða stafnbúa síns og er það líkneski sjö \textbf{fé} hátt\\
\hline
428&fé&nheo&Gáfu kertisveinar klokkurum fé til að hringja miklu fyrr en vant var og varð Halldór víttur og fjöldi annarra manna og settust í hálm um daginn og skyldu drekka \textbf{fé} \\
\hline
429&fé&nheþg&Nú skiptir Gísli fénu en Þorkell kýs meira lausafé til sín en Gísli hlaut \textbf{fénu} \\
\hline
430&fé&nheþg&Öndóttur kvaðst halda mundu fénu til handa Þrándi dóttursyni \textbf{fénu} \\
\hline
431&fé&nheþ&« Þið skuluð ríða austur í Hornafjörð eftir fé ykkru og koma heim um þing öndvert en ef þið eruð heima mun Gunnar vilja að þið ríðið til \textbf{fé} með honum\\
\hline
432&fé&nhen&Öll fé Þórðar hlaut Björn sem hann hafði þar nema skip en hver kaupmaður skal hafa sín fé er Björn hafði upp tekið \textbf{fé} \\
\hline
433&fé&nheo&Hann græddi þar af fé til þess er hann varð maður \textbf{fé} \\
\hline
434&fé&nheo&Tók Ingibjörg það fé allt til sín því að þau Auðgísl áttu ekki \textbf{fé} \\
\hline
435&fé&nheo&Tók Þorgautur fé þeirra allt og gersemar Ólafs \textbf{fé} \\
\hline
436&fé&nheo&Arngeir karl fór til Þorsteins Kuggasonar með mikið fé er hann tók við en Þórdís tók af mund sinn og heimanfylgju og fór vestur á Barðaströnd við Breiðafjörð til frænda \textbf{fé} \\
\hline
437&fé&nhen&Nú var honum boðið fé mikið að þiggja en hann lést þetta ekki til fjár hafa gert né til konu heldur af vinfengi við þá \textbf{fé} \\
\hline
438&fé&nhen&Vil eg ekki hark manna að fé mitt skemmist af vopnum \textbf{fé} \\
\hline
439&fé&nheo&Þess er getið hið síðasta sumar er þeir Sæmundur áttu fé saman og komu þá með miklu meira herfangi en fyrr þá gerðust þau tíðindi í Noregi að her safnaðist saman austur við Jaðar og var þá kominn nálega allur landher í tvo \textbf{fé} \\
\hline
440&fé&nhen&En er eyddist fé fyrir Karli þóttist hann ei búa mega að Upsum fyrir kostnaðar sakir og sakir Ljóts \textbf{fé} \\
\hline
441&fé&nhen&Eru þeir menn er fé hafa gefið til höfuðs honum og harmsakar átt að \textbf{fé} \\
\hline
442&fé&nheo&Eftir þingið safnar Kolbeinn liði um öll héruð og ætlar að heyja féránsdóm að Hólum eftir þá sem sekir voru kallaðir og taka upp fé \textbf{fé} \\
\hline
443&fé&nheo&» « Eigi vil eg fé ykkað bræðra hafa en ef ykkur þykir nokkurra launa vert vera þá takið þið mér far til Noregs því að mér er forvitni á að sjá þann konung er þar ræður fyrir er svo mikið er af \textbf{fé} \\
\hline
444&fé&nheo&hafa gefið það fé til þess að Þorgils væri heill og lífs og þó annað \textbf{fé} \\
\hline
445&fé&nhen&Fé var rekið norður til Kollafjarðar úr allri \textbf{Fé} \\
\hline
446&fé&nheþg&Hann kom það sumar til Danmerkur í Hróiskeldu og fékk mikið af fénu þó að miklir spænir væru af telgdir og fóru sunnan um sumarið er á leið en leið hans var um \textbf{fénu} \\
\hline
447&fé&nheo&Þeir fluttu heim fé sitt og réðu skipi til \textbf{fé} \\
\hline
448&fé&nheo&En Ívar dynta var leiddur á land upp og höggvinn því að þeir Sigurður og Gyrður Kolbeinssynir vildu eigi taka fé fyrir hann því að þeir kenndu honum að hann hefði verið að vígi Benteins bróður \textbf{fé} \\
\hline
449&fé&nheo&En Vigfús andaðist litlu síðar en hann kvongaðist og átti barn eitt og lifði það litlu lengur en hann og bar af því undir Hallfríði allt fé til helmings við þau Glúm og Ástríði en Eyjólfur var andaður þá er hér var \textbf{fé} \\
\hline
450&fé&nhen&» En er þessi orð komu til Þórðar frá Höskuldi og þar með stórar fégjafir þá sefaðist Þórður gellir og kvaðst það hyggja að það fé væri vel komið er Höskuldur varðveitti og tók við gjöfum og var þetta kyrrt síðan og um nokkuru færra en \textbf{fé} \\
\hline
451&fé&nhen&kvað það maklegast að það fé færi þeim til sáluhjálpar er aflað höfðu og til þeirrar kirkju er bein þeirra voru að \textbf{fé} \\
\hline
452&fé&nheo&Þar hittust þeir Helgi Ásbjarnarson og sættust á víg Þorgríms og lauk Þorkell fé \textbf{fé} \\
\hline
453&fé&nheþ&Vilja þeir þá að af fé því er fæst sé skipt að jafnaði milli skipanna fyrir utan kaupeyri þann er menn \textbf{fé} \\
\hline
454&fé&nheo&Þeim líkaði stórilla og beiddu bóta fyrir og kváðust sæst hafa við frændur hins vegna og grið tekið og síðan goldið fé fyrir \textbf{fé} \\
\hline
455&fé&nheo&Þeir tóku fé allt er heima var og langskip hálfþrítugt er Einar átti og son hans fjögurra vetra gamlan er lá hjá verkmanni \textbf{fé} \\
\hline
456&fé&nhen&Ríða nú vestur yfir Blöndu til Eiríks viðsjá og koma þar er fé var embætt að morgunmáli milli miðdegis og dagmála og hitta smalamann og spurðu hvort Eiríkur væri \textbf{fé} \\
\hline
457&fótur&nkfþ&Og er á leið kveldið og nóttina og morgna tekur þá vaknar Þorvarður og sér að Eysteinn var á fótum og \textbf{fótum} \\
\hline
458&fótur&nkeog&Þórhallur hjó þegar til Steinólfs og kom á \textbf{fótinn} \\
\hline
459&fótur&nkeþ&Ekki voru þeir vingaðir alþýðu manns en konungur mat þá mikils og voru þeir allra manna best færir bæði á fæti og á \textbf{fæti} \\
\hline
460&hestur&nkeog&Halli kastaði honum undir bakka og huldi hræ hans en hafði hestinn með \textbf{hestinn} \\
\hline
461&hestur&nkfo&Gæt þú hesta okkarra en eg mun ganga til dyngju \textbf{hesta} \\
\hline
462&hestur&nkeng&Þá var upp brotið húsið er bóndi kom til en hesturinn dreginn til dyra utar og lamið í sundur í honum hvert \textbf{hesturinn} \\
\hline
463&hestur&nkfo&Og nú sér hann hesta marga en vissi ekki von \textbf{hesta} \\
\hline
464&hestur&nkeog&» Þorgrímur hrökkvir hestinn og hleypir \textbf{hestinn} upp úr nesinu\\
\hline
465&hestur&nkfo&Síðan stíga þeir Guðmundur á hesta og ríða og er þeir koma á brekkurnar upp þá var þar fyrir flokkur þeirra \textbf{hesta} \\
\hline
466&hestur&nkeng&höggspjót mikið í hendi og lagði spjótinu fram millum eyrna hestinum og sá hann að spjótið tók lengra fram en hesturinn og svo \textbf{hesturinn} \\
\hline
467&hestur&nkfo&Hann biður menn stíga af baki og mælti að sumir skyldu geyma hesta þeirra en suma biður hann reisa \textbf{hesta} \\
\hline
468&hestur&nkeo&Litlu síðar lét Þorbjörg söðla sér hest og fór með annan mann til \textbf{hest} \\
\hline
469&hestur&nken&Gengur hestur Ingólfs betur í öllum \textbf{hestur} \\
\hline
470&hestur&nken&Grettir hljóp undir hömina á hesti sínum en rak stafinn á síðu Oddi svo hart að þrjú rifin brotnuðu í honum en Oddur hraut út á hylinn og svo hestur hans og hrossin öll þau er bundin \textbf{hestur} \\
\hline
471&hestur&nkeog&Vill Ólafur henda hross Þorgils og vill slá beisli við hestinn en Grímur safnar að hrossunum öðrum og á \textbf{hestinn} \\
\hline
472&hestur&nkeþ&« Það sýnist mér ráð að þú leggist niður til svefns en rís upp ofanverða nátt og stíg þá á bak hesti þínum og ríð vestur til \textbf{hesti} \\
\hline
473&hestur&nkeo&Síðan tók Þorsteinn hest þann er Böðvar hafði riðið og reið í brott og til \textbf{hest} \\
\hline
474&hestur&nkeþ&En um vorið þá er sumra tók færði Guðbrandur lið sitt í sel og var svo til skipað að húsfreyja reið ein saman en Guðbrandur og Svartur einum hesti báðir og reið Svartur að \textbf{hesti} \\
\hline
475&hestur&nkeo&að hann fékk sér hest og reið þegar á fund Vémundar og segir honum hvar nú er komið \textbf{hest} \\
\hline
476&hlutur&nken&hvergi er þeirra hlutur verður af \textbf{hlutur} \\
\hline
477&hlutur&nkeo&« Til þess hefir engi orðið fyrri en þú að skora mér á hólm svo skarðan hlut sem margur hefir fyrir mér borið og em eg þessa \textbf{hlut} \\
\hline
478&hlutur&nkeo&Það var og siður hans að láta löngum vera með sér göfugra manna sonu og setti þá svo ágætlega að þeir skyldu engan hlut eiga að iðja annan en vera ávallt í samsæti með \textbf{hlut} \\
\hline
479&hlutur&nkfþ&En Gunnhildur drottning lagði svo miklar mætur á hann að hún hélt engin hans jafningja innan hirðar hvorki í orðum né öðrum \textbf{hlutum} \\
\hline
480&hlutur&nkeo&» Nollar fer á burt og þykir sín för ill orðin og segir Helga því oft lítinn hlut mundu hafa fyrir Droplaugarsonum « ef þeim skal aldrei refsa \textbf{hlut} ókynni\\
\hline
481&hlutur&nken&« Ekki geri eg fyrir orð þín eða eggjan og svo vænti eg að konungur trúi eigi rógi þínu þó að honum mislíki nokkur hlutur til \textbf{hlutur} \\
\hline
482&hlutur&nkfo&Glúmur kveðst eigi það vita að hann ætti honum góða hluti að launa « en fyrir þá sök að þú sýnir þig vinlausan og lætur að hér liggi við líf þitt þá vil eg að þú farir norður til Þverár og bíðir þar minnar \textbf{hluti} \\
\hline
483&hlutur&nkfþ&Nú varð Hrafni rætt um sauðahvarfið og varð þann veg í orðum Þorleifs að honum þótti menn engan gaum að gefa slíkum hlutum « er ekki skal eftir leita þvílíkum hvörfum er svo eru \textbf{hlutum} \\
\hline
484&hlutur&nken&Sá var einn hlutur til þess er þeir Knútur og Hákon höfðu kyrru haldið um tilkall í Noreg að þá fyrst er Ólafur Haraldsson kom í land hljóp upp allur múgur og margmenni og vildi ekki heyra annað en Ólafur skyldi vera konungur yfir landi \textbf{hlutur} \\
\hline
485&hlutur&nkeo&Fékk Þorbjörn öngull þá mikinn hlut eyjarinnar með litlu verði en hann bast undir að koma Gretti á \textbf{hlut} \\
\hline
486&hlutur&nkfn&En í þessu vináttumarki er konungur gerði til Íslands bjuggu enn fleiri hlutir þeir er síðan urðu \textbf{hlutir} \\
\hline
487&hlutur&nkeo&Tóku þeir Þóri höndum en Viðar komst á braut og fór á Staðarhól til Einars fóstra síns og sagði honum svo búið og kveðst ætla að hann mundi vilja rétta hans hlut þá er þeir léku saman \textbf{hlut} \\
\hline
488&hlutur&nken&Þeir kváðust allir þess skyldir að hans hlutur lægi eigi í óvina \textbf{hlutur} \\
\hline
489&hlutur&nkfþ&Nú er sá okkar mestur vinur er til þess heldur að við séum æ sem sáttastir og jafnast haldnir í öllum \textbf{hlutum} \\
\hline
490&hlutur&nkfo&En er voraði þá lét Grímur enn sem fyrr innan handar allt það er hann átti um land eða aðra \textbf{hluti} \\
\hline
491&hlutur&nkfo&» Þórir segir það vel mega « því að marga hluti hefur þú vel til mín gert og ráð sjálf fyrir þessum \textbf{hluti} \\
\hline
492&hlutur&nkeo&« Þetta er líkast að þú hafir það helst af nafni því er þú ert eftir heitinn að hann vildi hvers manns hlut óhæfan af sér verða láta og það annað að menn þoli eigi og liggir þú drepinn er stundir \textbf{hlut} \\
\hline
493&hlutur&nkfo&Hann var manna hagastur og gervastur að sér um alla \textbf{hluti} \\
\hline
494&hlutur&nkfo&« Hér mun hann mælt mál hafa og vörumst vér að eigi verði svo þetta því að Gestur er sannspár um marga hluti og sé eg ráð til þessa að vér megum vel gera við \textbf{hluti} \\
\hline
495&hlutur&nken&Hann hrataði af garðinum og kom niður standandi og varð undir honum sá hlutur fótarins er af var \textbf{hlutur} \\
\hline
496&höfuð&nheo&Og þótti oss þó Haraldur konungur Gormsson vera minni fyrir sér en Uppsalakonungar því að Styrbjörn frændi vor kúgaði hann og gerðist Haraldur hans maður en Eiríkur hinn sigursæli faðir minn steig þó yfir höfuð Styrbirni þá er þeir reyndu sín á \textbf{höfuð} \\
\hline
497&höfuð&nheþ&Dálkur kallar og einsætt vera að neyta nú þess færis er hann hefir fátt manna og kvað þeim þungt vegist hafa við Björn og mundi mál þykja að eiga eigi hans ofsa yfir höfði sér ef réttast mætti og kvað Þórð skyldan til að beitast fyrir og skipa til « en aðrir að fylgja \textbf{höfði} \\
\hline
498&höfuð&nheþ&Þess er við að geta að höfði sá gekk einum megin hjá sundunum er Hofshöfði heitir og skyldi þar hittast lið konungsins allt við \textbf{höfði} \\
\hline
499&höfuð&nheog&» Þorvaldur leitaði þess á að hann skyldi ekki fleiri orð mæla þeim til óþurftar og höggur á hálsinn svo að af tók höfuðið og stakk höfðinu milli \textbf{höfuðið} \\
\hline
500&höfuð&nhfn&keisara öll skip sín og höfuð gullbúin voru á því skipi er konungur hafði \textbf{höfuð} \\
\hline
501&höfuð&nheog&Gunnbjörn hafði einn tygilkníf á hálsi er fóstra hans hafði gefið honum og með því að hann hafði ekki vopn til þá tekur hann þenna litla kníf og sker af Rauði höfuðið \textbf{höfuðið} \\
\hline
502&höfuð&nheo&« Oss þykir engi ofgæðakostur í að Haraldur liggi hér þótt hann dragi eigi hér þræla og stafkarla » og grípur hann upp riðvöl og laust í höfuð sveininum svo að blóð féll um \textbf{höfuð} \\
\hline
503&höfuð&nheþg&» Síðan setti hann sjóðinn á nasir Þorkels svo fast að brotnuðu tvær tennur úr höfðinu og stóðu blóðbogar úr andlitinu og gekk Þorkell burt við svo búið en Karl fór til sinna \textbf{höfðinu} \\
\hline
504&hönd&nvfo&Indriði lagði fyrstur hendur á Hörð og batt hendur hans heldur \textbf{hendur} \\
\hline
505&hönd&nveþg&Þá þreif Björn sporð skjaldarins hinni hendinni og rak í höfuð Þórði svo að hann fékk þegar \textbf{hendinni} \\
\hline
506&hönd&nvfþ&« að þá er þú hafðir hið fyrsta þing haft hér í Þrándheimi og höfðum þig til konungs tekinn og þegið af þér óðul vor að vér hefðum þá höndum himin \textbf{höndum} \\
\hline
507&hönd&nvfþ&» Tók hann þó til og svipti þegar í brott höndum \textbf{höndum} \\
\hline
508&hönd&nven&« að þig hendi það víti að þú komir eigi undir borð eða til kirkju þá skal þér framar upp gefa en \textbf{hendi} \\
\hline
509&hönd&nveþ&Hún hafði haldið honum svo fast að sér að hann mátti hvorigri hendi taka til nokkurs utan hann hélt um hana \textbf{hendi} \\
\hline
510&hönd&nveo&því að þá Svartur vissi sér minnst von höggur Helgi það högg til hans að sverðið kom á öxl honum og sneiddi af hans hægri hönd með \textbf{hönd} \\
\hline
511&hönd&nveþ&Þorvarður sótti á fund Ketils Kálfssonar og hafði í sinni hendi \textbf{hendi} \\
\hline
512&hönd&nvfo&Nú veistu að það er vandi þinn að fara á hendur þingmönnum þínum norður um sveitir á vorið með þrjá tigu manna og setjast að eins bónda sjö \textbf{hendur} \\
\hline
513&hönd&nveþ&Njósnarmenn Knúts konungs voru jafnan í her þeirra og komu sér í tal við marga menn og höfðu þeir fram féboð og vináttumál af hendi Knúts konungs en þar létu margir eftir leiðast og seldu þar til trú sína að þeir skyldu gerast menn Knúts konungs og halda landi honum til handa ef hann kæmi í \textbf{hendi} \\
\hline
514&hönd&nvfo&Og þá kallaði Brandur biskup til bóka og messufata í hendur honum og ollu því öfundarmenn hans en biskup kallaði staðinn að Hólum eiga arf eftir Ingimund \textbf{hendur} \\
\hline
515&hönd&nveo&Þá gekk Grímur að Helga og tók hönd Þórdísar af honum er hún hafði lagt yfir \textbf{hönd} \\
\hline
516&hönd&nvfþ&Þetta sumar ríða menn til þings og leggur Þorkell Geitisson fé til höfuðs Gunnari og fékk öllum höfðingjum umboð að hann skulu höndum \textbf{höndum} \\
\hline
517&hönd&nvfo&Nikulás flutti Harald til Björgynjar og fékk í hendur Erlingi \textbf{hendur} \\
\hline
518&hönd&nveo&Þórólfur bróðir hans gekk fram á aðra hönd honum og hlífði þeim báðum því Þorsteinn sá fyrir engu öðru en drepa allt sem fyrir \textbf{hönd} \\
\hline
519&hönd&nveþ&Hann hafði í hendi sverðið Ættartanga og gekk síðan að \textbf{hendi} \\
\hline
520&hönd&nvfþ&Sendið menn yðra þá er þér trúið til fundar við þá menn er þetta ráð hafa með höndum og freista ef þessi kurr mætti niður \textbf{höndum} \\
\hline
521&hönd&nvfþ&Einar faxi tók hann og ætlaði að fylgja honum til kirkjunnar en þeir tóku hann í höndum Einari og var hann þá \textbf{höndum} \\
\hline
522&hönd&nvfo&Eftir það er Finnbogi svo ákafur að hann höggur á tvær hendur og eigi létta þeir fyrr en fallnir eru fimm fylgdarmenn Jökuls en hann með öllu \textbf{hendur} \\
\hline
523&hönd&nvfo&» En með áeggjun þá bjó Þórarinn Þórisson mál til alþingis á hendur Glúmi um víg Sigmundar en Glúmur bjó mál til á hendur Þorkatli hinum háva um illmæli við þrælana og annað bjó hann á hönd Sigmundi og stefndi honum um stuld og kvaðst hann drepið hafa á eign sinni og stefnir honum til óhelgi er hann féll á hans eign og gróf Sigmund \textbf{hendur} \\
\hline
524&hönd&nvfo&Nú mun eg gera hann jarl minn og fá honum í hendur Danaveldi til yfirsóknar meðan eg em í Noregi svo sem Knútur hinn ríki setti Úlf \textbf{hendur} \\
\hline
525&hönd&nveþ&Þórður drap hendi hans af sér og varð laus og ætlaði að taka á \textbf{hendi} \\
\hline
526&hönd&nveþ&« Þakka viljum vér yður konungur er þér gefið oss góðan frið og þannig máttu oss mest teygja að taka við trúnni að gefa oss upp stórsakir en mælir til alls í blíðu þar sem þér hafið þann dag allt ráð vort í hendi er þér \textbf{hendi} \\
\hline
527&hönd&nveþ&sverðið \textbf{hendi} í hendi\\
\hline
528&hönd&nvfþ&Því söfnuðu bændur sér liði og tóku Gretti höndum og dæmdu hann til dráps og reistu honum gálga og ætluðu að hengja \textbf{höndum} \\
\hline
529&hönd&nvfo&Um sumarið eftir voru mál til búin á hendur Þorsteini Þorfinnssyni og varð hann sekur um víg \textbf{hendur} \\
\hline
530&hönd&nvfo&« en hví sætti það að Gunnar lýsti vígi Hjartar á hendur Kol þar sem Austmaðurinn vó \textbf{hendur} \\
\hline
531&hönd&nveog&Kári höggur til hans og kom sverðið á öxlina og varð höggið svo mikið að hann klauf frá ofan höndina og hafði Snækólfur þegar \textbf{höndina} \\
\hline
532&hönd&nvfo&En er til tók lag þeirra Ara og Úlfheiðar lét hún koma í hendur honum fimmtán hundruð þriggja alna aura til forráða og hafði hún þá eftir gullhring og marga gripi \textbf{hendur} \\
\hline
533&hönd&nveo&Á aðra hönd sat Þórhalla dóttir Ásgríms \textbf{hönd} \\
\hline
534&hönd&nvfo&« Mikill atburður er sjá orðinn og mikil vandkvæði hefir þessi maður ráðið sér og mikið er að takast á hendur skógarmann frænda Þorgríms og allra helst er hann hefir nú unnið svo mikið stórvirki í sekt \textbf{hendur} \\
\hline
535&hönd&nveþ&Þorgils Arason og Illugi af hendi \textbf{hendi} \\
\hline
536&hönd&nveo&Glúmur tók það til ráðs að hann fékk byrðing einn mikinn í hönd Þorsteini bróður sínum og skal hann halda vestur fyrir og koma til þings með herklæði og \textbf{hönd} \\
\hline
537&hönd&nvfþ&Þá reið að jarli Sæmundur Haraldsson og hjó til hans tveim höndum með breiðöxi og hugðist kjósa mundu á og ætlaði að höggva á \textbf{höndum} \\
\hline
538&hönd&nveo&« Þá vil eg til hlutast með þér og svo ráð setja að þú takir sök hverja er þú færð á hönd þingmönnum Þóris Helgasonar og mun það fé brátt \textbf{hönd} \\
\hline
539&hönd&nveþ&Fer Íslendingur er síðan var kallaður Þórarinn stuttfeldur og kemur að drykkjustofunni og stóð maður úti og hafði horn í hendi og mælti til \textbf{hendi} \\
\hline
540&hönd&nvfo&Þá drógu þeir hann milli sín til skógar og bundu hendur hans á bak \textbf{hendur} \\
\hline
541&hönd&nvfþ&þótt eigi sé vel í höndum haft þá er þeir sáu góða gripi er frændur þeirra höfðu átt og náðu \textbf{höndum} \\
\hline
542&hönd&nveþ&En þar kom um síðir að hann særði Lýting á hendi en drap heimamenn hans tvo og féll \textbf{hendi} \\
\hline
543&hönd&nveþ&Hann hafði í hendi bolöxi mikla á hávu \textbf{hendi} \\
\hline
544&hönd&nveþ&Gengu menn þá að Þorkatli presti og báðu hann bjóða þeim sæmdir fyrir fjörráð og fyrir ákomur þær er heimamenn hans höfðu veitt þeim Þorgilsi og Einari meðan þeir voru inni því að hvortveggi þeirra var skemmdur á hendi er þeir komu \textbf{hendi} \\
\hline
545&hönd&nvfþ&Ásgrímur þreif hana tveim höndum og hljóp upp á pallsstokkinn og hjó til höfuðs \textbf{höndum} \\
\hline
546&hús&nheo&« Í þessum stað skaltu láta smíða hús einum og sönnum guði til sæmdar eftir því móti sem eg mun sýna \textbf{hús} \\
\hline
547&hús&nhfþ&« Eigi vil eg spark annarra manna í húsum \textbf{húsum} \\
\hline
548&hús&nhfog&Og nú var það hans úrræði að hann safnaði saman öllu ganganda fé því er hann átti og rak inn í húsin og lagði eld í húsin og brenndi upp allt \textbf{húsin} \\
\hline
549&hús&nhen&Það var enn til jartegna að hvert hús var fullt af vopnum og \textbf{hús} \\
\hline
550&hús&nhfþ&Þá er flokkarnir riðu á staðinn voru biskupsmenn á húsum uppi og höfðu búist þar til \textbf{húsum} \\
\hline
551&hús&nheo&« Það eru ill tíðindi og mikil og þó verr að sönn séu því að eg sendi Kolbak inn í hús með veft og kom hann eigi heim í gærkveld og get eg að hann hafi eigi þorað að koma á minn fund því að hann vissi hvert vinfengi mér er til \textbf{hús} \\
\hline
552&jarl&nken&Þeir jarl fundust um haustið eftir því sem þeir höfðu áður rætt og sóru þá nokkurir bændur konungi trúnaðareiða og þeir sumir er áður höfðu mjög í móti \textbf{jarl} \\
\hline
553&jarl&nken&Ólafur konungur hafði þá verið konungur í Noregi fimmtán vetur með þeim vetri er þeir Sveinn jarl voru báðir í landi og þessum er nú um hríð hefir verið frá sagt og þá var liðið um jól fram er hann lét skip sín og gekk á land upp sem nú var \textbf{jarl} \\
\hline
554&jarl&nken&Þann vetur fór Rögnvaldur jarl hið innra um Eiðsjó suður í Fjörðu og hafði njósnir af ferðum Vémundar konungs og kom um nótt þar sem heitir Naustdalur og var Vémundur þar á \textbf{jarl} \\
\hline
555&jarl&nken&Gissur jarl reið norður til héraðs þegar eftir þessi \textbf{jarl} \\
\hline
556&jarl&nken&Nú kemur jarl til liðs síns og safnar saman flokki sínum og lætur grefta menn sína þá er fallið \textbf{jarl} \\
\hline
557&jarl&nken&En er hið fyrsta full var skenkt þá mælti Sigurður jarl fyrir og signaði Óðni og drakk af horninu til \textbf{jarl} \\
\hline
558&jarl&nken&Rögnvaldur jarl og Erlingur skakki komu í þeirri ferð til Jórsalalands og út til árinnar \textbf{jarl} \\
\hline
559&jarl&nken&Gerðist jarl þá hans maður og batt það \textbf{jarl} \\
\hline
560&jarl&nken&konungs þegar er hann kom til Englands og var jarl í öllum orustum \textbf{jarl} \\
\hline
561&jarl&nken&En er hann kom norður til Agðaness þá spurði hann að Hákon jarl er inn í firðinum og það með að hann var ósáttur við \textbf{jarl} \\
\hline
562&jarl&nken-s&» Jarl fýsti þess að Þorkell skyldi fara austur til Noregs á fund Ólafs \textbf{Jarl} \\
\hline
563&jarl&nken&En Sigurður jarl hafði ekki orðið var við sunnanferðina Erlings og var hann þá enn austur við Elfi en Hákon konungur var í \textbf{jarl} \\
\hline
564&jarl&nken&Þenna vetur bjóst Sigurður jarl til Írlands og þá barðist hann við Brján konung og hefir sú orusta frægust verið fyrir vestan \textbf{jarl} \\
\hline
565&jarl&nken&En þegar er Hákon jarl spurði það sendi hann menn til þeirra og vildi vita hvað manna væri á skipi þeirra er höfuðburður væri \textbf{jarl} \\
\hline
566&jarl&nken&Eftir andlát Végeirs varð Vébjörn ósáttur við Hákon \textbf{jarl} \\
\hline
567&jarl&nken&Víkina austur hafa þeir Tryggvi og Guðröður og hafa þeir þar nokkura tiltölu fyrir ættar sakir en Sigurður jarl ræður öllum Þrændalögum og veit eg það eigi hver skylda yður ber til þess að láta jarl einn ráða ríki svo mikið á önnur lönd en látið jarl innanlands taka af yður föðurleifð \textbf{jarl} \\
\hline
568&jarl&nken&Var þá spurt til sanns að Hákon jarl var týndur en land í Noregi var \textbf{jarl} \\
\hline
569&jarl&nken&« Eigi vill Eiríkur jarl nú berjast og hefna föður \textbf{jarl} \\
\hline
570&jarl&nken-s&» Jarl tjáði þá fyrir þeim fræknleik \textbf{Jarl} \\
\hline
571&jarl&nken&« Það vildi eg að þú værir heima á meðan eg færi til veislu þeirrar sem jarl hefir mér \textbf{jarl} \\
\hline
572&jarl&nken&Þá kemur jarl með liði sínu til Njálssona og spurði ef Hrappur hefði komið \textbf{jarl} \\
\hline
573&jarl&nken&Í þann tíma var Hákon jarl yfir Noregi og fór Sölmundur til hans og mat jarl hann \textbf{jarl} \\
\hline
574&jarl&nkfn&Þar voru áður jarlar synir Þorfinns \textbf{jarlar} \\
\hline
575&jarl&nken&Fóru þeir þegar á fund hans og sögðu honum hrakning sína og sýndu honum sár sín og kváðu þá mundi jarl í \textbf{jarl} \\
\hline
576&kona&nveþ&« Nú munum við hér skilja en hér eru tvö fingurgull er þú skalt selja í hendur Ingjaldi og konu \textbf{konu} \\
\hline
577&kona&nven&« að sjá kona kom til mín og batt á höfuð mér dreyruga húfu og þó áður höfuð mitt í blóði og jós á mig allan svo að eg varð \textbf{kona} \\
\hline
578&kona&nven&Kona hans hét \textbf{Kona} \\
\hline
579&kona&nven&Þá er þetta var tíðinda andaðist Þorbjörn súr og Ísgerður kona \textbf{kona} \\
\hline
580&kona&nveo&Hann kvað það aldrei skyldu lengur að gamall maður flekkaði svo væna konu og tók hana af honum og svo hest hans er Máni \textbf{konu} \\
\hline
581&kona&nveo&Þessi maður gat barn við konu en bræður hennar sóttu Kolbein að þessu \textbf{konu} \\
\hline
582&kona&nven&Sigríður kona \textbf{kona} \\
\hline
583&kona&nven&Hann var kvongaður maður og hét kona \textbf{kona} Þorbjörg\\
\hline
584&kona&nven&Ingibjörg kona jarls gekk að Hjalta og hvarf til \textbf{kona} \\
\hline
585&kona&nven&Og þá er þau höfðu þar verið nokkura hríð kom Helgi heim og Þórhalla kona \textbf{kona} \\
\hline
586&kona&nven&Hann var kvongaður maður og hét kona \textbf{kona} Hallbera\\
\hline
587&kona&nven&En á hinni sömu nótt bar Signý kona Öndótts á skip allt lausafé sitt og fór með sonu \textbf{kona} \\
\hline
588&kona&nvfn& \textbf{konur} og konur\\
\hline
589&kona&nvfn&Gísla þóttu fylgdarmenn Eyjólfs glepja konur þær er honum gast eigi að og gerðist með þeim hinn mesti \textbf{konur} \\
\hline
590&kona&nven&Og um jól um veturinn eða litlu fyrir jólin fellur þar voðmeiður og kemur einhver kona að Skútu og bað hann að gera og hann fer til þegar og þeir með honum veturtaksmenn \textbf{kona} \\
\hline
591&kona&nveþ&ef hún þjónaði svo Þuríði konu Þorsteins \textbf{konu} sem drottningu\\
\hline
592&kona&nven&Björgólfur var þá gamall og önduð kona hans og hafði hann selt í hendur öll ráð syni sínum og leitað honum \textbf{kona} \\
\hline
593&kona&nven&Í þenna tíma var Þórir viðleggur kominn á framfærslu til Fróðár og svo Þorgríma galdrakinn kona hans og lagðist heldur þungt á með þeim \textbf{kona} \\
\hline
594&kona&nvfn&Dyggvi var fyrst konungur kallaður sinna ættmanna en áður voru þeir drottnar kallaðir en konur þeirra drottningar en drótt \textbf{konur} \\
\hline
595&kona&nven&Guðrún kona Þorgils fæddi þá \textbf{kona} \\
\hline
596&kona&nven&En Gróa kona hans undi eigi hjá honum sakir skapsmuna sinna því að hún þóttist honum of góð og fór í hellisskúta einn og ruddi með \textbf{kona} \\
\hline
597&kona&nveo&Hefir hann eigi fíflt konu \textbf{konu} \\
\hline
598&kona&nvfn&Konur sögðu að hann hefði gengið til hrútahúss \textbf{Konur} \\
\hline
599&kona&nven&Ásta kona hans var ör og \textbf{kona} \\
\hline
600&kona&nven&Hann brá þá sverði en annarri hendi greip hann í skegg Ármóði og hnykkti honum á stokk fram en kona Ármóðar og dóttir hljópu upp og báðu Egil að hann dræpi eigi \textbf{kona} \\
\hline
601&kona&nven&Þá mælti ein kona að vorkunn væri á að henni þætti mikið að missa þvílíks \textbf{kona} \\
\hline
602&kona&nveþ&Þessar konu bað Helgi til handa Andríði og þessi konu var honum \textbf{konu} \\
\hline
603&kona&nven&kona Hrafns var ættuð úr \textbf{kona} \\
\hline
604&kona&nvfþ&Jarl sendir og eftir þeim konum sem hann vissi kurteisastar og lætur kenna dóttur sinni allar þær kvenlegar listir er burðugum konum byrjaði að kunna og það hugsaði jarl sem honum gafst að svo skyldi hans dóttir bera af öllum konum hannyrðir sem hún var hverri \textbf{konum} fríðari\\
\hline
605&kona&nveo&Þá kallaði Hrappur til sín Vigdísi konu sína og \textbf{konu} \\
\hline
606&kona&nvfo&Þeir Ingólfur höfðu með sér konur \textbf{konur} \\
\hline
607&kona&nvfn&» Hún skildi þó raunar hvað hann mælti til hjálpar manninum og þótti bæði skjótt og skörulegt hans úrræði og tók hún síðan búnaðinn af höfði sér og bjó hann með en settist í rúm hans að eigi gengju fleiri konur út en von \textbf{konur} \\
\hline
608&kona&nvfo&Þess er getið að Halldóra kona Glúms kvaddi konur með sér « og skulum vér binda sár þeirra manna er lífvænir eru úr hvorra liði sem \textbf{konur} \\
\hline
609&kona&nven&Guðrún hét kona \textbf{kona} \\
\hline
610&kona&nven&Vigdís kona Hrapps réðst vestur til Þorsteins surts bróður \textbf{kona} \\
\hline
611&kona&nven&Kona kom til fundar við Grís og hafði tvö börn meðferðar og beiddi Grís að hann mundi flytja börnin til \textbf{Kona} \\
\hline
612&kona&nvfn&Konur tvær voru innar frá Geiti og báðar fríðar og sat sú honum nær er eldri var og var sú kona hans en dóttir hans hin \textbf{Konur} \\
\hline
613&kona&nven&Gunnhildur kona hans var kvinna \textbf{kona} \\
\hline
614&konungur&nkeþ&Hann var þá með konungi og heyrði á ræður þeirra \textbf{konungi} \\
\hline
615&konungur&nken&Konungur nefndi menn til skipstjórnar og svo sveitarhöfðingja eða hvert hvergi sveit skyldi til \textbf{Konungur} \\
\hline
616&konungur&nken&Svínf Þá er Hákon konungur Hákonarson hafði þrjá tigu vetra ráðið Noregi kom Vilhjálmur kardínáli í Noreg og vígði Hákon konung undir \textbf{konungur} \\
\hline
617&konungur&nken&Ólafur konungur hafði fylkt þar er hæð nokkur var og steyptust þeir ofan á lið bónda og veittu svo hart athlaupið að fylking bónda bognaði fyrir svo að þar stóð þá brjóstið konungs fylkingar sem áður höfðu staðið þeir er efstir voru í búandaliði og var þá bóndaliðið mart búið að flýja en lendir menn og lendra manna húskarlar stóðu fast og varð þá allsnörp \textbf{konungur} \\
\hline
618&konungur&nkeo&hvort hann skyldi sættast við konung eða fara af landi brott og leita sér annarra forlaga en fréttin vísaði Þórólfi til \textbf{konung} \\
\hline
619&konungur&nken&Konungur lá þar nokkurar nætur og var veður hið \textbf{Konungur} \\
\hline
620&konungur&nken&Var þá sonur Guðbrands höndum tekinn og gaf Ólafur konungur honum grið og hafði með \textbf{konungur} \\
\hline
621&konungur&nken&Er hitt bæn mín og vilji að þér konungur farið að heimboði til mín og heyrið þá orð þeirra manna er þú \textbf{konungur} \\
\hline
622&konungur&nken&Aun konungur fékk andsvör af Óðni að hann skyldi enn lifa sex tigu \textbf{konungur} \\
\hline
623&konungur&nken&Og er þeir jarl og hans förunautar voru komnir í einnhvern skóg og voru eigi langt farnir frá því er þeir konungur fundust og þar tekur jarl sótt mikla og hættlega og þegar sendir hann menn aftur á fund Ólafs konungs til þess að hann vildi að konungurinn kæmi þá að hitta \textbf{konungur} \\
\hline
624&konungur&nken&en konungur gekk til hans og bar hann undir hendi sér ofan á \textbf{konungur} \\
\hline
625&konungur&nken&Öndurðan vetur þann er Ólafur konungur sat í Niðarósi lýsti konungur því að hann vill menn senda til \textbf{konungur} að heimta skatt\\
\hline
626&konungur&nken&Þá var Hreiðar skotinn spjóti milli herðanna og þar í gegnum en svo segja menn að þar fengi Magnús konungur bana af því sama spjóti og féll Hreiðar á bak aftur á þiljurnar en Magnús á hann \textbf{konungur} \\
\hline
627&konungur&nken&Þá mælti konungur til þeirra \textbf{konungur} \\
\hline
628&konungur&nken&Það vitni bar Haraldur konungur Halldóri að hann hefði verið með honum allra manna svo að síst brygði við voveiflega hluti hvort sem að höndum bar mannháska eða fagnaðartíðindi þá var hann hvorki að glaðari né \textbf{konungur} \\
\hline
629&konungur&nken&« Máttkum guði stýrir þú herra konungur og fyrir því að eg hefi það reynt þá vil eg nú á hann trúa og láta \textbf{konungur} \\
\hline
630&konungur&nken&Sigvaldi jarl lagðist þá ferð eigi undir höfuð og fer á fund Sveins Danakonungs og ber þetta mál fyrir hann og kemur jarl svo fortölum sínum að Sveinn konungur fær í hendur honum Þyri systur sína og fylgdu henni konur nokkurar og fósturfaðir hennar er nefndur er Össur \textbf{konungur} \\
\hline
631&konungur&nkeo&Páll biskup fór fyrir því út til páfa að hann varð missáttur við Hákon \textbf{konung} \\
\hline
632&konungur&nken&En er Þórólfur hafði heiman farið þá hafði hann til greitt finnskatt þann allan er hann hafði haft af fjalli og konungur átti og fékk í hendur Þorgísli og bað hann færa konungi ef hann kæmi eigi heim áður um það er konungur færi norðan og suður \textbf{konungur} \\
\hline
633&konungur&nkeþ&En miklu best leist mér þó í dag á hann og öll ætla eg oss þar við liggja vor málskipti að vér trúum þann vera sannan guð sem konungur býður og fyrir engan mun má konungi nú tíðara til vera að eg taki við trúnni en mér er að láta skírast og það eina dvelur er eg geng nú eigi þegar á konungs fund er framorðið er dags því að nú mun konungur yfir borðum vera en sá dagur mun dveljast er vér sveitungar látum allir \textbf{konungi} \\
\hline
634&konungur&nkeo&gekk á orð sín við þann milda konung og hljópst þá á brott um dag og kom að kveldi til bónda \textbf{konung} \\
\hline
635&konungur&nken&Gekk Haki konungur fram svo hart að hann felldi alla þá er honum voru næstir og að lyktum felldi hann Eirík konung og hjó niður merki þeirra \textbf{konungur} \\
\hline
636&konungur&nken&Önundur konungur tók þeim ræðum ekki fljótt og þóttust sendimenn það á finna að Önundur konungur mundi vera mjög snúinn til vináttu við Ólaf \textbf{konungur} \\
\hline
637&konungur&nken&En Geirviður konungur lét það þá í ljós við Dagfinn að hann skyldi hans sóma meira gera í alla staði heldur en hvers manns annars í sínu ríki og bauð honum það að hann mundi afla honum kvonfangs og sagði svo að hann mundi þá konu fá honum til handa er hann vildi helst \textbf{konungur} \\
\hline
638&konungur&nken&þá leituðu þeir til brautsiglingar þegar konungur var í bænum því að þeim var sagt að konungur nauðgaði alla menn til \textbf{konungur} \\
\hline
639&konungur&nken&En er konungur spurði ófrið þann þá lét hann njósnum til halda um ferðir \textbf{konungur} \\
\hline
640&konungur&nkeþ&Einar þambarskelfir mágur hans var með honum og margir aðrir lendir menn og margir þeir er áður um veturinn höfðu trúnaðareiða svarið Ólafi \textbf{konungi} \\
\hline
641&konungur&nkeþ&Þá tóku þeir heit sitt með handfesti og hétu því að reisa kirkju í Miklagarði með sínum kostnaði og góðra manna stoðum og láta þá kirkju vígja til vegs og dýrðar hinum helga Ólafi \textbf{konungi} \\
\hline
642&konungur&nken&Eiríkur konungur lét leggja eld í Hallvarðskirkju og víða um býinn og mikið en Eiríkur konungur lagði í brott skipaliði sínu og máttu þeir hvergi á land koma fyrir norðan fjörðinn fyrir safnaði lendra manna en hvar sem þeir leituðu til landgöngu þá lágu eftir fimm eða sex eða \textbf{konungur} \\
\hline
643&konungur&nken&Eftir það kom Aun konungur enn til \textbf{konungur} \\
\hline
644&konungur&nken-s&» Konungur mælti til \textbf{Konungur} \\
\hline
645&konungur&nken&Knútur konungur launaði kvæðið fimm tigum marka \textbf{konungur} \\
\hline
646&konungur&nken&Lagði Ólafur konungur þar til \textbf{konungur} \\
\hline
647&konungur&nken&Fám dögum síðar kom Haraldur konungur til jarls og taka þeir þá \textbf{konungur} \\
\hline
648&konungur&nken&En er Önundur konungur varð þessa var og þá lýsti af degi þá lét hann blása til \textbf{konungur} \\
\hline
649&konungur&nkeþ&Er og frændsemi með okkur Ólafi konungi en þótt hann virði lítils það við mig þá samir mér þó eigi að ganga framar í þenna ófrið fallinn að vera höfuðsmaður að halda bardaga við Ólaf \textbf{konungi} \\
\hline
650&konungur&nken&Ívar tekur nú ógleði mikla og er konungur fann það heimti hann Ívar til máls við sig og spurði hví hann væri svo ókátur « og fyrr er þér voruð með oss var margs konar skemmtan að yðrum \textbf{konungur} \\
\hline
651&konungur&nken&En þeim hinum færeyskum virtist svo orð konungs sem grunur mundi á vera hvernug þeirra mál mundi snúast ef þeir vildu eigi undir það allt ganga sem konungur beiddi \textbf{konungur} \\
\hline
652&konungur&nken&En er konungur kom til þings þá var kominn bóndamúgurinn með \textbf{konungur} \\
\hline
653&konungur&nkeo&Barði gengur fyrir konung og hans \textbf{konung} \\
\hline
654&konungur&nkeo&Þeir báðu konung orlofs til að þeir bræður skyldu fara með liði því er vant var þeim að fylgja og sækja heim að \textbf{konung} \\
\hline
655&konungur&nkeo&En er þeir komu fyrir konung þá tók Þorgrímur sinni hendi hvorn þeirra bræðra og veik þeim frá sér og gekk fram á millum þeirra og sté upp á fótskörina og kvaddi konung og hvarf til hans en konungur tók sveininn brosandi og setti hann niður hjá sér og frétti hann að móðurætt \textbf{konung} \\
\hline
656&konungur&nken&að eg vil miklu heldur vera lendur maður hans en heita konungur og væri annar samlendur við mig sá er mig mætti gera að þræli sér ef \textbf{konungur} \\
\hline
657&konungur&nken&Haraldur konungur lét reisa af grundvelli \textbf{konungur} uppi á melinum\\
\hline
658&konungur&nkeþ&Síðan tjáði Finnur fyrir Hákoni hversu mikill munur honum var að betra var að taka af konungi svo mikil metorð sem hann kynni sjálfur að beiða heldur en hætta til þess að reisa orustu í móti konungi þeim er hann var áður þjónustubundinn \textbf{konungi} \\
\hline
659&konungur&nken&Í þeim sama stað lét Ólafur konungur Tryggvason efna til \textbf{konungur} þar sem nú er\\
\hline
660&konungur&nken&En það var mjög jafnfram að Magnús konungur kom sunnan úr Danmörk norður til Þrándheims og Þórður dó af sári því er Kolgrímur veitti \textbf{konungur} \\
\hline
661&konungur&nkeo&bað guð skipta milli þeirra og hinn helga \textbf{konung} konung\\
\hline
662&konungur&nkeþ&Sá biskup var í her þessum og talaði oftlega fyrir búandaliði og eggjaði mjög uppreistar móti Ólafi \textbf{konungi} \\
\hline
663&konungur&nken&En er konungur sá að þeir vildu ekki til hlýða þess er hann kenndi þeim og það annað að þeir höfðu þann múg manns er ekki stóðst við þá sneri hann \textbf{konungur} \\
\hline
664&konungur&nken&Konungur gerði orðsendingar til vina \textbf{Konungur} í Færeyjar\\
\hline
665&konungur&nkeþ&» Gestur þakkaði konungi sín ummæli og segist það munu \textbf{konungi} \\
\hline
666&konungur&nkeo&Þess getur Hallfreður í einni drápu er hann orti um Ólaf \textbf{konung} \\
\hline
667&konungur&nken&Jarl bað þá bíða þess er konungur gengi að og þeir stæðu allir jafn \textbf{konungur} \\
\hline
668&konungur&nken&falar skipið að Halldóri og vill að hann hafi verð sæmilegt en Sveinn hafi skip og kaupir konungur skip og á Halldór við hann um verð og gelst allt upp nema hálf mörk gulls stendur \textbf{konungur} \\
\hline
669&konungur&nken&Sögðu þeir að af þessi orðsending þótti þeim heldur grunir á dregnir um það er Einar hafði getið að konungur mundi ætla til pyndinga nokkurra við Íslendinga ef hann mætti \textbf{konungur} \\
\hline
670&konungur&nken&Aðalsteinn konungur hafði skattgilt undir sig Skotland eftir fall Ólafs konungs en þó var það fólk jafnan ótrútt \textbf{konungur} \\
\hline
671&konungur&nkeþ&Þar var og Þjóðólfur skáld með konungi og þótti vera nokkuð öfundsjúkur við þá menn er komu til \textbf{konungi} \\
\hline
672&konungur&nkeþ&Þótti konungi og hans mönnum þá vænt um sína \textbf{konungi} \\
\hline
673&konungur&nkeþ&En eftir um vorið þá kærði Þyri drottning oft fyrir Ólafi konungi og grét sárlega það er eignir hennar voru svo miklar í Vindlandi en hún hafði eigi fjárhlut þar í landi svo sem drottningu \textbf{konungi} \\
\hline
674&konungur&nken&Finnur Árnason bróðir hans gerði orð Kálfi og lét segja honum einkamál þau er þeir Haraldur konungur höfðu við mælst að Kálfur skyldi hafa landsvist í Noregi og eignir sínar og slíkar veislur sem hann hafði haft af Magnúsi \textbf{konungur} \\
\hline
675&konungur&nkeo&En er Þrændir vissu að landsmenn veittu þeim ámæli þá könnuðust þeir við að það var sannmæli og þá hafði hent glæpska mikil er þeir höfðu Ólaf konung tekið af lífi og láði og það með að þeim var sín óhamingja miklu illu \textbf{konung} \\
\hline
676&konungur&nkeþ&Ólafur var níu vetra er hann kom í Garðaríki en dvaldist þar með Valdimar konungi aðra \textbf{konungi} vetur\\
\hline
677&konungur&nken&Er yður konungur vandsettir hér menn yfir til forráða því að þér munuð hér sjaldan koma \textbf{konungur} \\
\hline
678&konungur&nken&Þetta sama kveld sendir konungur menn til herbergis Íslendinga og bað þá verða vísa hvað þeir \textbf{konungur} \\
\hline
679&konungur&nken&» Þessi ráðagerð var framgeng að Ólafur konungur fór til skipa \textbf{konungur} \\
\hline
680&konungur&nkeo&Síðan fór Egill með sveit sína á fund Aðalsteins konungs og gekk þegar fyrir konung er hann sat við \textbf{konung} \\
\hline
681&konungur&nken&Konungur nam staðar og stöðvaði her sinn er hann kom á \textbf{Konungur} \\
\hline
682&konungur&nken&Sveinn konungur bauð honum til sín og segir að hann skal fá jarlsríki í \textbf{konungur} \\
\hline
683&konungur&nken&« Hví mun eigi það til að þú farir leið þína sem þú vilt og kom þá til mín er þú ferð aftur og seg mér hversu Sveinn konungur launar þér \textbf{konungur} \\
\hline
684&konungur&nken&En ef þar væru nokkur önnur efni í og væru sakir milli okkar Þórálfs þá em eg svo viti borinn að eg mundi heldur til þessa verks hætta heima í Færeyjum en hér undir handarjaðri yðrum \textbf{konungur} \\
\hline
685&konungur&nken&Ólafur konungur hjó til Þóris \textbf{konungur} um herðarnar\\
\hline
686&konungur&nken&Konungur segir að jarl skal vera með honum ef hann vill það og hafa þar ríki til forráða það er honum þyki sæmilegt « og að öðrum \textbf{Konungur} \\
\hline
687&konungur&nkeþ&Aron var hirðmaður Hákonar konungs og var kær \textbf{konungi} \\
\hline
688&konungur&nken&Segir hann að Hákon konungur hefði spurt dráp Eyjólfs \textbf{konungur} \\
\hline
689&konungur&nken&Gekk konungur til kirkju og hlýddi \textbf{konungur} \\
\hline
690&konungur&nkeþg&En Dagfinnur vildi eigi hringinn þiggja og sagði svo að honum var mikil öfúsa á því að hafa sóma og virðing af konunginum en fé kvaðst hann eigi þurfa að þiggja af honum og kallaði sig ekki skorta meðan hann héldi honum heilum « en þeir eru margir aðrir er þar sjá til fjárins sem þér \textbf{konunginum} \\
\hline
691&konungur&nken&Þann dag bauð Haraldur konungur Magnúsi konungi til borðs síns og gekk hann um daginn með sex tigu manna til landtjalda Haralds konungs þar sem hann hafði veislu \textbf{konungur} \\
\hline
692&konungur&nkeþ&Biskup fór að skrifta þeim og sagði konungi að engi voru svik af þeirra hendi og beiddi konung að skírsla væri ger um þetta mál og bæri guð vitni um þetta hvorir satt \textbf{konungi} \\
\hline
693&konungur&nken&« Þakka viljum vér yður konungur er þér gefið oss góðan frið og þannig máttu oss mest teygja að taka við trúnni að gefa oss upp stórsakir en mælir til alls í blíðu þar sem þér hafið þann dag allt ráð vort í hendi er þér \textbf{konungur} \\
\hline
694&konungur&nken&Og er Arinbjörn varð þess var þá gekk hann með alla sveit sína alvopnaða í konungsgarð þá er konungur sat yfir \textbf{konungur} \\
\hline
695&konungur&nken&þá er konungur var klæddur var hann fámálugur og ókátur og hræddust vinir hans að þá mundi enn að honum komið \textbf{konungur} \\
\hline
696&konungur&nkeþ&Kom það í einkamál með konungi og jarli að eignir þær í Vindlandi er átt hafði Gunnhildur drottning skyldi þá hafa Þyri til eiginorðs og þar með aðrar stórar eignir í tilgjöf \textbf{konungi} \\
\hline
697&konungur&nken&En er Ólafur spurði að Knútur konungur fór liði sínu norður fyrir land þá hélt Ólafur konungur inn í Óslóarfjörð og upp í vatn það er Drafn heitir og hafðist hann þar við til þess er her Knúts konungs var farinn um \textbf{konungur} \\
\hline
698&konungur&nken&» segir konungur og gaf honum gullhring er stóð sex aura « en því skaltu heita \textbf{konungur} \\
\hline
699&konungur&nken&Þá hafði Magnús konungur og aðrir frændur Jóns tekið við frændsemi \textbf{konungur} \\
\hline
700&konungur&nken&segir að hún vill að konungur sjái fyrir hennar ráði slíkt sem hann \textbf{konungur} \\
\hline
701&konungur&nken&Magnús konungur skyldi fá Margrétar dóttur Inga \textbf{konungur} \\
\hline
702&konungur&nken&En Ólafur sænski konungur er kallar til \textbf{konungur} \\
\hline
703&konungur&nken&Í þenna tíma réð fyrir Danmörku Knútur hinn ríki Sveinsson og hafði nýtekið við föðurleifð sinni og heitaðist jafnan að herja til Englands fyrir því að Sveinn konungur faðir hans hafði unnið mikið ríki á Englandi áður hann andaðist vestur \textbf{konungur} \\
\hline
704&konungur&nkeþ&Búendur höfðu engi styrk til þess að halda ósætt við konung og lauk svo að þeir veittu konungi viðurtöku og bundu það \textbf{konungi} \\
\hline
705&konungur&nken&» Og nú gaf konungur honum silfur mjög mikið og fór hann suður síðan með Rúmferlum og skipaði konungur til um ferð \textbf{konungur} \\
\hline
706&konungur&nken&að Noregur mundi liggja laus fyrir ef nokkurir stórir höfðingjar vildu til sækja er engi var konungur yfir landinu og lendra manna forráð var þar yfir ríkinu en þeir lendir \textbf{konungur} \\
\hline
707&konungur&nken&» Haraldur konungur skildi nú að þetta var með spotti gert en hann vildi einskis manns þegn \textbf{konungur} \\
\hline
708&konungur&nken&Þórður segir það vera sættarboð af hendi Þorgils að hann vildi halda og hafa Borgarfjörð í friði og frelsi og öll þau ríki sem konungur hafði honum \textbf{konungur} \\
\hline
709&konungur&nken&« Til annars hugði eg af yður konungur en þér munduð gera mig lögræning þá er eg settist í Kvalðinsey er fáir vildu aðrir vinir yðrir og sögðu sem satt var að þeir voru fram seldir er þar sátu og til dauða dæmdir ef Ingi konungur hefði eigi lýst við oss meira höfðingskap en þú hafðir fyrir oss séð og mun þó mörgum sýnast sem vér bærum þaðan svívirðing ef það væri \textbf{konungur} vert\\
\hline
710&konungur&nken&Magnús konungur gaf honum síðan knörr með farmi og gerðist mikill ástúðarvinur \textbf{konungur} \\
\hline
711&konungur&nkeþ&Þá voru fyrir vestan haf margir ágætir menn þeir sem flúið höfðu óðul sín úr Noregi fyrir Haraldi konungi því að hann gerði alla \textbf{konungi} \\
\hline
712&konungur&nken&Knútur konungur bað þau orð segja jarli að hann safnaði her og skipum og færi svo til fundar við konung en ræddi síðan um sættir \textbf{konungur} \\
\hline
713&konungur&nkeþ&Og er Þórður hafði verið þrjá vetur með Gamla konungi sagði hann konungi að hann fýstist að vitja eigna \textbf{konungi} \\
\hline
714&konungur&nken&Og er hún bjóst í brott þá sá konungur ferð \textbf{konungur} og mælti\\
\hline
715&konungur&nkeþ&« Fara suður til Danmerkur og gefa Sveini \textbf{konungi} \\
\hline
716&konungur&nken&kveldið þá mataðist hann að náttverði en þá klæddist hann og húskarlar hans og fór ofan til vatns og tók karfann er Ketill átti er Ólafur konungur hafði gefið \textbf{konungur} \\
\hline
717&konungur&nken&Aðalsteinn konungur sneri í brott frá orustunni en menn hans ráku \textbf{konungur} \\
\hline
718&konungur&nken&Ingjaldur konungur var þá staddur á Ræningi að veislu er hann spurði að her Ívars konungs var þar nær \textbf{konungur} \\
\hline
719&konungur&nkeo&segja svo ef hann helst í vináttu við konung að honum mun þá allt auðveldlegt að koma slíku fram sem hann vildi við hvern \textbf{konung} \\
\hline
720&konungur&nkeþ&Voru þeir allir ráðnir til útferðar og beiddust \textbf{konungi} af konungi\\
\hline
721&konungur&nken&svo það að Haraldur konungur veitti honum sár það er ærið mundi eitt til bana og Þórólfur féll nær á fætur konungi á \textbf{konungur} \\
\hline
722&konungur&nkeþ&En Þorvaldur fór til Íslands með mikilli sæmd af Ólafi konungi og þótti mikilmenni og hinn \textbf{konungi} \\
\hline
723&konungur&nken&Svo er sagt að það var einn dag snemma um vorið að konungur gekk eftir stræti en við torg gekk maður í móti honum með hvannir margar og undarlega stórar þann tíma \textbf{konungur} \\
\hline
724&konungur&nken&að fylgja konungi innan lands og utan lands ef konungur vill þess \textbf{konungur} \\
\hline
725&konungur&nken&Nú mun þér það ekki hér tjá því að Eiríkur konungur og Gunnhildur drottning hafa mér því heitið að eg skuli rétt hafa af hverju máli þar er þeirra ríki stendur \textbf{konungur} \\
\hline
726&konungur&nkeo&Fór Arinbjörn heim og sagði Agli erindislok sín « mun eg eigi slíkra \textbf{konung} oftar leita við konung\\
\hline
727&konungur&nken&» Hann skildi þann draum til þess að hann mundi konungur vera yfir landi og hans ættmenn langa \textbf{konungur} \\
\hline
728&konungur&nken&Hálfdan konungur fór þá með her sinn á Heiðmörk eftir honum og áttu þeir þar aðra orustu og hafði Hálfdan sigur en Eysteinn flýði norður í Dala á fund Guðbrands \textbf{konungur} \\
\hline
729&konungur&nken&» Og þegar um morguninn snemma er konungur gekk til kirkju mætti Kjartan honum á strætinu með mikilli sveit \textbf{konungur} \\
\hline
730&konungur&nken&» Eftir viðurtal þeirra bræðra gera þeir allir bræður heiman ferð sína með mikinn flokk manna og stefna á Upplönd þangað sem þeir spurðu að Sigurður konungur var á \textbf{konungur} \\
\hline
731&konungur&nkeþ&Leitið yður nú heldur vaskra manna dæma þeirra er vel fylgdu Sverri konungi eða öðrum \textbf{konungi} \\
\hline
732&konungur&nken&En litlum tíma síðar en Þórður fór í burt frá Gamla konungi börðust þeir Hákon konungur hinn góði og Gamli konungur og í þeirri orustu féll Gamli konungur sem segir í sögum \textbf{konungur} \\
\hline
733&konungur&nkeo&slíks sem hann hafði bundið fyrr við Magnús \textbf{konung} \\
\hline
734&konungur&nken&Konungur sigldi hraðbyrja til þess er hann kom í \textbf{Konungur} \\
\hline
735&konungur&nken&» Bárður hittir Halldór og lætur að konungur vilji einskis kostar láta hans þjónustu og það ræðst úr að Halldór fer og halda þeir konungur suður með \textbf{konungur} \\
\hline
736&konungur&nkeþ&Hann skemmtir oft konungi og er það frá sagt að hann skemmti allra manna best og oft kvæði hann vísur um það er við \textbf{konungi} \\
\hline
737&konungur&nken&En er Kálfur kom á fund Knúts konungs þá fagnaði konungur honum forkunnar vel og hafði á tali við \textbf{konungur} \\
\hline
738&konungur&nken-s&» Konungur nam \textbf{Konungur} og svaraði\\
\hline
739&konungur&nken&Hrollaugur konungur fór upp á haug þann er konungar voru vanir að sitja á og lét þar búa hásæti konungs og settist þar \textbf{konungur} \\
\hline
740&konungur&nken&Eftir um morguninn lét konungur kalla til sín steikara og þann er drykkinn varðveitti og spyr ef nokkur ókunnur maður hefði komið til \textbf{konungur} \\
\hline
741&konungur&nken&Um kveldið þá spyr konungur son Guðbrands hvernug goð þeirra væri \textbf{konungur} \\
\hline
742&konungur&nkeo&að smyrja konung til \textbf{konung} \\
\hline
743&konungur&nken&Fór þá Ingi konungur inn til Óslóar en Gregoríus var í \textbf{konungur} \\
\hline
744&konungur&nken&» En er sendimenn komu á fund Önundar konungs þá báru þeir fram gjafir þær er Knútur konungur sendi honum og vináttu hans \textbf{konungur} \\
\hline
745&konungur&nken&Hákon konungur lét þá Gissur og Þórð kæra mál sín svo að kardínálinn var við og lét tjá honum alla málavöxtu \textbf{konungur} \\
\hline
746&konungur&nken&að þeir skyldu aftur snúa en konungur og annað liðið beið þeirra og skipuðu liði sínu til \textbf{konungur} \\
\hline
747&konungur&nkfn&» Er það og sögn manna að flestir konungar hafi það varast \textbf{konungar} \\
\hline
748&konungur&nkeþ&faðir Þóris er var á Fitjum með Hákoni konungi og skar rauf á húð og hafði það fyrir \textbf{konungi} \\
\hline
749&konungur&nken&Hafði hann þá spurt að Magnús Ólafsson bróðurson hans var orðinn konungur í Noregi og svo í \textbf{konungur} \\
\hline
750&konungur&nken&að leita um sættir en fyrir sakir frændsemi gaf Hálfdan konungur upp Eysteini konungi hálfa Heiðmörk svo sem þeir frændur höfðu fyrr \textbf{konungur} \\
\hline
751&konungur&nken&Tók þá konungur undir sig eignir hans og rak á brutt Björn son \textbf{konungur} \\
\hline
752&konungur&nken&systur Þóru er Haraldur konungur \textbf{konungur} \\
\hline
753&konungur&nken&Ber Sigtryggur konungur þegar upp erindi sín en Bróðir skarst undan allt til þess er Sigtryggur konungur hét honum konungdómi og móður \textbf{konungur} \\
\hline
754&konungur&nken&» Þakkar konungur guði sending sína og lét hann gera byrðar matar bændum þeim er ofan fóru eftir \textbf{konungur} \\
\hline
755&kostur&nken&Halldór segir þá Þórði hvers af var \textbf{kostur} \\
\hline
756&kostur&nken&Þótti sá kostur ágætagóður en forráð hennar hafði þá \textbf{kostur} \\
\hline
757&kostur&nkeþ&Það var síðar um haustið að Sturla hafði njósn af að Aron væri að Valshamri á kosti Vigfúss er þar \textbf{kosti} \\
\hline
758&kostur&nkeo&Síðan lét Símon varðveita kost \textbf{kost} \\
\hline
759&kostur&nkeo&« Faðir minn mun sjá kost fyrir mér en annarra minna frænda ertu sá er eg vil helst mín ráð undir \textbf{kost} \\
\hline
760&kveld&nheþ&Á einu kveldi kom hann til þriggja búenda og ráku hann allir \textbf{kveldi} \\
\hline
761&kveld&nheþ&Að kveldi höfðu þeir brotið glugg á hauginn með atgangi prests en um morguninn var hann gróinn sem \textbf{kveldi} \\
\hline
762&kveld&nheo&Þeir komu síð um kveld að bæ þeim er Sokki svaf að um nætur og Sóti bróðir \textbf{kveld} \\
\hline
763&kveld&nheþ&Tíu vetrum síðar en Gestur fór úr Tungu bar það til tíðinda að allt sauðfé hvarf í brott það Þorbjörn bóndi átti í geymslu Gusts sauðamanns og leitaði þrjá daga í samt svo hann fann öngan sauð og kom svo heim að kveldi og sagðist mundu upp gefa leitina fjárins « því að eg hefi þessa daga leitað í allar áttir og þær leitir er mér þykir nokkur líkindi vera að fénaður megi verið \textbf{kveldi} \\
\hline
764&kveld&nheþ&Kom hann þar að kveldi dags og reið hann þegar í \textbf{kveldi} \\
\hline
765&land&nhfo&En um vorið eftir páska skipar Otkatla lönd sín og tók þá til þess fjárskiptis er verið hafði um sumarið og síðan fór hún af Helgastöðum með allt sitt og inn til Möðrufells til föður síns og er úr þessi \textbf{lönd} \\
\hline
766&land&nheo&Setti þá Magnús konungur sína menn til stjórnar um land allt þar í \textbf{land} \\
\hline
767&land&nheo&það land fór Jörundur eldi og lagði til \textbf{land} \\
\hline
768&land&nheþg&að Noregur mundi liggja laus fyrir ef nokkurir stórir höfðingjar vildu til sækja er engi var konungur yfir landinu og lendra manna forráð var þar yfir ríkinu en þeir lendir \textbf{landinu} \\
\hline
769&land&nheþ&Þeir lágu í meginhöfninni og var þar gert sæluhús mönnum þeim til ívistar er fara með landi fram og var borinn í húsið hálmur \textbf{landi} \\
\hline
770&land&nhen&land upp og allt lið hans en Magnús konungur sótti eftir þeim og rak \textbf{land} \\
\hline
771&land&nheog&herjaði þá víða um landið og fékk ófa mikið \textbf{landið} \\
\hline
772&land&nheo&Og nú kaupir hann land á Mel í Miðfirði og eflir þar bú á og gerist brátt rausnarmaður mikill í búinu og er svo sagt að eigi þótti um þetta minna vert en um farar \textbf{land} \\
\hline
773&land&nheþ&« Berið þau orð mín Ólafi konungi að eg vil gefa honum orlof til þess að fara heim til Skotlands með lið sitt og gjaldi hann aftur fé það allt er hann tók upp að röngu hér í \textbf{landi} \\
\hline
774&land&nhfo&Sagði hann þá fráfall Bárðar og það með að Bárður hafði gefið honum eftir sig lönd og lausa aura og kvonfang það er hann hafði áður \textbf{lönd} \\
\hline
775&land&nheo&Ráða þeir til ferðar er þeir voru búnir og óku upp á \textbf{land} \\
\hline
776&land&nheo&Síðan lagði Eiríkur jarl að skipi Vagns og var þar allhörð viðurtaka en að lyktum var hroðið skip þeirra en Vagn handtekinn og þeir þrír tigir og fluttir á land upp \textbf{land} \\
\hline
777&land&nheþ&Þormóður tekur öxi sína og hleypur á skipið og rær frá landi og stefnir til \textbf{landi} í Vík\\
\hline
778&land&nheo&Á Þingskálaþingi um haustið sótti Kolskeggur til lands að Móeiðarhvoli en Gunnar nefndi sér votta og bauð þeim undan Þríhyrningi lausafé eða land annað að löglegri \textbf{land} \\
\hline
779&land&nheo&að eg andist þá gerið mér kistu og látið mig fara fyrir borð og verður þetta annan veg en eg hugði að vera mundi ef eg skal eigi koma til \textbf{land} og nema þar land\\
\hline
780&land&nheo&Nú verður Ólafur konungur var þess að Knútur konungur var í svikum við hann og hyggur að sækja fund þenna með fjölmenni og segir Ólafur konungur nú mönnum sínum hverja frétt hann hafði af Knúti konungi « og vil eg eigi að vér bíðum hans hér og þykir mér hann nú brugðið hafa fyrri þessum sáttarfundi og munum vér nú freista að gera hér upprásir og herja á land Knúts konungs og gjalda honum svo svikin og ginning \textbf{land} \\
\hline
781&land&nhen&Og munum vér eigi það ófrelsi gera einum oss til handa heldur bæði oss og sonum vorum og allri ætt vorri þeirri er þetta land byggir og mun ánauð sú aldregi ganga eða hverfa af þessu \textbf{land} \\
\hline
782&land&nheþ&Komu þeir að sunnarlega síð um haustið og fengu sér eina róðraskútu og fóru norður með landi og ætluðu að fara á \textbf{landi} fund\\
\hline
783&land&nheo&er kista Kveld-Úlfs kom á \textbf{land} \\
\hline
784&land&nheþ&En er liðið kom saman þá nefndi konungur menn til að fara alla vega frá bænum að leita Hræreks á sæ og \textbf{landi} \\
\hline
785&land&nhen&« hvor okkar átt hefir land \textbf{land} \\
\hline
786&land&nheþ&Eyvindur sér hvar flokkur Þorvalds er í vörum fyrir og mælti við sína menn að þeir skyldu eigi að landi \textbf{landi} \\
\hline
787&land&nheo&Gísli hleypur nú á land og var það eyland er þá var hann á kominn og er þó skammt þaðan til \textbf{land} \\
\hline
788&land&nhfþ&Mun þá sá ráða löndum er sigurs verður \textbf{löndum} \\
\hline
789&land&nheþ&suma rak hann úr landi og tók upp fé fyrir öllum þeim er nokkuð mæltu móti honum en fór með her manns um landið en ekki með því fjölmenni er lög voru \textbf{landi} \\
\hline
790&land&nhfþ&lét þá búa skip sín og fór um vorið út til Túnsbergs og sat þar um vorið þá er þar var fjölmennast og þungi var fluttur til bæjar af öðrum \textbf{löndum} \\
\hline
791&land&nheo&Þeir Gísli róa norður fyrir land og stefna til \textbf{land} \\
\hline
792&land&nheo&En Ísólfur fór á Búland og átti land milli Kúðafljóts og \textbf{land} \\
\hline
793&land&nhfþ&Víkverjar höfðust og mjög í kaupferðum til Englands og Saxlands eða Flæmingjalands eða Danmerkur en sumir voru í víking og höfðu vetursetu á kristnum \textbf{löndum} \\
\hline
794&land&nheþ&Ólafur son Haralds konungs fór og með honum úr \textbf{landi} \\
\hline
795&land&nheþ&Ólafur konungur hafði þá verið konungur í Noregi fimmtán vetur með þeim vetri er þeir Sveinn jarl voru báðir í landi og þessum er nú um hríð hefir verið frá sagt og þá var liðið um jól fram er hann lét skip sín og gekk á land upp sem nú var \textbf{landi} \\
\hline
796&land&nheþ&Nú er svo að segja að eftir þessi miklu tíðindi er þar í landi höfðu gerst þá sýndist það öllum hinum vitrustum mönnum og hinum bestum vinum konungsins að taka annan mann til konungs og landstjórnar í stað þvílíks höfðingja sem þá var við \textbf{landi} \\
\hline
797&land&nheo&En ef þér viljið heldur þann kost að hafa land vort til yfirferðar þá skal það heimilt og viljið þér fara landveg í ríki yðart í \textbf{land} \\
\hline
798&land&nheþ&» Nú kasta þeir akkerum eigi allnær landi og brjóta upp vopn sín og búast þeir til bardaga ef þess þyrfti \textbf{landi} \\
\hline
799&land&nhfo&En er hann frá til öndvegissúlna sinna í Leiruvogi fyrir neðan heiði þá seldi hann lönd sín Úlfljóti lögmanni er þar kom út í \textbf{lönd} \\
\hline
800&land&nheo&Síðan fór hann austur í Hornafjörð og nam land austan frá Horni til Kvíár og bjó fyrst undir Skarðsbrekku í \textbf{land} \\
\hline
801&land&nheo&Eiríkur konungur lét leggja eld í Hallvarðskirkju og víða um býinn og mikið en Eiríkur konungur lagði í brott skipaliði sínu og máttu þeir hvergi á land koma fyrir norðan fjörðinn fyrir safnaði lendra manna en hvar sem þeir leituðu til landgöngu þá lágu eftir fimm eða sex eða \textbf{land} \\
\hline
802&land&nheþ&Hér er gott mannval saman komið þess er kostur er á landi \textbf{landi} \\
\hline
803&land&nheo&Og að sumri fór Hallfreður út til Íslands og kom skipi sínu í Leiruvog fyrir sunnan \textbf{land} \\
\hline
804&land&nheo&Hann fór norður yfir Jökulsá að Landbroti og nam land á milli Glóðafeykisár og Djúpár og bjó á \textbf{land} \\
\hline
805&land&nheog&Var þá talið silfrið og goldið fyrir landið hver peningur og var þá eftir í sjóðnum sex tigir \textbf{landið} \\
\hline
806&land&nhfo&Sendi hann suma brott í önnur lönd en suma hafði hann innan hirðar við sér og var það Eiríkur blóðöx er hugði að vera skyldi fyrir öllum sonum hans og hlýddu því allir sem hann vildi vera láta meðan hann \textbf{lönd} \\
\hline
807&land&nheo&þeir höfðu farið vestur um haf eftir Eysteini og fylgdu honum í land og héldu þegar norður til Þrándheims um gagndaga svo að hann skyldi hafa þriðjung Noregs við bræður \textbf{land} \\
\hline
808&land&nheþ&Finnst og varla á voru landi eða víðara sá maður er þokkasælli hafi verið af sínum vinum en þessi hinn blessaði biskup svo sem votta bréf Þóris erkibiskups eða Guttorms erkibiskups eða hins ágæta konungs Hákonar og margra annarra dýrðlegra manna í Noregi að þeir unnu honum sem bróður sínum og báðu hann fulltingis í bænum sem föður \textbf{landi} \\
\hline
809&land&nheo&beiddi Gísla ferðar suður um land til móts við Hálfdan og þar með alla aðra er hans flokk vildu \textbf{land} \\
\hline
810&land&nheþg&» Síðan kastaði hann sér til sunds og lagðist inn að \textbf{landinu} \\
\hline
811&leið&nveo&Förum vér þá norður til Þrándheims þannug sem landsmegin er mest fyrir oss og tökum lið það allt um leið er vér \textbf{leið} \\
\hline
812&leið&nveo&Leið svo veturinn framan til jóla að ekki bar fleira til \textbf{Leið} \\
\hline
813&leið&nven&Höskuldur Njálsson átti bú í Holti og móðir hans og reið hann jafnan til bús síns frá Bergþórshvoli og lá leið hans um garð á \textbf{leið} \\
\hline
814&leið&nveo&Og eftir það ráða þeir til ferðar og ríða alla hina sömu leið þar til að þeir koma í Jökulsdal í \textbf{leið} \\
\hline
815&leið&nveþ&En er á leið vorið tók mein Kolbeins að vaxa og lagðist hann í \textbf{leið} \\
\hline
816&leið&nveþ&En er á leið haustið hélt hann liði sínu til Noregs og dvaldist í Elfinni nokkura \textbf{leið} \\
\hline
817&leið&nveþ&Og er á leið kveldið og nóttina og morgna tekur þá vaknar Þorvarður og sér að Eysteinn var á fótum og \textbf{leið} \\
\hline
818&leið&nveþ&Og er þrælarnir sáu fall Hauks tóku þeir á rás og hljópu heim á leið og elti Arnkell þá allt um \textbf{leið} \\
\hline
819&leið&nveþ&Og er þeir voru á leið komnir með líkið fara þeir Þorgrímur við marga menn til \textbf{leið} við þá\\
\hline
820&leið&nveo&Þótti honum gesturinn vera beðinn skemmtanar en hann tók til og sagði sögu og hóf á þessa \textbf{leið} \\
\hline
821&leið&nveo&» En er á leið þá sendir Geitir orð þingmönnum sínum og fara þeir síðan úr Krossavík og stefndu leið til \textbf{leið} \\
\hline
822&leið&nveþ&Hann kom það sumar til Danmerkur í Hróiskeldu og fékk mikið af fénu þó að miklir spænir væru af telgdir og fóru sunnan um sumarið er á leið en leið hans var um \textbf{leið} \\
\hline
823&leið&nveþ&Um haustið er á leið sneri Grettir aftur hið syðra og léttir eigi fyrr en hann kom í Ljárskóga til Þorsteins Kuggasonar frænda síns og þar var vel við honum \textbf{leið} \\
\hline
824&leið&nveo&Fer hann nú leið sína þar til er leiti bar á milli \textbf{leið} \\
\hline
825&leið&nveþ&Hann svimur þá þangað á leið sem honum þótti skemmst til \textbf{leið} \\
\hline
826&leið&nvfn&Þeir riðu inn í Þorskafjörð og sátu í dæl einni inn frá Kinnastöðum þar sem leiðir skiljast til \textbf{leiðir} og inn með firði\\
\hline
827&leið&nveo&Þorgils svarar vel erindum ábóta og kvað rán skyldu rakna á hverja leið er hann vildi en dóm á máli okkru Halldórs kvaðst hann öngvum mundu í hendur fá nema sjálfum \textbf{leið} \\
\hline
828&leið&nveo&« Eg mun ríða þá leið er eg hefi sagt Guðmundi því að hann mun virða mér til hugleysis ef eg fer eigi \textbf{leið} \\
\hline
829&lið&nheo&« Ef eg skal til orustu fara þá vil eg konungi lið veita því að honum er liðs þörf \textbf{lið} \\
\hline
830&lið&nheo&Eigi hæfir að eg selji þig svo undir öxi því að þetta mætti okkur eigi endast þótt eg hefði hér til allra manna lið og \textbf{lið} \\
\hline
831&lið&nheo&Þeir bræður fjölmenna mjög til veislunnar og höfðu valið \textbf{lið} \\
\hline
832&lið&nheþ&Noregskonungur hafði leiðangur úti svo og það að hann fór með liði því til Danmerkur og þar var ófriður í ríki \textbf{liði} \\
\hline
833&lið&nheo&Fór þá Kálfur inn á Eggju en Finnur fór til konungs en Þorbergur og annað lið þeirra fór heim \textbf{lið} \\
\hline
834&lið&nhen&« Til mun eg leggja orð mín ef þér mætti lið að verða en annars staðar skaltu til leita um vistaferli þín en hjá \textbf{lið} \\
\hline
835&lið&nheo&Fór hann með það lið til fundar Ólafs \textbf{lið} \\
\hline
836&lið&nheo&En er þeir komu upp á holtið þá sáu þeir lið \textbf{lið} \\
\hline
837&lið&nheþ&Bjuggust þeir þá allir til bardaga og fylktu liði sínu á völlunum fyrir neðan lögréttu milli og \textbf{liði} \\
\hline
838&lið&nheþ&Þorgils hafði verið í Hafursfirði í liði Haralds konungs og stýrði þá skipi því er Þórólfur átti og hann hafði haft í \textbf{liði} \\
\hline
839&lið&nheþ&» Hann kvaðst hvergi fara mundu « og skal eg hér drepinn þér til svívirðingar og man eg það enn að faðir minn féll í liði föður þíns og Ingimundar og hefir það af þér hlotist og þínum \textbf{liði} \\
\hline
840&lið&nheo&En er hann snerist þaðan í brott þá setti hann þar eftir til landsgæslu Guttorm hertoga og lið mikið með \textbf{lið} \\
\hline
841&lið&nheo&Snorri kom sunnan í páskavikunni og drógu þeir Þorleifur þá lið saman af Rosmhvalanesi og um öll nes fyrir sunnan Borgarfjörð og höfðu þeir nær fjögur hundruð manna er þeir fóru utan á \textbf{lið} \\
\hline
842&lið&nheo&Þá sáu þeir skipalið Inga konungs og þóttust eigi lið við hafa og hljópu á skóg \textbf{lið} \\
\hline
843&lið&nheo&Karlsefni fór suður fyrir land og Snorri og Bjarni og annað lið \textbf{lið} \\
\hline
844&lið&nheo&Þá fór Sigurður konungur með miklu liði til kastala þess er jarl átti og flýði jarl undan því að hann hafði lítið \textbf{lið} \\
\hline
845&lið&nheþ&Síðan fóru þeir þaðan nær hálfu öðru hundraði manna og Kolbeinn Arnórsson var fyrir því liði og Hafur Brandsson fóstri \textbf{liði} \\
\hline
846&lið&nheþ&» Síðan keypti Þórólfur mjöl og malt og það annað er hann þurfti til framflutningar liði \textbf{liði} \\
\hline
847&lið&nheo&Markús og hans lið fóru upp á Markir og ætluðu þaðan til \textbf{lið} \\
\hline
848&lið&nheþ&Þá nátt er Ólafur konungur lá í safnaðinum og áður er frá sagt vakti hann löngum og bað til guðs fyrir sér og öðru liði sínu og sofnaði lítt og rann höfgi á hann í mót deginum en er hann vaknaði þá rann dagur \textbf{liði} \\
\hline
849&lið&nheo&Þorgils hafði mikið lið haft af Snæfellsnesi vestan með sér sem fyrr segir og veitti hann Þorvarði mikið lið með sjálfs síns framkvæmd og rösklegri \textbf{lið} \\
\hline
850&lið&nheo&Og er þeir koma til Limafjarðar og lágu að Hálsi þá átti Arinbjörn húsþing við lið sitt og sagði mönnum fyrirætlan \textbf{lið} \\
\hline
851&lið&nheþ&Lagði konungur liði sínu inn í Löginn og hélt upp í land til \textbf{liði} \\
\hline
852&lið&nheo&En jafnan er þeir skiptu liði sínu þá fylgdi konungi Norðmanna lið en Dagur fór þá í annan stað með sitt lið en Svíar í þriðja stað með sínu \textbf{lið} \\
\hline
853&lið&nhen&En fyrir þá sök að skipið var borðmikið svo sem borg væri en fjöldi manns á og valið hið besta \textbf{lið} \\
\hline
854&lið&nheþ&Þá sigldi Ólafur sínu liði vestan að eyjunum og lagði þar til hafnar því að Péttlandsfjörður var eigi \textbf{liði} \\
\hline
855&lið&nheþ&Þá hélt hann sunnan og lagði upp í Elfi í eystri kvísl og vann þar þrjú skip af liði þeirra Þóris hvínantorða og Ólafs sonar Haralds kesju systursonar \textbf{liði} \\
\hline
856&lið&nheo&Íslsag Kolbeinn ungi dró lið saman um Skagafjörð og öll héruð vestur þaðan til \textbf{lið} \\
\hline
857&lið&nheþ&Síðan ætluðu þeir til sjö hundruð manna að fara á njósn norður til Breiðu en fyrir því liði var höfðingi sonur \textbf{liði} \\
\hline
858&lið&nheo&Þá sneri mannfallinu brátt í lið \textbf{lið} \\
\hline
859&lið&nhen&Féll þar lið mart af Eyvindi en hann sjálfur hljóp fyrir borð og komst með sundi til lands og svo allt það lið er undan \textbf{lið} \\
\hline
860&lið&nheþ&Þegar er Hálfdan svarti spurði andlát hans þá byrjar hann ferð sína með miklu liði og fer norður til \textbf{liði} \\
\hline
861&lið&nheo&Og er menn voru drukknir um kveldið þá mælti Ingjaldur konungur til Fólkviðar og Hulviðar sona Svipdags að þeir skyldu vopnast og lið þeirra sem ætlað var um \textbf{lið} \\
\hline
862&lið&nheþg&Héldu þeir Þórólfur liðinu til \textbf{liðinu} \\
\hline
863&lið&nheþg&en Gissur reið frá liðinu með sjö tigu \textbf{liðinu} \\
\hline
864&lið&nheo&Sighvatur snýr þá til Flatatungu og var þar um nótt með lið \textbf{lið} \\
\hline
865&lið&nheþ&Hann gekk þar í milli því að Eyjólfur kallaði lög til þess að þeir hæðu þar féránsdóma í liði sínu sem þeim var óhætt framast að koma en þeir Önundur og Þorvarður hétu því að þegar skyldi bardagi \textbf{liði} \\
\hline
866&lið&nheo&Fær konungur bana og mart lið með \textbf{lið} \\
\hline
867&lið&nheo&En er á leið daginn þá fundu þeir konungur róðrarskip mörg í hverju eyjarsundi og hafði lið það ætlað til fundar við Þórólf því að njósnir hans höfðu verið allt suður í Naumudal og víða um \textbf{lið} \\
\hline
868&lið&nheo&en nú fengu þeir eina njósn af öllum skipum er sunnan fóru að Erlingur skakki hafði upp sett skip sín í Björgyn og mundu þeir hans eiga þangað að vitja og sögðu að hann hefði mikið \textbf{lið} \\
\hline
869&lið&nheþ&Þar fellur Arnór úr Hlíð af liði þeirra \textbf{liði} \\
\hline
870&lið&nheþ&Þeir skiptu þar liði sínu og fór Sigurður slembidjákn vestur um haf þegar um veturinn en Magnús fór til Upplanda og vænti sér þar mikils liðs sem hann \textbf{liði} \\
\hline
871&lið&nhen&Þaðan fór hann til Vatnsfjarðar og svo til Steingrímsfjarðar til Jóns Brandssonar og var mart lið í för við honum og var mælt að sendir mundu menn fyrir að segja að þeir kæmu eigi \textbf{lið} \\
\hline
872&lið&nhen&Var lið þeirra allt drepið utan þeir buðu formanni \textbf{lið} grið\\
\hline
873&lið&nheþ&Síðan safnaði hann sér liði og fékk nær tólf hundruð \textbf{liði} \\
\hline
874&lið&nheþ&En er hann fréttir tilkomu konungs lét hann búa fagra veislu og til þeirrar veislu býður hann konungi með öllu liði sínu og það þá konungur og gekk á land með lið sitt en jarl leiddi konung heim með allri hirð sinni til hallar með \textbf{liði} handa hljóðfærum\\
\hline
875&lið&nheþg&Þegar um morguninn lét Erlingur blása öllu liðinu út á Eyrar til \textbf{liðinu} \\
\hline
876&lið&nheo&Kom þar til hans Erlingur Skjálgsson með lið \textbf{lið} \\
\hline
877&lið&nheþ&Féll þar mart af keisarans liði en þeir fengu ekki unnið að \textbf{liði} \\
\hline
878&lið&nheo&En er hann kom suður um Kjöl reið hann frá liðinu með hundrað manna suður til Keldna og bað Hálfdan veita sér lið með allan sinn \textbf{lið} \\
\hline
879&lið&nheþg&En koma mun eg til þings en liðinu heit eg \textbf{liðinu} \\
\hline
880&lið&nheþ&Þá sendi Ólafur konungur um haustið Hrana fóstra sinn til Englands að eflast þar að liði og sendu Aðalráðssynir hann með jartegnum til vina sinna og frænda en Ólafur konungur fékk honum lausafé mikið að spenja lið undir \textbf{liði} \\
\hline
881&lið&nheo&Þegar jafnskjótt býst jarl til ferðar og með honum Ástríður konungsdóttir og höfðu nær hundraði manna og valið lið bæði af hirðinni og af ríkum bóndasonum og vandaðan sem mest allan \textbf{lið} \\
\hline
882&maður&nkfn&Ragnhildur sendi menn austur á Jaðar til Erlings föður síns og bað hann senda sér \textbf{menn} \\
\hline
883&maður&nkeng&Sá kveðst engi tíðindi kunna að segja önnur en hann kveðst sjá mann ríða handan um vaðla og þar til er hross Þorleiks voru « og sté maðurinn af baki og höndlaði \textbf{maðurinn} \\
\hline
884&maður&nken&Sátu þeir þar lengi einir saman svo að engi maður kom til \textbf{maður} \\
\hline
885&maður&nkfo&Hann var kominn af risakyni í föðurætt sína og er það vænna fólk og stærra en aðrir menn en móðir hans var komin af tröllaættum og brá því Dumbi í hvorutveggju ætt sína því hann var bæði sterkur og vænn og góður viðskiptis og kunni því að eiga allt sambland við mennska \textbf{menn} \\
\hline
886&maður&nkfþ&Annan dag eftir safnar Þorkell mönnum og reið í Gautavík með þrjá tigu \textbf{mönnum} \\
\hline
887&maður&nken&Maður heitir Gjafvaldur og er mér sagt að nú sé hirðmaður \textbf{Maður} \\
\hline
888&maður&nkfo&Það er sumra manna sögn að það væri gert með ráði Snorra goða og hafi hann svo fyrir sagt að hann skyldi vita ef hann mætti leynast inn í skálann og leita þaðan til áverka við menn og bað hann ganga ofan skarð það er upp er frá Leikskálum og ganga þá ofan er máleldar væru gervir því að hann sagði það mjög far veðranna að vindar lögðust af hafi um kveldum og hélt þá reykinum upp í skarðið og bað hann þess bíða um ofangönguna er skarðið fyllti af \textbf{menn} \\
\hline
889&maður&nkfn&« Standi menn upp og vopnist og verjið góðan dreng því að hér er mart röskra drengja og manna fyrir og látið illa verða för Skeggja \textbf{menn} \\
\hline
890&maður&nken&lendur maður \textbf{maður} \\
\hline
891&maður&nken&Sturla svarar fá en mælti þetta svo að nokkurir menn heyrðu að þótt hann hefði vitað þetta fyrir að svo hefði orðið sem nú var að heldur vildi hann þenna kjósa en standa yfir drápi Þorgils frænda síns og vita það víst að hann þætti aldrei slíkur maður sem \textbf{maður} \\
\hline
892&maður&nkfn&Síðan riðu menn brott af þeim fundi og fóru málin of sumarið til \textbf{menn} \\
\hline
893&maður&nkfn&Þá fóru þeir úr eyjunni þrír tigir manna og fóru allir hinir röskvari menn nema Pétur \textbf{menn} \\
\hline
894&maður&nkfn&Þorkell sat á miðjum palli og menn hans alla vega út í frá \textbf{menn} \\
\hline
895&maður&nken&Verður sú lítil virðing sem snemma leggst á ef maður lætur síðan með ósóma og hefur eigi traust til að reka þess réttar síðan og eru slíkt mikil undur um þá menn sem hraustir hafa \textbf{maður} \\
\hline
896&maður&nkfo&Sendi Sturla þá menn til \textbf{menn} að leita um sættir\\
\hline
897&maður&nkfþ&Lík Brands var fært til Staðar og þar jarðað fyrir sunnan kirkju við sönghúsið fyrir stúkudyrum og var hann mjög harmdauði sínum \textbf{mönnum} \\
\hline
898&maður&nkfn&Nú koma menn til jólaveislu til þeirra \textbf{menn} \\
\hline
899&maður&nkeo&Látið þér skipa skútu með góðum drengjum og fá til að stýra mig eða annan lendan mann að fara á fund Haralds konungsfrænda þíns og bjóða honum sættir eftir því sem réttlátir menn gera í milli \textbf{mann} \\
\hline
900&maður&nkfo&Þá sendi Kirjalax konungur menn til \textbf{menn} \\
\hline
901&maður&nken&Sagði Þórirs maður ei þar jafn vel skemmt hafa verið « því engi er skemmtunarmaður betri en Halli er þar var um \textbf{maður} \\
\hline
902&maður&nkfn&Nutu menn lítt tals \textbf{menn} \\
\hline
903&maður&nkfo&En Þorsteinn sá það og lýstur á móti hest Þórðar heldur meira högg og rann nú hesturinn Þórðar og æptu menn þá með \textbf{menn} \\
\hline
904&maður&nkeo&Skildu þeir með hinni mestu vináttu og þótti öllum mikils háttar hversu jarl gerði við þenna mann umfram alla \textbf{mann} \\
\hline
905&maður&nken&Þá gekk Bárður til drottningar og sagði henni að þar var maður sá er skömm færði að þeim og aldregi drakk svo að eigi segði hann sig \textbf{maður} \\
\hline
906&maður&nken&En sá maður er hlutinn skyldi upp taka þá tók hann upp annan og hélt milli fingra sér og brá upp hendinni og \textbf{maður} \\
\hline
907&maður&nkfn&Kolbeins menn tóku þá drjúgum menn af Þórði er hestana \textbf{menn} \\
\hline
908&maður&nkfo&Nú líða nokkur misseri frá því og eitthvert sinn reið Ásbjörn til þings með menn \textbf{menn} \\
\hline
909&maður&nkfog&Nú sáu þeir mennina og áttu skammt til \textbf{mennina} \\
\hline
910&maður&nkfn&Allir menn hljópu úr sundinu fyrir Högna en þó var lagið af húsunum til \textbf{menn} og höggið\\
\hline
911&maður&nkfþ&En eg veit að við erum báðir illmenni því að þú mundir ekki hér kominn frá öðrum mönnum nema þú værir nokkurs \textbf{mönnum} útlagi\\
\hline
912&maður&nkfn&Orms menn sóttu \textbf{menn} \\
\hline
913&maður&nkeo&« Eru þetta þín ráð frændi að drepa saklausan mann og ganga á \textbf{mann} \\
\hline
914&maður&nkfþ&og svo hirðsveitinni og handgengnum mönnum \textbf{mönnum} \\
\hline
915&maður&nkfn&Gengu menn þá vopnlausir til dóma þar er þá skyldi mál fram \textbf{menn} \\
\hline
916&maður&nkeþ&Eftir það kvaddi jarl þings og sagði þar fyrir hverju áfelli hann var orðinn « vil eg þeim manni gifta Guðrúnu systur mína er Surti verður að bana því að eg veit að sá einn mun til þess ráðast að mér mun engi ósæmd í því \textbf{manni} \\
\hline
917&maður&nkfþ&Hann var því vanur að fara norður um héruð á vorið og hitta þingmenn sína og ræða um héraðsstjórn og skipa málum með \textbf{mönnum} \\
\hline
918&maður&nkeo&Þar fann hann mann er nefndist Halli og kvaðst fara skyldu ofan til \textbf{mann} á Völlu\\
\hline
919&maður&nkeþ&Hann var til þess tekinn að honum var verra til hjóna en öðrum mönnum og galt nær öngum manni \textbf{manni} \\
\hline
920&maður&nkfn&Allir \textbf{menn} menn\\
\hline
921&maður&nkfn&Þá gengu þeir Sigurður og hans menn til skútu nokkurrar og skipuðust menn við árar og reru út á voginn undir \textbf{menn} \\
\hline
922&maður&nkfn&» Í því bili hljópu þangað menn jarls fjórir og sögðu þeim ill \textbf{menn} \\
\hline
923&maður&nkfn&Og þá komu njósnarmenn þeirra Steinólfs og segja að eigi mundu færri menn ríða inn fyrir Króksfjarðarmúla en fimm \textbf{menn} \\
\hline
924&maður&nkeo&Og þessu næst heyra þau út mannamál og er Eyjólfur kominn við fimmtánda mann og hefir komið áður heim til húsa og sjá þeir döggslóðina hvar þau höfðu farið og var þá sem vísað væri til \textbf{mann} \\
\hline
925&maður&nkeo&Hinn þriðja dag páska reið Kjartan heiman við annan \textbf{mann} \\
\hline
926&maður&nkfþ&» Þeir Barði höfðu skipað til mönnum áður að tveir skyldu annast einn hvern \textbf{mönnum} \\
\hline
927&maður&nkfn&Var Atli þá skírður með heimamenn sína og margir aðrir menn er þeir komu til því að heilags anda miskunn nálgaðist af orðum \textbf{menn} \\
\hline
928&maður&nkfo&lét menn sína ríða hið efra dag og nótt svo sem lið konunga fór hið \textbf{menn} \\
\hline
929&maður&nkfn&En menn Þórðar vildu það eigi og létu hann eigi ná að ríða en er Þórður kom eftir bað hann biskup ríða hvert er hann vildi og reið eftir sem ákafast þar til er hann kom í geilar hjá Skálaholti og bað hann menn þá stíga af baki og búast til \textbf{menn} \\
\hline
930&maður&nken&Var hann hinn vinsælsti maður og þótti vænn til \textbf{maður} \\
\hline
931&maður&nkeng&Og gerir hann svo og segir konungi að maðurinn er fram kominn mjög og vill nú iðrast mjög « að hann hefir í móti yðrum vilja gert og vill nú leggja allt á yðvart vald herra og ger nú svo vel að þú fyrirlít eigi manninn svo mjög og miskunna \textbf{maðurinn} \\
\hline
932&maður&nken&Maður þeirra einn skeindist er hann skyldi á bak \textbf{Maður} \\
\hline
933&maður&nkfþ&Jafnskylt var öllum mönnum í lögum þeirra að færa dauða menn til graftrar sem nú ef þeir eru \textbf{mönnum} \\
\hline
934&maður&nkfþ&Þykir mönnum þar til gleði gott að drekka mörgum \textbf{mönnum} \\
\hline
935&maður&nkeo&ber það þó upp fyrir engan mann því að hann vildi eigi að Eiður son hans og bræður Þórðar hefðu nokkurn grun af hans ráðagerðum fyrr en fram \textbf{mann} \\
\hline
936&maður&nken&« Hver er sá hinn tígulegi maður er þar ríður á þeim hvíta hesti fyrir liði \textbf{maður} \\
\hline
937&maður&nkfþ&hvort þér viljið heldur að eg ráðist í mót berserkjunum með þér eða viltu að eg sjái til yðvarrar sameignar af hólinum og kunni eg frá að segja öðrum \textbf{mönnum} \\
\hline
938&maður&nken&Og þótti oss þó Haraldur konungur Gormsson vera minni fyrir sér en Uppsalakonungar því að Styrbjörn frændi vor kúgaði hann og gerðist Haraldur hans maður en Eiríkur hinn sigursæli faðir minn steig þó yfir höfuð Styrbirni þá er þeir reyndu sín á \textbf{maður} \\
\hline
939&maður&nkeo&Hann sýnir búum sár Höskulds og nefnir votta að benjum og nefnir mann til hvers sárs nema \textbf{mann} \\
\hline
940&maður&nken&En er hann hafði fram borið með mikilli snilld mörg og sönn stórmerki almáttigs guðs þá svaraði fyrstur kynstór maður og göfugur þó að heiðinn væri og \textbf{maður} \\
\hline
941&maður&nkfn&» En er bónda þótti ósýnt til bóta en menn ósvífir þá réð hann það af að hann gifti Helgu dóttur sína Sleitu-Helga og er gert brúðhlaup þeirra snemma vetrar og þaðan af voru þeir eigi stórilla við menn ef ekki var til gert við \textbf{menn} \\
\hline
942&maður&nkfn&Og er málin komu í dóm gengu menn að og voru málin í gerð lagin með umgangi og sættarboðum góðgjarnra manna og kom svo að Snorri goði gekk til handlaga fyrir víg Vigfúss og voru þá gervar miklar \textbf{menn} \\
\hline
943&maður&nkfn&Þá komu þar Kolbeins menn þeir er seint höfðu orðið og flettu þá alla en særðu einn \textbf{menn} \\
\hline
944&maður&nken&maður af skipi hans sá er hét Kolbeinn Þorljótsson úr \textbf{maður} \\
\hline
945&maður&nkfþ&» Nú þó að þeir þyldu mörg vandræði af vondum mönnum þá léttu þeir eigi því heldur af að fara um sveitir og boða \textbf{mönnum} erindi\\
\hline
946&maður&nkfn&Hann tekur það ráð að hann sagði Þórdísi toddu um kvöldið það hið sama og hann skyldi heiman ríða um morguninn eftir að þar væri Gunnar Þiðrandabani í hans varðveislu « vil eg nú að þú takir við valdi hans meðan eg er í burtu og gætir að eigi verði menn varir við að hann sé \textbf{menn} \\
\hline
947&maður&nkfn&Þeir koma á Flugumýri og var Kolbeinn í hvílu kominn og menn \textbf{menn} \\
\hline
948&maður&nkeþ&« Það vildi eg Björn að þú létir af komum til Þórdísar og er þér fremd engi að skaprauna gömlum manni og lát að orðum mínum og mun eg veita þér annan tíma slíkt \textbf{manni} \\
\hline
949&maður&nkfn&Þeir fóru ofan til Brunnár og er þeir komu mjög að garði þá fóru í móti þeim menn \textbf{menn} \\
\hline
950&maður&nken&Maður lést af Lofti snemma \textbf{Maður} \\
\hline
951&maður&nken&Síðan fór hann til útbús þess og tók þar við og var hann hinn nýtasti maður og hafði auð \textbf{maður} \\
\hline
952&maður&nken&« Mikill maður ertu fyrir þér Búi og mun nú skilja með okkur og far nú til áttjarða þinna í friði fyrir \textbf{maður} \\
\hline
953&maður&nkfn&En allir Sveins menn reru þá \textbf{menn} \\
\hline
954&maður&nkfo&Þá er þeir biskup fóru norðan um Eyjafjörð hljópu nokkverjir óspektarmenn úr flokki biskups til Gása og rændu útlenda menn þá er biskup kallaði í banni að samneyti við Kolbein og \textbf{menn} \\
\hline
955&maður&nkeo&Nú ríður hann heiman við tvítjánda mann og norður til Reykjadals og riðu snemma aftans \textbf{mann} \\
\hline
956&maður&nkfo&Sighvatur sendi þannig menn og bauð biskupi til sín og vildi að hann væri með honum í þessum \textbf{menn} \\
\hline
957&maður&nkfn&Í þenna tíma voru margir menn og göfgir á Íslandi þeir er í frændsemistölu voru við Ólaf konung \textbf{menn} \\
\hline
958&maður&nkfo&Þessa menn nefnum vér til \textbf{menn} \\
\hline
959&maður&nkfn&» Nú heyra þeir gný mikinn að menn ríða margir til \textbf{menn} \\
\hline
960&maður&nkfo&Fátt var manna heima því að Halldór hafði sent menn norður í \textbf{menn} \\
\hline
961&maður&nkfn&En er þau komu í Reykjaholt voru menn settir til þess að telja um fyrir þeim að þau skyldu handsala Gissuri arf \textbf{menn} \\
\hline
962&maður&nken&Hann kvaðst mikla þökk kunna hans þarkomu « skal eg taka saman fé þitt því að þú ert frægur maður og muntu mér að liði verða því að eg er í nauðum \textbf{maður} \\
\hline
963&maður&nkfn&Tók þá að flýja meginlið Birkibeina en fjöldi féll því að Magnúss konungs menn drápu allt það er þeir máttu og voru engum manni grið \textbf{menn} \\
\hline
964&maður&nken&En Knútur konungur segir svo að Ólafur konungur væri svo vitur maður að hann hefði eigi farið einskipa í gegnum her Knúts konungs og lést líklegra þykja að þar mundi verið hafa Hárekur úr Þjóttu eða \textbf{maður} maki\\
\hline
965&maður&nkeo&Eftir víg þeirra Kálfs og Guttorms feðga riðu þeir Kolbeinn og Órækja suður í Reykjaholt með átjánda mann að sækja ráð að Snorra og leituðu eftir hvert liðsinni þeir skyldu þar \textbf{mann} \\
\hline
966&maður&nken&Þetta sumar reið Þórður til þings og mælti þá engi maður mót honum á \textbf{maður} \\
\hline
967&maður&nkfo&Njósnarmenn hljópu upp og réðu að þeim en þeir í mót og varð sá fundur þeirra að Egill felldi tvo menn en hinir er eftir voru hljópu þá í \textbf{menn} \\
\hline
968&maður&nken&Og er þeir komu nærri bænum kom maður í för \textbf{maður} \\
\hline
969&maður&nkfþ&að sú sögn væri þar höfð af fróðum mönnum að Zóe drottning vildi sjálf hafa Harald sér til manns og sú sök væri reyndar mest við Harald er hann vildi brott fara úr Miklagarði þó að annað væri upp borið fyrir \textbf{mönnum} \\
\hline
970&maður&nkfn&Það segja menn að Einar Sigmundarson hafi kallað á Bárð til \textbf{menn} sér\\
\hline
971&maður&nkfn&hljóp á bak og reið leið sína og allir hans menn en konungur reið aftur til \textbf{menn} \\
\hline
972&maður&nkfn&Og er þeir höfðu skamma stund barist kom Knútur við fimmtánda mann og veitti Steinólfi og sneri þá skjótt mannfallinu á hendur þeim Þórarni og féll hann þar og níu menn með honum en \textbf{menn} af Steinólfi\\
\hline
973&maður&nkfn&En þau voru orð á að þeir mundu af þinginu fara að biskupi með öllum flokkum þessum og taka menn þessa er þeir höfðu \textbf{menn} \\
\hline
974&maður&nken&« Svo þykir mér sem Hrolleifur láti eigi af sínum ferðum og þætti mér til þín koma Oddur frændi því að þú ert nú maður ungur og til alls vel fær en eg er örvasi fyrir \textbf{maður} sakir\\
\hline
975&maður&nken&Oddur biður hann fara með sína eign sem hann yrði mestur maður af og vinsælastur og kveðst reynt hafa að hann vildi og kynni best hans fé að \textbf{maður} \\
\hline
976&maður&nken&Úlfur var maður svo mikill og sterkur að eigi voru \textbf{maður} jafningjar\\
\hline
977&maður&nkfn&en menn segja að hjartað væri harðla lítið og höfðu sumir menn það fyrir satt að minni séu hugprúðra manna hjörtu en huglausra því að menn kalla minna blóð í litlu hjarta en miklu en kalla hjartablóði hræðslu fylgja og segja menn því detta hjarta manna í brjóstinu að þá hræðist hjartablóðið og hjartað í \textbf{menn} \\
\hline
978&maður&nkfn&Og er menn koma til þings þá vill Þórður halda fram vígsök en Björn fékk vörn í málinu og bar þá vörn fram að svo hefði mælt verið að sá skyldi óheilagur falla er vísuna kvæði svo að hann \textbf{menn} \\
\hline
979&maður&nkfn&Allir menn áttu þá hlut í að eigi skyldi í greinir fara með þeim og sættust með því að Kolbeinn gerði átján hundrað þriggja alna aura fyrir víg \textbf{menn} \\
\hline
980&maður&nkfn&Hrærekur var fámálugur og svaraði stirt og stutt þá er menn ortu \textbf{menn} á hann\\
\hline
981&maður&nkfþ&Vil eg nú gefa leyfi öllum mönnum að fara til Noregs þeim er það vilja heldur en fylgja \textbf{mönnum} \\
\hline
982&maður&nkfo&Þeir komu í Eiríksfjörð og sóttu menn til fundar við þá og slógu \textbf{menn} \\
\hline
983&maður&nkfn&Og höfðu menn það fyrir satt að það mundi mjög vera fyrir sakir mála Snorra Sturlusonar er lát hans hafði nakkvað af konunginum \textbf{menn} \\
\hline
984&maður&nkfo&Eru þá stefnur að áttar ef það mætti af ráðast og þar kom að fé var lagt til höfuðs dýrinu og gerðu menn úr hvorritveggju \textbf{menn} \\
\hline
985&maður&nken&Atli hét maður er bjó í Otradal og átti systur Steinþórs á Eyri er Þórdís \textbf{maður} \\
\hline
986&maður&nkfþ&En í þessi hríð hafði orðið mikið mannfall og þó meira af \textbf{mönnum} mönnum\\
\hline
987&maður&nkfn&« sem fáir menn mæla til vinar síns að eg vildi að þú þyrftir manna við og vissir þú hvort eg gengi þér fyrir nokkurn mann eða \textbf{menn} \\
\hline
988&maður&nkfo&gekk á land og bað menn sína bíða á skipi til annars dags í þær mundir en halda á burt ef hann kæmi þá \textbf{menn} \\
\hline
989&maður&nkfn&Eigum vér sigurs von af skjótum atburðum en hitt mun oss þungt falla ef vér berjumst til mæði svo að menn verði fyrir því \textbf{menn} \\
\hline
990&maður&nkfn&Menn fara eftir honum og vildu sjá ferð hans og svo ef honum mætti nokkuð við \textbf{Menn} \\
\hline
991&maður&nkfn&lagst síðan til Vindasnekkjunnar og hefðu menn Ástríðar flutt hann til \textbf{menn} \\
\hline
992&maður&nken&En Þórður segir að Björn væri hinn röskvasti maður « og mér að góðu kunnur og því sendi Skúli yður þenna mann að hann átti eigi annan frænda sæmilegra \textbf{maður} \\
\hline
993&maður&nken&« Miklu ertu Þorsteinn óvitrari maður en eg hugði ef þú vilt eiga náttból undir exi minni og hætta til þess virðingu \textbf{maður} \\
\hline
994&maður&nkfn&Þá var það ráð gert að Finnbjörn og Ögmundur skyldu ríða upp til Hörgárdalsheiðar og gæta að að hvorki færu menn norður né \textbf{menn} \\
\hline
995&maður&nkfþ&Fýsti eg þig mjög hingaðfararinnar en nú vil eg hins biðja að þú farir heim sem skyndilegast og þess með að þú komir eigi á fund Haralds konungs nema betri verði sætt ykkur en mér þykir nú á horfast og gæt þín vel fyrir konungi og \textbf{mönnum} mönnum\\
\hline
996&maður&nkfn&« Hví spyrja þeir hinir ungu menn eigi að forlögum sínum því að mér þykir þeir merkilegastir menn af þeim sem hér eru saman \textbf{menn} \\
\hline
997&maður&nkeo&Hann ríður við sétta mann í Skál til Helga \textbf{mann} \\
\hline
998&maður&nkfo&Ólafur konungur hugsaði fyrir sér um þetta boð en er hann bar það fyrir menn sína þá löttu allir að staðfestast þar og eggjuðu konung að ráða norður til Noregs til ríkis \textbf{menn} \\
\hline
999&maður&nkfn&« Veg þú aldrei meir í hinn sama knérunn en um sinn og rjúf aldrei sætt þá er góðir menn gera meðal þín og annarra og þó síst á því \textbf{menn} \\
\hline
1000&maður&nkfog&Og er þeir koma mjög svo að upp sér hann mennina og sprettur upp og hleypur og ætlar að gera vart við komu \textbf{mennina} \\
\hline
1001&maður&nkeo&kvað mann ekki mikils háttar « en með því að vér höfum riðið heiman allt hingað til þín þessa erindis en maður mjög kominn á vort vald er það vænst Finnbogi er þú leggur mikið kapp á við þenna mann er þér dragi þetta til \textbf{mann} \\
\hline
1002&maður&nken&Sighvatur kom á fund Hákonar konungs og gerðist \textbf{maður} maður\\
\hline
1003&maður&nkfn&En hér virða menn meir afla \textbf{menn} en sannsýni\\
\hline
1004&maður&nkeo&Hann sendi mann til \textbf{mann} að segja Þórði\\
\hline
1005&maður&nkfþ&Grettir fór úr Tungu upp til Haukadals og þaðan norður á Kjöl og hafðist þar við um sumarið lengi og var nú eigi traust að hann tæki eigi af mönnum plögg \textbf{mönnum} \\
\hline
1006&maður&nken&Hafði hann áður verið hleypipiltur þeirra bræðra en var nú orðinn gildur maður og lagamaður \textbf{maður} \\
\hline
1007&maður&nkfn&Tóku þá menn hans og drógu hann öfgan milli skipanna til sín og í því fékk hann fjögur \textbf{menn} \\
\hline
1008&maður&nkfn&Og of sumarið er menn komu til alþingis þá var að sóttur Jón Loftsson þessum \textbf{menn} \\
\hline
1009&maður&nkfn&« Ef vorir konungar hafa fleiri menn látið þá munu sýslumenn Svíakonungs jafna það með tólf manna fjörvi þá er þeir koma sunnan eftir jólin og vitið þér ógerla veslir menn til hvers þér eruð \textbf{menn} \\
\hline
1010&maður&nkfþ&Fékk Skúli frændi hans og faðir hans honum góðan farareyri svo að hann var vel sæmdur af að fara með góðum \textbf{mönnum} \\
\hline
1011&maður&nkfo&» En Finnbogi kvaðst lítið gerast um hlaupandi menn og þóttist illa á brenndur lygðum þeirra « en þó heyrði eg getið þessa á sumri að maðurinn hafði sekur \textbf{menn} \\
\hline
1012&maður&nkfn&En þó lét Þorgils leiðast eftir bæn Brodda að hann reið upp til Hóla og menn \textbf{menn} \\
\hline
1013&maður&nkfþ&tekið við þeim mönnum er rekið hafa harma \textbf{mönnum} \\
\hline
1014&maður&nken&« Ákafur maður ertu og eigi mjög stilltur og muntu ná goðorði þínu þó að þú heitist eigi til og eigi gerði faðir þinn svo þá er hann missti Ljóts sonar \textbf{maður} á alþingi\\
\hline
1015&maður&nkfn&Hann var svo sigursæll að í hverri orustu fékk hann gagn og svo kom að hans menn trúðu því að hann ætti heimilan sigur í hverri \textbf{menn} \\
\hline
1016&maður&nkfn&« Þar mun liggja meinvættur nokkur er menn eru tregari til að geyma síður þíns fjár en annarra \textbf{menn} \\
\hline
1017&maður&nkeþ&Eyjólfur hljóp upp og hans menn og hélt þar maður á \textbf{manni} \\
\hline
1018&maður&nkfþ&En er Þórður ætlaði aftur á sitt skip þá sá hann að skipið var autt af hans mönnum en hann hafði þá eigi liðskost til að sækja \textbf{mönnum} \\
\hline
1019&maður&nkfþ&En þó gerist nú það miklu meira vandmæli en fyrr hefir verið því að vér höfum hér til náð í friði að sitja af útlendum höfðingjum en nú spyrjum vér hitt að Noregskonungur ætli að herja á hendur oss og er mönnum þó grunur á að Svíakonungur muni og til þeirrar ferðar \textbf{mönnum} \\
\hline
1020&maður&nken&Hafði og engi maður verið einfaldari í öllum málaferlum við \textbf{maður} en hann\\
\hline
1021&maður&nken&Þar féllu og báðir húskarlar Þóris og hinn þriðji \textbf{maður} \\
\hline
1022&maður&nkfþ&Hurfu mönnum gripir margir úr hirslum og var svo mikill gangur að því að nálega úr hvers manns hirslum hvarf nokkuð hversu rammlegur lás sem fyrir var en þó var engi lásinn \textbf{mönnum} \\
\hline
1023&maður&nken&En Gísl rak það til þessa snarræðis að hann átti að hefna föður síns en þessi maður hét Gjafvaldur er Gísl \textbf{maður} \\
\hline
1024&maður&nkeo&Þá skyldi og reiða hundrað fyrir hvern mann er til brennu hafði verið og voru það tíu tigir hundraða og skyldi það Kolbeinn greiða að helmingi og bæta Önund að \textbf{mann} \\
\hline
1025&maður&nkeo&Hann fór nú við fjórða mann á Jarlsstaði til Örnólfs og kveðst vilja kaupa yxnin til \textbf{mann} að gefa Áskatli\\
\hline
1026&maður&nkfn&En um kvöldið er menn brutu upp vistir sínar sat Sturla kyrr og bauð engi maður honum til \textbf{menn} \\
\hline
1027&maður&nkfn&Og er þeir höfðu þar litla stund við land verið þá koma menn til \textbf{menn} við þá\\
\hline
1028&maður&nkfn&Tjáðu menn þá fyrir jarli hver ófæra honum var í að gera svo mikið hervirki á konungs þegnum og í \textbf{menn} landi\\
\hline
1029&maður&nken&» Nú gaf Magnús konungur honum silfur fyrir hólminn og vill nú eigi þar hætta honum og fór Hreiðar út til Íslands og bjó norður í Svarfaðardal þar sem síðan heitir á Hreiðarsstöðum og gerist mikill maður fyrir \textbf{maður} \\
\hline
1030&maður&nkfn&En er menn Ólafs konungs komu til þeirra þá höfðu þeir fjölmennt fyrir framan tjöldin öll og náðu þeir ekki inn að \textbf{menn} \\
\hline
1031&maður&nkeo&Helgi selseista hjó í fyrstu mann þeirra Þórarins banahögg og eftir það hljóp hann á \textbf{mann} \\
\hline
1032&maður&nkfn&Og er menn koma til þingsins hefir Bolli fram sakir á hendur \textbf{menn} \\
\hline
1033&maður&nkfo&Hann sendi menn til Haralds konungs og bauð honum til \textbf{menn} \\
\hline
1034&maður&nken&Sá maður bjó í Haga er Þorvaldur hét og var kallaður menni og átti dóttur Þórðar Hrafnssonar frá Stokkahlöðu er Helga \textbf{maður} \\
\hline
1035&maður&nkfn&Látið þér skipa skútu með góðum drengjum og fá til að stýra mig eða annan lendan mann að fara á fund Haralds konungsfrænda þíns og bjóða honum sættir eftir því sem réttlátir menn gera í milli \textbf{menn} \\
\hline
1036&maður&nkfn&Hann var alroskinn og vel að sér ger og það segja menn að fyrir hann var \textbf{menn} örvænt\\
\hline
1037&maður&nken&» Síðan lét þessi maður búa skipið með þeim og var þar við til þess er byr kom sá er þeim var hagstæður út að \textbf{maður} \\
\hline
1038&maður&nkfn&Riðu þeir Kolbeins menn þá ofan eftir Langey að \textbf{menn} \\
\hline
1039&maður&nken&Á þingi voru knjáð mál þeirra og voru í fyrstu öll vitni borin í hag Hrafni en Þorvalds menn báru með honum allir nema einn \textbf{maður} \\
\hline
1040&maður&nkeo&Síðan kallaði konungur til sín þann mann er Arnaldur \textbf{mann} \\
\hline
1041&maður&nkfn&Karl gengur þegar í lið með Skíða og hans menn og berjast þann dag allan og gekk Karl jafnan í gegnum lið \textbf{menn} \\
\hline
1042&maður&nkfn&En það var eigi af því undarlegt að margir menn komu til konungs úr héruðum en því þótti það nýnæmi að þessi var maður svo hár að engi annarra tók betur en í öxl \textbf{menn} \\
\hline
1043&maður&nkfn&Það er nú næst sagt að Gissur hvíti og Hjalti mágur hans komu út með kristniboð og allir menn voru skírðir á Íslandi og kristni var í lög tekin á alþingi og flutti Snorri goði mest við Vestfirðinga að við kristni væri \textbf{menn} \\
\hline
1044&maður&nkfo&Hann hafði lagið í lægið Þórðar og ræddu menn um við Þórð að honum mundi eigi hlýða er hann hafði skipinu Ingimars lagt úr læginu en Þórður gáði ekki að því hvað um var \textbf{menn} \\
\hline
1045&maður&nken&Hann var mikill maður og öflugur og vel stilltur og vann svo fyrir búi föður síns að eigi mundi þriggja verk manna annarra \textbf{maður} \\
\hline
1046&maður&nkfo&En sent hefi eg fyrrum menn til Þorleifs og hafa þeir engu á leið komið við hann því er eg \textbf{menn} \\
\hline
1047&maður&nkfn&Hann eggjaði þá sína menn til \textbf{menn} \\
\hline
1048&maður&nkfn&Snorri goði sótti þessa veislu með Þorkatli og höfðu þeir nær sex tigu manna og var það lið mjög valið því að flestir allir menn voru í \textbf{menn} \\
\hline
1049&maður&nken&Þá var sá maður á vist með Eiríki er Skeggi hét Skinna-Bjarnarson \textbf{maður} \\
\hline
1050&maður&nken&Og átti Guðbrandur þar þing við þá og segir að sá maður var kominn á Lóar « er Ólafur heitir og vill bjóða oss trú aðra en vér höfum áður og brjóta goð vor öll í sundur og segir svo að hann eigi miklu meira goð og máttkara og er það furða er jörð brestur eigi í sundur undir honum er hann þorir slíkt að mæla eða goð vor láta hann lengur \textbf{maður} \\
\hline
1051&maður&nkfn&Hann var þá í úthlaupi því er liðsafnaður var á Hálogalandi og menn ætluðu til liðs við Þórólf svo sem fyrr var \textbf{menn} \\
\hline
1052&maður&nkfn&Um kveldið tóku menn til drykkju á Hóli og fóru menn í rekkjur eftir \textbf{menn} \\
\hline
1053&maður&nkfn&þeirra er menn hafi \textbf{menn} \\
\hline
1054&maður&nkfn&Nú ríða menn til hestavígs og er þar komið fjölmenni \textbf{menn} \\
\hline
1055&maður&nkfn&» Að liðnu matmáli Klaufa ríða tólf menn að \textbf{menn} \\
\hline
1056&maður&nken&En Sigurður jarl herjaði víða um sumarið um Skotland og frýði engi maður Þorsteini framgöngu og \textbf{maður} \\
\hline
1057&maður&nkfn&En er þings var kvatt og sett þá leitar konungur ráðs við lið sitt og kveður Gregoríus Dagsson og Erling skakka mág sinn og aðra lenda menn og skipstjórnarmenn og segir allan umbúnað þeirra Hákonar \textbf{menn} \\
\hline
1058&maður&nkeþg&» Hún skildi þó raunar hvað hann mælti til hjálpar manninum og þótti bæði skjótt og skörulegt hans úrræði og tók hún síðan búnaðinn af höfði sér og bjó hann með en settist í rúm hans að eigi gengju fleiri konur út en von \textbf{manninum} \\
\hline
1059&maður&nkfo&Eftir þessi tíðindi fór Þórir heim til bús síns og fóru þá menn í millum og varð griðum á komið um \textbf{menn} \\
\hline
1060&maður&nkfþ&Hann gekk þegar heim til bæjarins og inn í stofu og sagði þar öllum mönnum að hann margir menn að hann hafði lengi blindur verið því að hann hafði þar áður verið og gengið um \textbf{mönnum} \\
\hline
1061&maður&nkeþg&kvaðst það ætla að hann landi mundi eigi dæma menn hans til dauða og kvaðst svo skyldu koma öðru sinni að skyldi ná manninum og Þórður skyldi aldrei verri för farið \textbf{manninum} \\
\hline
1062&maður&nkfn&Og er þeir sáu að menn hljópu innan eftir firðinum þóttust þeir vita hverjir þar mundu vera og hugðu að Eyrbyggjar mundu vilja sækja fund \textbf{menn} \\
\hline
1063&maður&nkeo&Og er þeir komu fyrir þenna mann þá mælti hann til þeirra á norrænu og spyr hvaðan af löndum þeir \textbf{mann} \\
\hline
1064&maður&nkfþ&skipa þeim að vera með nokkurum góðum mönnum um \textbf{mönnum} \\
\hline
1065&maður&nkfn&Menn þeirra undruðust hví þeir höfðu svo skjótt skapskipti tekið þar sem þeir voru óglaðir er þeir fóru \textbf{Menn} \\
\hline
1066&maður&nkfn&Eftir þetta bjuggust menn brott að ríða og báðu vinir Páls að hann skyldi selja Sturlu \textbf{menn} \\
\hline
1067&maður&nkfo&Skulum vér fyrir því heldur hafa hinna höfðingja dæmi er oss eru kunnari og betra er eftir að líkja að berjast um ljósa daga og með fylking en stelast eigi um nætur á sofandi \textbf{menn} \\
\hline
1068&maður&nkfn&Og um morguninn er menn risu upp og fóru til kirkju er að messum kom sá Guðmundur að Þórlaug var eigi í \textbf{menn} \\
\hline
1069&maður&nkfþ&Ólafur konungur gaf heimleyfi mörgum mönnum sínum þeim er bú áttu og börn fyrir að hyggja því að þeim mönnum þótti ósýnt hver friður gefinn væri varnaði þeirra manna er af landi brott færu með \textbf{mönnum} \\
\hline
1070&maður&nkfn&Fundust eigi fremri menn þar í nánd \textbf{menn} jafnaldrar\\
\hline
1071&maður&nkfn&Broddi leysti Þorstein í brott með góðum gjöfum og er þeir fóru norðan um Smjörvatnsheiði þá hrapaði maður þeirra fyrir brekku nokkura og hló Þorsteinn að og margir menn er manninn sakaði \textbf{menn} \\
\hline
1072&maður&nkfn&En er bóndi hafði sagt sín erindi stóð jarl þegar upp og allir \textbf{menn} menn\\
\hline
1073&maður&nkfn&Margir hans menn löttu hann þess en eigi að síður fór hann með mikla sveit manna og kom til þess bæjar er drottning réð \textbf{menn} \\
\hline
1074&maður&nkfþ&Gera nú það ráð að þeir safna mönnum og þann dag er þeirra Finnboga var utan von komu saman að Hofi sex tigir \textbf{mönnum} \\
\hline
1075&maður&nkfn&Ásbjörn heyrði til að menn spurðu Þóri frá skiptum þeirra Ásbjarnar og svo það að Þórir sagði af langa sögu og þótti Ásbirni hann halla sýnt \textbf{menn} \\
\hline
1076&maður&nkfn&En um kveldið er menn komu heim til skála segir Egill svo að allir menn heyrðu hversu ferð hans hafði ætluð \textbf{menn} \\
\hline
1077&maður&nkfn&Menn hans náðu \textbf{Menn} \\
\hline
1078&maður&nkfn&Gissur talaði þá fyrir liðinu og eggjaði menn til \textbf{menn} \\
\hline
1079&maður&nkfn&» En er menn voru komnir til Húnavatnsþings þá hjó Brandur \textbf{menn} \\
\hline
1080&maður&nkfo&Eftir þetta sendi Snorri menn til Vatnsfjarðar og bauð þeim bræðrum suður fyrir föstu og var það erindi að hann vildi treysta þá til liðs við sig um sumarið \textbf{menn} \\
\hline
1081&maður&nkfþ&» Og er þeir höfðu barist lengi þá geta þeir gert Karl fráskila sínum mönnum og sækja sjö menn en hann hörfar undan þar til er hann kom til \textbf{mönnum} \\
\hline
1082&maður&nkeþ&En þó mun það einmælt að vitrum manni missýnist slíkt í meira lagi ef þú vilt alla hina herfilegstu menn með þér í sinni en þenna hinn dýrlega mann á móti þér er messudaginn á á morgun og göfgastur er nær allra guðs ástvina að vitni sjálfs \textbf{manni} \\
\hline
1083&maður&nkfn&Bjuggust menn þá til \textbf{menn} \\
\hline
1084&maður&nken&Það hið sama vor var bónorð í Svarfaðardal og bað maður konu sá er Snorri hét og var Grímsson vestan úr Skagafirði frá \textbf{maður} \\
\hline
1085&maður&nkfn&Mjög lögðu menn til orðs Þorkatli Eyjólfssyni er hann rak eigi þessa \textbf{menn} \\
\hline
1086&maður&nkfn&Konungur sendi til skips og þegar hann vissi hverjir menn að landi voru komnir gekk hann sjálfur til \textbf{menn} \\
\hline
1087&maður&nkfn&En er Magnús konungur kom til Suðureyja með her sinn þá flýði Lögmaður undan herinum og var í eyjunum en að lyktum tóku menn Magnúss konungs hann með skipsögn sína þá er hann vildi flýja til \textbf{menn} \\
\hline
1088&maður&nkfn&Menn biðu Böðvars Barkarsonar en hann kom eigi \textbf{Menn} \\
\hline
1089&maður&nken&Það er sagt að Össur er vitur maður og mjög hafður við mál \textbf{maður} \\
\hline
1090&maður&nken&Síðan lagði maður til Ólafs Klökkusonar og kom á hann \textbf{maður} \\
\hline
1091&maður&nkfn&Eftir það skipaði Sturla menn til að geyma hans en Svertingur var þar hjá \textbf{menn} \\
\hline
1092&maður&nkeo&að Vigdís frændkona hans hafði þenna mann sent honum til halds og trausts er þar var \textbf{mann} \\
\hline
1093&maður&nkeo&Þá sendu þeir Hrafn og Eyjólfur mann út til Hóla og fundust þeir biskup og Hrafn og Eyjólfur niðri í \textbf{mann} \\
\hline
1094&maður&nkfo&Eftir bardagann í Hestanesi fór Bolli heim með Ljóti á Völlu við alla sína menn en Ljótur bindur sár þeirra og greru þau skjótt því að gaumur var að \textbf{menn} \\
\hline
1095&maður&nkfn&En er konungur sótti upp á land þá buðu honum heim lendir menn og ríkir bændur og léttu svo \textbf{menn} kostnaði\\
\hline
1096&maður&nken&Það viljum vér og segja hversu Helgi Ásbjarnarson er kominn af landnámsmönnum er göfgastur maður er í þessari sögu að vitra \textbf{maður} virðingu\\
\hline
1097&maður&nken&Þar kom til þeirra sá maður er Hrafn hét og var kallaður \textbf{maður} \\
\hline
1098&maður&nkfn&Hann sagði og svo að þá voru aðrir eigi meiri menn á Íslandi en bræður hans er Sæmund leið en kallaði þá mundu mjög eftir sínum orðum víkja þá er hann kæmi \textbf{menn} \\
\hline
1099&maður&nkfþ&Sendi hann þá Árna Auðunarson til loftsins og bauð Þórði grið og öllum \textbf{mönnum} \\
\hline
1100&maður&nkeo&» Finnbogi sá hjá stólinum hvar stóð einn blámaður og þóttist hann eigi hafa séð \textbf{mann} mann\\
\hline
1101&maður&nken&Þar var þá með honum Ögmundur sneis og var hann þá á hinum átta tigi vetra og sögðu menn svo að hann þætti þar þá maður víglegastur í því \textbf{maður} \\
\hline
1102&maður&nkfþ&Nú ætlar Illugi menn til að gæta líkanna og vita ef þeir yrðu svo veiddir ef þeir gerðu eftir mönnum sínum og voru sex tigir manna eftir og gerðu tjald og sátu of \textbf{mönnum} \\
\hline
1103&maður&nkfþ&Þú munt mörgum mönnum til trúar koma og \textbf{mönnum} \\
\hline
1104&maður&nkfn&En þú ert mannfýla því meiri er þú liggur inni kyrr sem hundur á hvelpum þar sem húsbóndi þinn er lagður við velli og margir \textbf{menn} menn\\
\hline
1105&maður&nkfn&En er hvortveggi herinn sótti mjög til móts við annan þá gerðu lendir menn njósn úr hvorutveggja liði til frænda sinna og vina og fylgdi það orðsending hvorratveggju að menn skyldu gera frið milli \textbf{menn} \\
\hline
1106&maður&nkfo&Það var háttur Þorbjargar á vetrum að hún fór á veislur og buðu menn henni \textbf{menn} \\
\hline
1107&maður&nkfþ&En Aðalsteinn konungur safnaði herliði að sér og gaf mála þeim mönnum öllum er það vildu hafa til féfangs sér bæði útlenskum og \textbf{mönnum} \\
\hline
1108&maður&nken&En það er og bæði að eg hefi lítt til ráða hlutast og vilt þú að eg ráði litlu ef sá maður skal hér eigi vist hafa er eg hefi hingað \textbf{maður} \\
\hline
1109&maður&nkfþ&« að nú sé frændsemi okkar og vingan sem þá er best hefir verið og tökum vér upp leika eða aðra gleði eftir því sem öðrum mönnum er \textbf{mönnum} \\
\hline
1110&maður&nkfn&En er þeir hafa litla hríð talað þá koma þar menn Ólafs \textbf{menn} \\
\hline
1111&maður&nkfn&Og einn dag býr hann ferð sína og ríður til Hjalla og voru menn að \textbf{menn} \\
\hline
1112&maður&nkeo&það er Þórður vottaði að konungur hafði gefið honum orlof til Íslands og gera hann mestan \textbf{mann} \\
\hline
1113&maður&nkfn&Allir menn fagna honum vel og frétta hann tíðinda en hann segir slík sem voru og orðið höfðu í \textbf{menn} ferð\\
\hline
1114&maður&nkfn&En hans skaplyndi er það að hann er maður þrautgóður ef menn þurfa hans og enn mætti svo verða ef þú gerir þitt mál \textbf{menn} \\
\hline
1115&maður&nkfo&Gekk þá Þórður aftur á skip sitt og svo gerðu nú margir \textbf{menn} menn\\
\hline
1116&maður&nkeo&Þá þóttist eg höggva mann þeirra einn sundur í \textbf{mann} \\
\hline
1117&maður&nkeþ&Ari hét son hans og var hann nær tvítugum \textbf{manni} \\
\hline
1118&maður&nkfn&En þeir menn sem vanhluta þóttust verða fyrir þeim fóstbræðrum fóru á fund Vermundar og báðu hann koma af sér þessum \textbf{menn} \\
\hline
1119&maður&nkfn&Þeir menn voru með konungi er vissu skipti þeirra Bjarnar og Þórðar er þeir voru með Eiríki jarli og sögðu það konungi og hafði Björn gefið allt með \textbf{menn} \\
\hline
1120&maður&nkfo&« Þeir hafa drepið tvo hina vöskustu vora menn og skulum vér þeirra svo hefna að þessa skal \textbf{menn} \\
\hline
1121&maður&nkfn&Tóku flestir menn undir það léttlega og þótti þá von að bardagi mundi undan dragast en menn mundu fara mega frjálslega sér til nauðsynlegra \textbf{menn} um héraðið\\
\hline
1122&maður&nken&Hákon og Hildibrandur bróðir hans og Guðrún milli þeirra og töluðu lágt en Hrafn sat í bekk og reist spón því að hann var hagur \textbf{maður} \\
\hline
1123&maður&nkfn&Kolur bróðir hans fór með honum og nokkurir menn aðrir og voru þar um hríð og voru þeir mágar jafnan á \textbf{menn} \\
\hline
1124&maður&nkfn&Hann sendi menn suður til Borgar að segja Þorsteini Egilssyni þessi tíðindi og það með að hann vildi hafa styrk af honum til \textbf{menn} \\
\hline
1125&maður&nkfn&Komu þá að menn \textbf{menn} \\
\hline
1126&maður&nkfn&Þar var Flosi hálfan mánuð og menn hans og hvíldu \textbf{menn} \\
\hline
1127&maður&nkfþ&Og jafnan sýndi Áskell það að hann var fám mönnum líkur sakar réttdæmis er hann hafði manna í millum og drengskapar við hvern \textbf{mönnum} \\
\hline
1128&maður&nkfn&Um morguninn gengu menn til \textbf{menn} eftir vanda\\
\hline
1129&maður&nkfn&Gengu menn þá til skriftar og bjuggust \textbf{menn} \\
\hline
1130&maður&nken&Þá andaðist Oddi Þorgilsson og þótti það mikill mannskaði því að hann var vitur maður og manna snjallastur í \textbf{maður} \\
\hline
1131&maður&nkfn&Má það sjá að nær standa vinir Gunnars og mun sá verða málahluti vor bestur að góðir menn geri um ef Gunnar vill \textbf{menn} \\
\hline
1132&maður&nken&« Mikill maður og ríkmannlegur og er vænna að farinn sé að \textbf{maður} \\
\hline
1133&maður&nkfn&Hefur hann nú upp bannsetning við Órækju og hans menn \textbf{menn} \\
\hline
1134&maður&nkfþ&Sögðu menn Aðalsteins að tjöld þeirra væru öll full af mönnum svo að hvergi nær hefði þar rúm lið \textbf{mönnum} \\
\hline
1135&maður&nkfn&Menn Ólafs konungs voru út á Gaularási og héldu \textbf{Menn} \\
\hline
1136&maður&nkeþ&Er og önnur dóttir hans gift ríkum \textbf{manni} \\
\hline
1137&maður&nkfn&Þau misseri andaðist Ásgrímur ábóti og Þorvarður auðgi og þau misseri börðust menn að réttum í Flóa suður og þá var vígður Hrói biskup til \textbf{menn} \\
\hline
1138&maður&nkfn&« Eigi er einsætt Lambi að skerast svo skjótt undan ferðinni því að hér eiga stórir menn í hlut og þeir er mikils eru verðir en þykjast lengi hafa setið yfir skörðum \textbf{menn} \\
\hline
1139&maður&nkfþ&Nú hlaupa þeir til klæða sinna og vopnast og ganga síðan í dyr út og nú senda þeir eftir mönnum að veita þeim \textbf{mönnum} \\
\hline
1140&maður&nkfn&« Það er eigi sami að menn séu ósáttir á kaupskipum í hafi því að þar fylgir mart til meins og sjaldan mun þeim skipum vel farast er menn eru ósáttir \textbf{menn} \\
\hline
1141&maður&nkfn&Virtu margir menn sem þeir héldu þá til \textbf{menn} \\
\hline
1142&maður&nkeo&Nú sendu þeir Ásgrímur mann til Þórhalls og létu segja honum í hvert óefni komið \textbf{mann} \\
\hline
1143&maður&nkfn&Fór þar sem jafnan ef menn missa skjótlega sinna höfðingja að flestum verður bilt eftir að sækja sínum óvinum og fór þar og svo og fóru þeir bræður heim til búa \textbf{menn} \\
\hline
1144&maður&nkfn&Þau Ólafur og börn hans voru lengi um veturinn í Borg og var Borghildur jafnan á tali við konung og mæltu menn allmisjafnt um vináttu \textbf{menn} \\
\hline
1145&maður&nkeo&Síðan sendi hann mann austur á Rangárvöllu til Marðar gígju að búast við \textbf{mann} \\
\hline
1146&maður&nkfn&Og eigi miklu síðar komu þeir sömu menn á Hítarnes til \textbf{menn} og segja þetta\\
\hline
1147&maður&nkfn&Síðan reið Illugi í brott en menn hljópu eftir honum allt til bæjar þess er á Kroppi \textbf{menn} \\
\hline
1148&maður&nkfn&Gekk þessi maður í bátinn en Bjarni upp í skipið og er það sögn manna að Bjarni létist þar í maðkahafinu og þeir menn sem í skipinu voru með \textbf{menn} \\
\hline
1149&maður&nkfo&Ætla bændur eigi svo hneppt til jólaveislu sér að eigi verði stór afhlaup og drukku menn það herra lengi \textbf{menn} \\
\hline
1150&maður&nken&Þorgils átti land nær skipalægi og lá þar á hafnartollur og heimti sá maður skiptollinn er á landinu bjó og galt Ásgrímur aldrei toll þeim er á landi \textbf{maður} bjó\\
\hline
1151&maður&nkfn&Ólafur konungur heimti til máls við sig þá menn er komið höfðu af \textbf{menn} \\
\hline
1152&maður&nkfn&Og er hann kom á land var hann alvotur og sneri ofan aftur til Þingness og sex menn með \textbf{menn} \\
\hline
1153&maður&nkfþ&En þá náðist sannmæli af mörgum mönnum til Ólafs \textbf{mönnum} \\
\hline
1154&maður&nkfn&« Veit eg að þið eruð mikils \textbf{menn} menn\\
\hline
1155&maður&nkfo&setti menn sína til \textbf{menn} í Dyflinni\\
\hline
1156&maður&nken&Þá spurði konungur Ingigerði hver sá maður er í hans ríki er hún vill kjósa til \textbf{maður} við sig\\
\hline
1157&maður&nkeo&» Skeggi kvað hann furðu óvitran mann er hann talaði \textbf{mann} \\
\hline
1158&maður&nkfn&hversu lengi þetta mun vera að eigi verði menn til að biðja okkar eða hvað ætlar þú að fyrir okkur muni \textbf{menn} \\
\hline
1159&maður&nken&Þeir Guðmundur riðu fyrir en Rindill var eftir og maður einn hjá honum og fóru til matar þegar hesturinn var \textbf{maður} \\
\hline
1160&maður&nkfn&En þeir er féið ráku fóru til matar í Búðardal en sendu sex menn til \textbf{menn} í Hvarfsdal\\
\hline
1161&maður&nkfþ&Gissur og Ormur riðu til brúar og bað Gissur þann aldrei þrífast er eigi væri hjá öðrum \textbf{mönnum} \\
\hline
1162&maður&nkeo&« Undarlegt þykir mér um þig svo vitran mann og fyrirleitinn er þú skalt rasað hafa í svo mikla óhamingju og hafa fengið konungs reiði þar er engi bar nauðsyn \textbf{mann} \\
\hline
1163&maður&nken&« Von má hver maður þess vita ef hann á við sér ríkara mann og sitji samhéraðs honum og hafi þó gert honum nokkura ósæmd að hann mun eigi mörgum skyrtum slíta og kann eg því ekki að sýta þig að mér þykir þú mikið til hafa \textbf{maður} \\
\hline
1164&maður&nkfþ&Skipt var mönnum í sveitir til gerða en það vissi eg eigi hvað hverjir \textbf{mönnum} \\
\hline
1165&morgunn&nkeo&Og einn morgun um veturinn fyrir jól tók Geir á fótum \textbf{morgun} \\
\hline
1166&morgunn&nkeo&En er vora tók geta þeir að líta einn morgun snemma að fjöldi húðkeipa reri sunnan fyrir \textbf{morgun} \\
\hline
1167&morgunn&nkeog&» Morguninn eftir gekk Skegg-Broddi til búðar Eyjólfs og var engi blíða við hann af \textbf{Morguninn} \\
\hline
1168&morgunn&nkeog&Og um morguninn gengur Kolur út úr tjaldi og sér hvergi skipið og við það leggst hann niður og vill eigi segja Þorgilsi og þykir áður ærinn harmur \textbf{morguninn} \\
\hline
1169&morgunn&nkeo&Það bréf skaltu láta koma á morgun fyrir brjóst þér og vefja dúkinum að utan og um búk þér svo sem hann \textbf{morgun} \\
\hline
1170&morgunn&nkfo&Þorkell gætti heima andvirkis um morgna en Sigmundur fylgdi \textbf{morgna} \\
\hline
1171&morgunn&nkeog&En um morguninn er Áslákur vildi ganga til skips síns þá veitti Vígleikur honum atgöngu og vildi hefna \textbf{morguninn} \\
\hline
1172&morgunn&nkeog&Um morguninn eftir lét Vésteinn bera að sér töskur er varningur hans var og hann hafði þeim Hallvarði fengið í hendur með að \textbf{morguninn} \\
\hline
1173&morgunn&nkeog&En um morguninn var séð fyrir hestum \textbf{morguninn} \\
\hline
1174&mál&nheþ&vestan úr Skagafirði úr sveit Gríms Snorrasonar og hann gekk með þessu \textbf{máli} \\
\hline
1175&mál&nhfo&sem áður var á kveðið og lagði við þrjú hundruð og lét færa honum og sendi þau orð með að Ögmundur skyldi eigi mótstöðumaður hans vera um þessi mál er nú höfðu \textbf{mál} \\
\hline
1176&mál&nheog&Björn vill það eigi og koma til þings og sættust þar á málið og hlaut Björn að gjalda þrjár merkur silfurs fyrir níðreising og \textbf{málið} \\
\hline
1177&mál&nhen&Það er allra \textbf{mál} mál\\
\hline
1178&mál&nheo&Mart ræddu þau um þetta mál og urðu á allar ræður sátt sín í \textbf{mál} \\
\hline
1179&mál&nheo&Vörðu þeir mál fyrir Odd með styrk þeirra \textbf{mál} úr Hrútafirði\\
\hline
1180&mál&nken-s&« Mál þykir mér að vitja manna \textbf{Mál} \\
\hline
1181&mál&nheþ&» Veik konungur þessu máli nokkuð til ýmissa manna að til þeirrar ferðar skyldu ráðast en þar komu þau svör í mót að allir menn töldust undan \textbf{máli} \\
\hline
1182&mál&nhfo&« að hún er sú kona er eg ætla mér að biðja og vildi eg að þessi mál kæmir þú fyrir mig við föður hennar og legðir á alendu að flytja því að eg skal þér fullkomna vináttu fyrir \textbf{mál} \\
\hline
1183&mál&nheng&Og er dæmt var málið þá nefndi Eyjólfur sér votta og kallaði ónýttan dóm þeirra og allt það er þeir höfðu að \textbf{málið} \\
\hline
1184&mál&nheo&En Andrés þóttist eigi mega deila um mál \textbf{mál} við Sturlu\\
\hline
1185&mál&nheo&kveðst vilja sættast við Órækju og með því einu efni að hann gerði einn um mál þeirra öll og til skildar utanferðir þeirra Órækju og Sturlu og skyldu þeir vera í valdi Kolbeins og Gissurar þar til er þeir færu \textbf{mál} \\
\hline
1186&mál&nhen&« Þetta mál skal fara óvélt af minni hendi því að á er ljóður mikill um ráð \textbf{mál} \\
\hline
1187&mál&nheþ&Fór Kolur þá til fundar við Björn og sótti hann að sínu \textbf{máli} \\
\hline
1188&mál&nheog&Og er á leið þingið spurði Brodd-Helgi hvar komið var um málið Þorleifs hins \textbf{málið} \\
\hline
1189&mál&nheo&Hann bað Þorgils hér fulltings um þetta \textbf{mál} \\
\hline
1190&mál&nhen&« Yðvart mál horfir til mikilla vandræða af tilfellum \textbf{mál} \\
\hline
1191&mál&nhen&« Það er sannlegast að mál þetta komi í lögmanns dóm og hefir Þórður það að gert er hann átti er hann batt þann er stolið hafði ella var hann \textbf{mál} \\
\hline
1192&mál&nheo&Björn svarar fyrir þeim á alþingi og kvað Þórð nú með réttu ganga og satt mæla og kvaðst eigi vilja synja laga um þetta mál og kvaðst vilja bæta fé fyrir þetta \textbf{mál} \\
\hline
1193&mál&nheþ&Klerkurinn sótti biskup að sínu máli en Kolbeinn kallar eftir og vill eigi \textbf{máli} dóm\\
\hline
1194&mál&nheþ&» Þá var sendur skilríkur maður til Þórhalls og sagði sá honum vandlega frá hvar þá var komið \textbf{máli} \\
\hline
1195&mál&nhfþ&En þess vil eg biðja þig að þú veitir mér að þessum \textbf{málum} \\
\hline
1196&mál&nheþ&ríður til dyra og hittir menn að \textbf{máli} \\
\hline
1197&mál&nhen&Það mál var upp borið fyrir liðsmenn hvort þar skyldi ráða til \textbf{mál} eða eigi\\
\hline
1198&mál&nhfog&« Eg hefi athugað málin og sýnist mér sem eigi hæfi ágangur við þá sem goldið hafa of fjár og flýið land \textbf{málin} \\
\hline
1199&mál&nheog&En þar kom að Gissur tók undir sig málið og lýsti sök að Lögbergi og kvað svo að orði að « eg lýsi lögmætu frumhlaupi á hönd Gunnari Hámundarsyni um það er hann hljóp lögmætu frumhlaupi til Þorgeirs Otkelssonar og særði hann holundarsári því er að ben gerðist en Þorgeir fékk bana \textbf{málið} \\
\hline
1200&mál&nheo&» En með áeggjun þá bjó Þórarinn Þórisson mál til alþingis á hendur Glúmi um víg Sigmundar en Glúmur bjó mál til á hendur Þorkatli hinum háva um illmæli við þrælana og annað bjó hann á hönd Sigmundi og stefndi honum um stuld og kvaðst hann drepið hafa á eign sinni og stefnir honum til óhelgi er hann féll á hans eign og gróf Sigmund \textbf{mál} \\
\hline
1201&mál&nheog&Gerði eg þá fyrir guðs sakir að gefa honum upp allt \textbf{málið} \\
\hline
1202&mál&nheþg&Svo lauk með þeim að Gunnar tók við málinu en fékk henni fé til bús síns sem hún þurfti og fór hún heim \textbf{málinu} \\
\hline
1203&mál&nheo&Viljum vér svo til lykta færa vort mál að leggja þetta fyrir vora hönd í þann dóm er vænst er að best megi haldast og sjatna mætti \textbf{mál} \\
\hline
1204&mál&nhfog&En þá er þeir Órækja og menn hans töluðust við um málin \textbf{málin} \\
\hline
1205&mál&nheog&Glúmur hefur málið á hendur Þorkatli og komu sakar í \textbf{málið} \\
\hline
1206&mál&nhfþ&« Þar muntu enn koma til Íslands og skipta málum við hann og mun þér það betur \textbf{málum} \\
\hline
1207&mál&nheo&» Nú ríður Glúmur heim til Þverár með sínum mönnum og ekki bjó hann þetta mál \textbf{mál} \\
\hline
1208&mál&nhen&En þeir lögðu það til að hann byggi mál til á hendur Þorvaldi og þeim mönnum er neytt höfðu af hvalnum til Dýrafjarðarþings og sæki þar að \textbf{mál} \\
\hline
1209&mál&nhen&» Síðan reið Snorri heim og segir þeim bræðrum hvert orðið hafði hans erindi og svo það að hann mundi við skiljast þeirra mál með öllu ef þeir vildu eigi játa \textbf{mál} \\
\hline
1210&mál&nheo&En þeim sýndist að vinna honum og þeim bræðrum trúnaðareiða að skiljast eigi við \textbf{mál} mál\\
\hline
1211&mál&nheo&Sigvaldi jarl lagðist þá ferð eigi undir höfuð og fer á fund Sveins Danakonungs og ber þetta mál fyrir hann og kemur jarl svo fortölum sínum að Sveinn konungur fær í hendur honum Þyri systur sína og fylgdu henni konur nokkurar og fósturfaðir hennar er nefndur er Össur \textbf{mál} \\
\hline
1212&mál&nheþ&segir að ámælissamt mun verða nema þeir fylgi máli mágs \textbf{máli} \\
\hline
1213&mál&nheog&Það varð til tíðinda að Hænsna-Þórir hvarf brott úr héraðinu við tólfta mann þegar hann spurði hverjir í málið voru komnir og fréttist alls eigi til \textbf{málið} \\
\hline
1214&mál&nheog&« Fyrir löngu sá eg það en svo mikil þykir mér nauðsyn á um málið Odds að eg veit eigi hvort eg nenni að vera í mót \textbf{málið} \\
\hline
1215&mál&nheo&Þóroddur sótti Snorra goða að eftirmáli um víg Þórissona og lét Snorri búa mál til Þórsnessþings en synir Þorláks á Eyri veittu Breiðvíkingum að málum \textbf{mál} \\
\hline
1216&mál&nheo&Eg vil vita hverju þú vilt svara fyrir mál það er þú tókst upp hey \textbf{mál} \\
\hline
1217&mál&nheo&Því vill Þorsteinn það mál uppi hafa við þig að biðja til handa Þorkatli Geitissyni Jórunnar dóttur \textbf{mál} \\
\hline
1218&mál&nhfog&En Þorgeir biskupsson kvað það skyldu fyrir sættum standa ef eigi fylgdu þar málin Guðmundar og sekt Koll-Odds og bjargir og sýndi svo mikla ást og einurð við hann í þessu máli að engi kostur var sætta \textbf{málin} \\
\hline
1219&mál&nhfþ&segir henni þá af trúnaði frá ferð þeirra Bjarnar og spyr hvað hún hyggur hvernug Svíakonungur muni taka þeim málum að sætt væri ger milli þeirra \textbf{málum} \\
\hline
1220&mál&nhen&Fæð hefir verið á með þeim en vér hyggjum að lygi hafi verið og \textbf{mál} mál\\
\hline
1221&mál&nheo&Drottning ræddi þetta mál við konung og segir að Hörða-Knútur sonur þeirra vildi bæta öllu því sem konungur vildi ef hann hefði það gert er konungi þætti í móti \textbf{mál} \\
\hline
1222&mál&nheog&Voru þar margir frændur og vinir Gísls og ræddu um málið hverja meðferð hafa \textbf{málið} \\
\hline
1223&mál&nhen&Var það og allra manna mál að engi hefði slíkur maður komið af Íslandi sem \textbf{mál} \\
\hline
1224&mál&nheo&Nú ríður Guðmundur til þings og svo aðrir menn og hefir Guðmundur fram mál á hönd \textbf{mál} \\
\hline
1225&mál&nheog&Töluðu þeir nú um málið og var það ráðið að Indriði skal eiga Þorbjörgu og skal hann hafa með henni fjóra tigu hundraða og skyldi þegar vera brúðkaupið að \textbf{málið} \\
\hline
1226&mál&nheo&En þessi mál lukust svo að um haustið Máritíusmessu sættust þeir Sigurður við biskup og lögðu sitt mál allt á \textbf{mál} dóm\\
\hline
1227&mál&nhen&En Þórður kveðst svo oft mundi hætta verða í óvænt efni ef nokkuð skyldi að vinnast um \textbf{mál} mál\\
\hline
1228&mál&nhfng&» Urðu þær nú málalyktir með ráði hinna vitrustu manna að málin voru öll lagið í \textbf{málin} \\
\hline
1229&mál&nhen&kvað lokið því héðan af að hann mundi hafa nokkura sorg eða angur í sínu hjarta og þykja eigi þann veg vel sem af reiddi \textbf{mál} mál\\
\hline
1230&mál&nhen&» Snorri kvaðst það banna mundu að fara á hendur þeim mönnum er mest voru virðir í héraði « en náfrændur þeirra er nær munu ganga hefndunum og er allt mál að ættvíg þessi takist \textbf{mál} \\
\hline
1231&mál&nheo&Sigríður sendi orð jarlinum og lét segja hvar þá var komið um mál \textbf{mál} \\
\hline
1232&mál&nheþ&Ölvir hnúfa var þá nær staddur og bað konung vera eigi reiðan « eg mun fara á fund Kveld-Úlfs og mun hann vilja fara á fund yðvarn þegar er hann veit að yður þykir máli \textbf{máli} \\
\hline
1233&mál&nhfog&En er leið að þinglausnum þótti mönnum ófriðlegt ef svo búin færu mál til héraðs og áttu menn þá hlut að og varð þá sæst á málin og skyldi Klængur biskup gera og Böðvar Þórðarson og var þá þegar upp lokið og þótti Sturlu verða gerðir skakkar og \textbf{málin} \\
\hline
1234&mál&nhfog&Síðan varð komið á með þeim griðum og lögðu þeir málin til \textbf{málin} \\
\hline
1235&mál&nhen&Hefi eg fyrir þá sök þetta mál fyrir engan mann borið fyrr en þig að eg veit að þú ert maður vitur og kannt góða forsjá til þess hvernug reisa skal frá upphafi þessa \textbf{mál} \\
\hline
1236&mál&nheþg&þinn eða þeirra annarra er í eru \textbf{málinu} \\
\hline
1237&móðir&nven&Þorleif var móðir \textbf{móðir} \\
\hline
1238&móðir&nven&Móðir Leifs var \textbf{Móðir} \\
\hline
1239&móðir&nven&En er hún vitkaðist mælti Þorbjörg móðir \textbf{móðir} \\
\hline
1240&móðir&nven&móðir \textbf{móðir} \\
\hline
1241&móðir&nven&móðir Þorsteins \textbf{móðir} \\
\hline
1242&móðir&nven&Þá kom og til hirðarinnar Álfhildur móðir Magnúss \textbf{móðir} \\
\hline
1243&móðir&nven&móðir Ásbjarnar \textbf{móðir} \\
\hline
1244&móðir&nven&En móðurkyn mitt vænti eg að þér munuð séð hafa fleira en eg því að Melkorka heitir móðir mín og er mér sagt með sönnu að hún sé dóttir þín konungur og það hefir mig til rekið svo langrar ferðar og liggur mér nú mikið við hver svör þú veitir voru \textbf{móðir} \\
\hline
1245&móðir&nven&Móðir Ólafs trételgju hét Gauthildur en hennar móðir Ólöf dóttir Ólafs hins skyggna \textbf{móðir} af Næríki\\
\hline
1246&móðir&nven&En nú hefir Ólafur konungur fengið Ástríðar en þó að hún sé konungsbarn þá er ambátt móðir \textbf{móðir} og þó vindversk\\
\hline
1247&móðir&nven&Gissur hvíti sonur Teits Ketilbjarnarsonar en móðir hans var Ólöf dóttir Böðvars hersis \textbf{móðir} \\
\hline
1248&móðir&nven&Hún var systir Þorbjarnar en móðir \textbf{móðir} \\
\hline
1249&móðir&nven&móðir þeirra Þórðar og \textbf{móðir} \\
\hline
1250&móðir&nven&Hann kom út nokkuru síðar en móðir hans og var með henni hinn fyrsta \textbf{móðir} \\
\hline
1251&móðir&nven&Gunnhildur móðir þeirra hafði mjög landráð með \textbf{móðir} \\
\hline
1252&móðir&nven&móðir \textbf{móðir} \\
\hline
1253&móðir&nven&Móðir hans hét Hróðný og var \textbf{Móðir} \\
\hline
1254&móðir&nven&Móðir Hámundar hét \textbf{Móðir} \\
\hline
1255&móðir&nveo&En er þeir fundu Gunnhildi móður sína sögðu þeir alla atburði um þessi tíðindi er þá höfðu gerst í för \textbf{móður} \\
\hline
1256&móðir&nveo&En því var hann kenndur við móður sína að hún lifði lengur en faðir \textbf{móður} \\
\hline
1257&móðir&nven&Móðir hans sagði að hann var dauður og hefði orðið \textbf{Móðir} \\
\hline
1258&móðir&nveo&Og nú bað Hávarður móður sína gefa karli mat en hann bjóst heiman og fór á fund Skútu vinar síns og sagði honum um ferðir Glúms það sem hann vissi \textbf{móður} \\
\hline
1259&móðir&nven&móðir Þorkels \textbf{móðir} í Krossavík\\
\hline
1260&móðir&nven&móðir \textbf{móðir} \\
\hline
1261&móðir&nven&að aldri og nam fyrir utan Þjórsá milli Rauðár og Þjórsár og upp til Skúfslækjar og Breiðamýri hina eystri upp til Súluholts og bjó í Gaulverjabæ og Oddný móðir \textbf{móðir} \\
\hline
1262&móðir&nveþ&Synir Vésteins fara til Gests frænda síns og skora á hann að hann komi þeim utan með ráðum sínum og Gunnhildi móður þeirra og Auði er Gísli hafði átta og Guðríði Ingjaldsdóttur og Geirmundi bróður \textbf{móður} \\
\hline
1263&móðir&nven&Var móðir mín vel mönnuð en föður átti eg heldur \textbf{móðir} \\
\hline
1264&móðir&nven&Leiddu þeir hana í brott en móðir hans bað konurnar vera eigi svo djarfar að þær gerðu vart við inn í \textbf{móðir} \\
\hline
1265&móðir&nven&Móðir hans hét Þórunn og var \textbf{Móðir} \\
\hline
1266&móðir&nven&Dóttir Eyjólfs hins halta var Þórey móðir \textbf{móðir} \\
\hline
1267&móðir&nven&móðir \textbf{móðir} \\
\hline
1268&móðir&nven&Móðir hennar var Róðbjartur \textbf{Móðir} \\
\hline
1269&móðir&nven&Móðir hans var \textbf{Móðir} \\
\hline
1270&móðir&nven&Móðir Böðvars var Valgerður dóttir Markúss \textbf{Móðir} \\
\hline
1271&nótt&nveo&Þessi minning var nálega hverja nótt frá páskum til \textbf{nótt} \\
\hline
1272&nótt&nveog&Oddur lá í almannastofu um nóttina og margt manna svo að stofan var \textbf{nóttina} \\
\hline
1273&nótt&nveog&Hann sendi þegar um nóttina Magnús Kollsson til Sauðafells að segja Sturlu um ferðir þeirra bræðra sem hann hafði \textbf{nóttina} \\
\hline
1274&nótt&nvfo&Veðrið hélst þrjár nætur og er upp létti þóttust menn engan stað sjá \textbf{nætur} \\
\hline
1275&nótt&nveo&« Eg ók í nótt eftir viði og fann eg sex menn á leið og voru það húskarlar Ármóðs og var það miklu fyrir \textbf{nótt} \\
\hline
1276&nótt&nveog&» Þann veg var næturbjörg þeirra að sumir komust úr brókum og héngu þær um nóttina á þili frernar og lögðust þá til \textbf{nóttina} \\
\hline
1277&nótt&nvfo&Nú hvíl þig hér fyrst þrjár nætur og vita þá hversu Frey þóknist til \textbf{nætur} \\
\hline
1278&nótt&nveog&Og um nóttina trúði hann sér eigi til vöku fyrr en hann lét glóð undir fætur \textbf{nóttina} \\
\hline
1279&nótt&nvfn&Þeir voru fáar nætur á Flugumýri áður Kolbeinn reið með þá norður til \textbf{nætur} \\
\hline
1280&nótt&nven&Þá nótt áður hafði fallið lítil snæfölva svo að sporrækt \textbf{nótt} \\
\hline
1281&nótt&nveþ&En á næstu nótt eftir þenna fund skiptist svo skjótt um með guðlegri forsjá að um morguninn eftir var á brottu allur grimmleikur frostsins en kominn í staðinn hlær sunnanvindur og hinn besti \textbf{nótt} \\
\hline
1282&nótt&nvfn&En við þessi orð Órækju varð Sighvatur ekki búinn til handsalanna og voru þeir þar tvær \textbf{nætur} \\
\hline
1283&nótt&nveo&« Hér er ey í milli og er eg vanur að vera þar um nótt þá er eg fer norðan en þá kem eg heim annan morgun til dagverðardrykkju \textbf{nótt} \\
\hline
1284&nótt&nveo&Hann kom um nótt á bæ Þorvalds og segir mönnum sínum að þar muni gott fang í hendur bera er auðigur maður bjó fyrir « og skulum vér bera eld að \textbf{nótt} \\
\hline
1285&nótt&nvfþ&Var þá ákveðin brúðlaupsstefna að Stað með jarli er Sturla skyldi gifta dóttur sína Ingibjörgu Þórði syni Þorvarðs úr Saurbæ tveim nóttum fyrir \textbf{nóttum} \\
\hline
1286&nótt&nveog&En þar kom um síðir að þeir söfnuðu að sér mönnum og riðu tuttugu saman vestur til Dala og komu á Höskuldsstaði og tók Höskuldur vel við þeim og voru þeir þar um \textbf{nóttina} \\
\hline
1287&nótt&nvfn&Og er liðnar voru þrjár nætur kom Ingimundur til \textbf{nætur} \\
\hline
1288&nótt&nvfn&Þá voru þrjár nætur til þess er brúðlaupið mundi \textbf{nætur} \\
\hline
1289&nótt&nven&Það var eina nótt þá er Magnús konungur lá í hvílu sinni að hann dreymdi og þóttist staddur þar sem var faðir \textbf{nótt} \\
\hline
1290&orð&nhfþ&Njálssynir fréttu Ketil en hann kveðst fátt mundu frá herma orðum þeirra « en það fannst á að Þráni þótti eg mikils virða mágsemd við \textbf{orðum} \\
\hline
1291&orð&nhfn&» orð meyjarinnar og telja að hún er furðu djörf og óvitur og segja það maklegt að konungur sendi lið mikið eftir henni við \textbf{orð} \\
\hline
1292&orð&nheo&Sendimenn biðja enn fresta um þrjá daga og þess með að Ólafur konungur sendi þá menn sína að heyra orð Aðalsteins konungs hvort hann vill eða eigi þenna \textbf{orð} \\
\hline
1293&orð&nhfo&En Hrafnssynir og Jónssynir riðu ofan í fjöruna og skorti þar eigi stór orð og eggjan er hvorir mæltu til \textbf{orð} \\
\hline
1294&orð&nheo&En fyrir því að Þórarinn var skyldur mjög Klaufa þá sendi hann honum orð og fer Klaufi þegar inn þangað og fundu þeir Hrana á akri Þórarins með féð sitt og sat hann á hesti sínum með \textbf{orð} \\
\hline
1295&orð&nhfþ&« að fara með bónorði til hans að hann láti af komum hingað og leggja þökk og aufúsu á og láta allvesallega en hafast ekki að þótt eigi sé látið að orðum \textbf{orðum} \\
\hline
1296&orð&nhfþ&« Það ætla eg að Guðmundur hyggi að reka þess fjandskapar við þig er honum er sagt frá orðum \textbf{orðum} \\
\hline
1297&orð&nhfn&« veit eg að þau orð eru komin upp á Arneiðarstaði til eyrna Droplaugar og sonum \textbf{orð} \\
\hline
1298&orð&nhfþ&« Koma mun eg orðum þeim er Sörli lagði fyrir mig sem er að biðja Þórdísar dóttur \textbf{orðum} \\
\hline
1299&orð&nhfn&Er hitt bæn mín og vilji að þér konungur farið að heimboði til mín og heyrið þá orð þeirra manna er þú \textbf{orð} \\
\hline
1300&orð&nheo&Gissur reið vestur með flokkinn til móts við Kolbein og sendu þeir orð Böðvari til Staðar að hann færi á fund \textbf{orð} \\
\hline
1301&orð&nhfo&Þætti mér það ráð fyrir liggja faðir að þú sendir menn til Noregs að bjóða sættir fyrir Björn og mun Þórir mikils virða orð \textbf{orð} \\
\hline
1302&orð&nheo&Reið Snorri goði þaðan suður yfir heiði og gerði það orð á að hann mundi ríða til \textbf{orð} í Hraunhafnarós\\
\hline
1303&orð&nheo&« Það er nú sýnt Arnór að sá hinn sami guð er þú kvaddir að þínu máli hefir sinn helgan anda sent í þitt brjóst til að byrja svo blessaðan manndóm sem þú hefir mönnum nú tjáð í tölu þinni og það hygg eg ef Ólafur konungur hefði þig heyrt slík orð segja að hann mundi gera guði þakkir og þér fyrir svo fagran framburð og því trúi eg að þá er hann spyr þvílíka hluti að hann verði forkunnar feginn og víst er oss það mikill skaði að vér skulum hann eigi mega sjá eða heyra hans orð sem mér þykir ugganda að hvorki \textbf{orð} \\
\hline
1304&orð&nhfo&« Hitt hugði eg að engi skyldi brjóta mín boð en fyrir þín orð Finnur þá má eg ganga á fund Egils og víst vildi eg þess guð biðja að hann léði honum líf til þess að eg mætti hefna honum og fá fullt víti fyrir það að hann \textbf{orð} \\
\hline
1305&orð&nhfo&» Síðan kemur Skeggi fyrir konung og segir honum þessi orð \textbf{orð} \\
\hline
1306&orð&nhfo&« Til mun eg leggja orð mín ef þér mætti lið að verða en annars staðar skaltu til leita um vistaferli þín en hjá \textbf{orð} \\
\hline
1307&orð&nheo&Þá er þeir húskarlar og Kolskeggur höfðu verið þrjár nætur í Eyjum þá hefir Þorgeir Starkaðarson njósn af þessu og gerir orð nafna sínum að hann skyldi koma til móts við hann á \textbf{orð} \\
\hline
1308&orð&nhfo&En við þessi orð Órækju varð Sighvatur ekki búinn til handsalanna og voru þeir þar tvær \textbf{orð} \\
\hline
1309&orð&nheo&gera nú skyndilega orð til \textbf{orð} \\
\hline
1310&orð&nheo&Ólafi þótti illa orðið hafa og þótti Hallfreður ótrúlegur að halda sættir og sendi orð Óttari að honum leist \textbf{orð} líklegt\\
\hline
1311&orð&nheo&Sörli kvaðst mundu það gera eftir hugþokka sínum en hirða ekki um orð \textbf{orð} \\
\hline
1312&ráð&nhen&« Hér er Halldór kominn og búinn til hafs og kominn á byr og er nú ráð að gjalda \textbf{ráð} \\
\hline
1313&ráð&nheo&Tveir menn af liði Harðar sneru í móts við griðunginn og brugðu á sitt \textbf{ráð} \\
\hline
1314&ráð&nhfn&Grímkell fer og biður konunnar og flytur vel og með hans framkvæmd þá takast þessi ráð með þeim Grími og \textbf{ráð} \\
\hline
1315&ráð&nhen&Nú töluðu þeir um höfðingjarnir að ráð væri að fara eftir Helgu og drepa sonu þeirra \textbf{ráð} \\
\hline
1316&ráð&nheo&« Eigi sé eg hér ráð til þau er góð séu en eg mun ekki þig um þetta ávíta fyrir því að mæla verður nokkur \textbf{ráð} \\
\hline
1317&ráð&nheþ&Hann fór á fund Sveins tjúguskeggs Danakonungs og bað til handa sér Gyðu dóttur hans og var það að ráði \textbf{ráði} \\
\hline
1318&ráð&nheo&« Mun nú eigi ráð að þú takir til \textbf{ráð} \\
\hline
1319&ráð&nhfþ&» Gunnar taldi á hann langa hríð en Sigmundur svaraði honum vel og kvaðst meir hans ráðum skyldu fram fara þaðan af en þar til hafði \textbf{ráðum} \\
\hline
1320&ráð&nheþ&öll með ráði \textbf{ráði} \\
\hline
1321&ráð&nheo&« Það sýnist mér ráð að þú leggist niður til svefns en rís upp ofanverða nátt og stíg þá á bak hesti þínum og ríð vestur til \textbf{ráð} \\
\hline
1322&ráð&nheþ&Um veturinn eftir jól fór Loftur vestur á Mýrar að ráði Sighvats Sturlusonar og Eyjólfs mágs \textbf{ráði} \\
\hline
1323&ráð&nheþ&Gísl fór til bæjarins og gerði það bragð á með ráði Hákonar húsbónda síns að hann lét steypa heitu vaxi á andlit sér og lét þar harðna \textbf{ráði} \\
\hline
1324&ráð&nhen&« Er nokkur sá maður í mínu liði er ráð kunni til þess að leggja að vér komumst á burt svo að bændur eigi öngan kost á oss því að eg veit að þeir munu þegar að oss leggja er lýsir ef vér bíðum \textbf{ráð} \\
\hline
1325&ráð&nhen&« ef nokkurir ráða til að rjúfa safnaðinn að þá muni brátt þynnast fylkingar þeirra því að svo er bóndum gefið að það ráð er þá er \textbf{ráð} \\
\hline
1326&ráð&nhfo&Buðu þeir bræður þeim mönnum grið sem eftir voru en með því þeir höfðu eigi önnur ráð þá gengu þeir allir Gunnari á hendur og Helga og sóru þeim \textbf{ráð} \\
\hline
1327&ráð&nhfþ&En mínum ráðum vil eg láta fram fara hvert tilstilli hafa skal en eg vil eigi hlaupa eftir ákafa Bjarnar eða annars manns um svo mikil \textbf{ráðum} \\
\hline
1328&ráð&nhen&« Ekki ætla eg þenna vel til sektar fallinn eða mun eigi hitt heldur ráð að við gerum hann feginn og látum hann kjósa menn til gerðar \textbf{ráð} \\
\hline
1329&ráð&nheþ&Þessa iðn hefir hann nú fyrir stafni nokkura hríð og svo kemur hans ráði að hann á einn knörrinn og mestan hluta \textbf{ráði} \\
\hline
1330&ráð&nheþ&Eftir þetta tóku menn að byggja í landnámi Helga að \textbf{ráði} ráði\\
\hline
1331&ráð&nhfo&Björgólfur var þá gamall og önduð kona hans og hafði hann selt í hendur öll ráð syni sínum og leitað honum \textbf{ráð} \\
\hline
1332&ráð&nheo&Kerling hafði ráð fyrir liði þeirra og hún hafði huliðshjálm yfir skipinu meðan þau reru yfir fjörðinn til \textbf{ráð} \\
\hline
1333&ráð&nheþ&skyldu þeir þá stafa fyrir þeim slíkt er þeir vildu með ráði \textbf{ráði} \\
\hline
1334&ráð&nheo&fyrst austur í Svíaveldi og gera þá ráð sitt hvert hann ætlar eða sneri þaðan af en bað svo vini sína til ætla að hann mundi enn ætla til landsins að leita og aftur til ríkis síns ef guð léði honum \textbf{ráð} \\
\hline
1335&ráð&nken-s&« Ráð sjálfur höfði þínu en gakk inn til borðs í rúm \textbf{Ráð} \\
\hline
1336&ráð&nheþ&» Hann nam land að ráði Flosa fyrir ofan Víkingalæk og út til \textbf{ráði} við Svínhaga\\
\hline
1337&ráð&nhen&Það var þeirra ráð bræðra að þeir komu báðir til Staðar um leið er Kálfur reið norður yfir \textbf{ráð} \\
\hline
1338&ráð&nhfþ&Og eftir leitar hann við Þorberg ef þetta væri nokkuð af hans ráðum en hann kvað það fjarri fara og kvað Auðgísl hafa stolið frá sér gripunum og barið á honum á svo gert \textbf{ráðum} \\
\hline
1339&ráð&nheþ&Og hvort sem um þetta var talað lengur eða skemur þá var það að ráði gert að drottning var gift Hjörvarði jarli með ráði konungs sonar \textbf{ráði} \\
\hline
1340&ráð&nhfþ&Voru nú sett fullkomin grið milli þeirra Sæmundar og Ög mundar og veittar fullar tryggðir af góðvilja og ráðum Brands ábóta og Steinunnar húsfreyju og Álfheiðar móður Sæmundar og margra annarra góðra \textbf{ráðum} tillögu\\
\hline
1341&ráð&nheo&Sýnist mér það ráð að Þorgeir sofi fram hjá oss í skála en Butraldi og hans förunautar sofi hér í \textbf{ráð} \\
\hline
1342&ráð&nheþ&En það er og bæði að eg hefi lítt til ráða hlutast og vilt þú að eg ráði litlu ef sá maður skal hér eigi vist hafa er eg hefi hingað \textbf{ráði} \\
\hline
1343&ráð&nheþ&Nú vil eg að þú bregðir þessu ráði og svo föður þíns og Kraka mágs þíns og ráðist allir til utanferðar með allt það er þér eigið því að eg ætla Helga frænda mínum og fóstra gerast mjög þungt til \textbf{ráði} \\
\hline
1344&sinn&fevfo&Vildi hann að Þorgils kæmi til móts við hann og töluðu um ráðagerðir \textbf{sínar} \\
\hline
1345&sinn&fekeþ&Bersi fór byggðum á Laugaból í Laugadal því að Vermundur vildi eigi svo nær bæ sínum láta vera hráskinn þeirra Þorgeirs og \textbf{sínum} \\
\hline
1346&sinn&fekfo&Nú kom Grís til seljanna með sína menn og voru þeir Hallfreður þá í \textbf{sína} \\
\hline
1347&sinn&fekfo&Þórir ríður nú heiman með félaga sína og kemur til Eyvindarár og drepur þar á dyr og bað Gróu til hurðar \textbf{sína} \\
\hline
1348&sinn&feveo&En þó gekk hann að Sturlu við umtölur manna og bað hann hafa þökk fyrir stilling sína er hann hafði þar gert á því \textbf{sína} \\
\hline
1349&sinn&nheo&Og eitt sinn er frá því sagt að búsmali hans hafði þar komið niður um nótt \textbf{sinn} \\
\hline
1350&sinn&fehfþ&Menn Þórðar fóru jafnan vestur í sveitir um veturinn að erindum sínum og hann sendi orð vinum sínum þeim er hann vildi að norður \textbf{sínum} \\
\hline
1351&sinn&fehfo&En er hann frá til öndvegissúlna sinna í Leiruvogi fyrir neðan heiði þá seldi hann lönd sín Úlfljóti lögmanni er þar kom út í \textbf{sín} \\
\hline
1352&sinn&feveo&Börkur býr nú ferð sína suður og ætlar nú að skipa til bús síns þar og fá þar til þess er þurfti að hafa en ætlaði þá að gera aðra ferð eftir fé sínu og \textbf{sína} \\
\hline
1353&sinn&feveo&Einn helgan dag fór Símon til fundar við frillu sína og sat á tali við hana en Jón gekk að honum og hjó hann \textbf{sína} \\
\hline
1354&sinn&feheþ&En er sendimaðurinn kom til Helgafells sat Snorri goði í öndugi sínu og var þar engi breytni á \textbf{sínu} \\
\hline
1355&sinn&feheþ&til Íslands og komu skipi sínu á Borðeyri í \textbf{sínu} \\
\hline
1356&sinn&feveþ&» Oddur sækir nú eftir en Óspakur fer undan og þar kemur að hann víkur til ráða Odds en Oddur heitir honum sinni ásjá og \textbf{sinni} \\
\hline
1357&sinn&feveþ&Hann hafði mjög í hirð sinni alls konar \textbf{sinni} \\
\hline
1358&sinn&fekfo&Um veturinn eftir jól fóru menn þeirra Sighvats og Sturlu vestur til Víðidals og var erindi að bændur skyldu járna hesta sína og vera búnir þann tíma er þeir væru upp \textbf{sína} \\
\hline
1359&sinn&feheo&Þeir ætluðu að búa skip sitt og lá leið þeirra um Fagrabrekku og voru þeir Hrómundur úti \textbf{sitt} \\
\hline
1360&sinn&fevfo&Hann venur komur sínar til Guðrúnar systur þeirra Starkaðar og \textbf{sínar} \\
\hline
1361&sinn&feveo&Hann varð léttbrýnn við féið og fékk til húskarla sína þá að flytja þá um nóttina í \textbf{sína} \\
\hline
1362&sinn&feveþ&Ketill kvað vera mega ef Símon fylgdi atgervi sinni « að hann mun muna dráp Össurar frænda \textbf{sinni} \\
\hline
1363&sinn&feheo&« Eg hefi athugað málin og sýnist mér sem eigi hæfi ágangur við þá sem goldið hafa of fjár og flýið land \textbf{sitt} \\
\hline
1364&sinn&fekfo&Ábóti hét að leggja til samnings með þeim en bað Ögmund eigi halda vini sína til rangra hluta með ofkappi því að þess er von að Sæmundur vilji það eigi \textbf{sína} \\
\hline
1365&sinn&feheo&Þá sigldu þeir norður til Beruvíkur og settu upp skip sitt og fóru upp í Hvítsborg í Skotlandi og voru með Melkólfi jarli þau \textbf{sitt} \\
\hline
1366&sinn&fekeo&Þá var blásið konungslúðri og stefnt saman öllum lendum mönnum og hirðmönnum en Sigurður og hans menn sáu þann sinn kost hinn fegursta að verða í \textbf{sinn} \\
\hline
1367&sinn&fevfo&Geirmundur skipar jarðir sínar á laun og ætlar út til Íslands um sumarið á skipi \textbf{sínar} \\
\hline
1368&sinn&feveþ&að svo sé nú að sinni og nýtur \textbf{sinni} að því\\
\hline
1369&sinn&nheo&Svo bjuggust þeir til göngu í hvert \textbf{sinn} \\
\hline
1370&sinn&feveo&Þenna vetur eftir jól bjóst Þorkell heiman norður til Hrútafjarðar að flytja norðan viðu \textbf{sína} \\
\hline
1371&sinn&fekeo&Þetta hið sama sumar er þeir feðgar færðu bústað sinn reið Þorbjörn til þings og hóf bónorð sitt og bað systur Gests \textbf{sinn} \\
\hline
1372&sinn&feheo&« Það er ort um það er hann bar út ösku með öðrum systkinum sínum og þótti þá til einkis annars fær fyrir vitsmuna sakir og varð þó um að sjá að ei væri eldur í því að hann þurfti allt vit sitt í þann \textbf{sitt} \\
\hline
1373&sinn&feheo&En um vorið eftir bjó hann skip sitt til \textbf{sitt} \\
\hline
1374&sinn&feveo&Tekur sína exi hvor þeirra í hönd sér og fara til \textbf{sína} við Snorra\\
\hline
1375&sinn&feheþ&« Svo mun mönnum nú sýnast að eg hafi vald og ríki til þess að neyða þig til trúar ef eg vil en það vil eg þó eigi gera fyrir því að það verður þægt mest í guðs augliti að mönnum sé eigi nauðgað til að gerast guðs þjónar og vill guð eigi nauðga þjónustu en verður feginn þeim hverjum er til hans vilja snúast að sjálfræði \textbf{sínu} \\
\hline
1376&sinn&feveo&Sem hann er á stað kominn gerði hann leið sína til Vatnsenda og er hann kom þar hitti hann Örn á hlaði úti og heilsar honum eigi öðruvís en svo að hann höggur til \textbf{sína} \\
\hline
1377&sinn&fekeo&En litlu eftir þetta lét Guðmundur taka hest sinn og reið ofan til \textbf{sinn} \\
\hline
1378&sinn&fekeþ&Hann leitaði griða föður sínum og sættar af konungi og bauð að setjast í gísling af hendi \textbf{sínum} \\
\hline
1379&sinn&fekeþ&segir og einkum Þórði bróður sínum að konungur hefir leyft honum að fara til \textbf{sínum} \\
\hline
1380&sinn&fekeo&hold og blóð Jesú krists í sinn \textbf{sinn} \\
\hline
1381&sinn&feveþ&Og litlu síðar sendi Glúmur mann til Mývatns og bauð hann Þórlaugu dóttur sinni til boðs og bað hana hafa gripi sína með sér og svo nokkuð \textbf{sinni} \\
\hline
1382&sinn&nheo&Eftir það heldur hann til Íslands og kemur skipi sínu í Svarfaðardalsárós og færir varnað sinn til \textbf{sinn} \\
\hline
1383&sinn&fekeþ&Er það til marks að hann hefir í pungi sínum líkneski Þórs af tönn gert og ert þú konungur of mjög dulinn að honum og færð hann eigi \textbf{sínum} \\
\hline
1384&sinn&fekeo&Þá gaf hann sonum sínum konunganöfn og setti það í lögum að hans ættmanna skyldi hver konungdóm taka eftir sinn föður en jarldóm sá er kvensifur væri af hans ætt \textbf{sinn} \\
\hline
1385&sinn&feheþ&Brátt varð Höskuldur vinsæll í búi sínu því að margar stoðar runnu \textbf{sínu} \\
\hline
1386&sinn&fekfþ&Vestfold og Agðir til Líðandisness og konungsnafn og lét hann þar hafa ríki með öllu slíku sem að fornu höfðu haft frændur hans og Haraldur hinn hárfagri gaf sonum \textbf{sínum} \\
\hline
1387&sinn&feveþ&Hann bað þeirrar konu er Þuríður hét systur Sumarliða og var hann fyrir svörum með systur sinni og \textbf{sinni} \\
\hline
1388&sinn&fehfo&Eysteinn konungur var inn í Ósló og lét draga skip sín meir en tvær vikur sjávar að ísi því að íslög voru mikil inn í \textbf{sín} \\
\hline
1389&sinn&fevfþ&» Svo kom Þórólfur fyrirtölum sínum að Skalla-Grímur skipaðist við og fékk menn til \textbf{sínum} um sumarið\\
\hline
1390&sinn&feveþ&Ólafur konungur braut landsfólk til kristni og réttra siða en refsaði grimmlega og jafndæmi og reistu her í móti honum og felldu hann á eigu sinni \textbf{sinni} \\
\hline
1391&sinn&fehfþ&munu þora að sitja að búum sínum í Fljótshlíð ef þeir eru utan sætta því að það verður \textbf{sínum} bani\\
\hline
1392&sinn&nheo&Og hvert sinn er Þóroddur kom á stöðul gekk Glæsir að honum og daunsnaði um hann og sleikti um klæði hans en Þóroddur klappaði um \textbf{sinn} \\
\hline
1393&sinn&feheþ&» Síðan keypti Þórólfur mjöl og malt og það annað er hann þurfti til framflutningar liði \textbf{sínu} \\
\hline
1394&sinn&feheþ&Eg vil bjóða fé fyrir mann þenna til þess að hann haldi lífi sínu og limum en þér konungur skapið og skerið um allt \textbf{sínu} \\
\hline
1395&sinn&fekfþ&Þar sáu þeir átján hross hjá selinu og ræddu um að þar mundu vera komnir þjófarnir og kváðu þá ráðlegast að leita eftir förunautum \textbf{sínum} \\
\hline
1396&sinn&feheo&Konungur lét gera honum bréf til vina sinna í Danmörk og setti fyrir sitt \textbf{sitt} \\
\hline
1397&sinn&feheþ&Eyvindur fór til Íslands og kom skipi sínu í Húsavík á Tjörnesi og nam Reykjadal upp frá Vestmannsvatni og bjó að Helgastöðum og þar var hann \textbf{sínu} \\
\hline
1398&sinn&fekfo&En er hann kom heim þá fannst honum það brátt í að Sigríður kona hans var heldur skapstór og taldi upp harma \textbf{sína} \\
\hline
1399&sinn&fekeo&En er við urðu varir Knúts konungs menn þá drógu þeir her saman og urðu brátt fjölmennir svo að synir Aðalráðs konungs höfðu ekki liðsafla við og sáu þann sinn kost helst að halda í brott og aftur vestur til \textbf{sinn} \\
\hline
1400&sinn&feveo&Þá færði Ásólfur byggð sína til Miðskála og var \textbf{sína} \\
\hline
1401&sinn&feheþ&En er á leið haustið hélt hann liði sínu til Noregs og dvaldist í Elfinni nokkura \textbf{sínu} \\
\hline
1402&sinn&fekeþ&spurðust tíðinda og ræddu um hvorir betur mundu veitt hafa um jólin og fylgdi hvor sínum \textbf{sínum} \\
\hline
1403&sinn&fekfo&Föstudag hinn langa sendi Guðrún menn sína að forvitnast um ferðir þeirra \textbf{sína} \\
\hline
1404&sinn&fekeo&Hún mundi Snorra föður sinn en hann var þá nær hálffertugur er kristni kom á Ísland en andaðist einum vetri eftir fall Ólafs konungs hins \textbf{sinn} \\
\hline
1405&sinn&feveþ&taka nú hesta sína og fara nú á bak og sjá eigi færi sitt á Vémundi að sinni og þótti Steingrími það ráðlegt að þeir sneru heimleiðis en þeir Gnúpur og Steinn svöruðu að aldrei vildu þeir við þetta heim snúa og þóttust vita hæðileg orð Vémundar við þá ef þeir snúa aftur við þetta og kváðu nú vænna að snúa til Mývatns að drepa Herjólf síðan þeir náðu eigi \textbf{sinni} \\
\hline
1406&sinn&feveo&En er Tumi hafði þar skamma stund verið þá hófu þeir ferð sína suður til Dala og voru tólf \textbf{sína} \\
\hline
1407&sinn&feheo&En Galti tók það ráð sem Vémundur lagði til með honum og var nú kominn á leið með lausafé sitt og ætlaði að flytja til \textbf{sitt} \\
\hline
1408&sinn&fekeþ&að þér dæmið rétta dóma því að til hvers dóms og þings kemur almáttigur guð með sínum helgum mönnum og vitjar góðra manna og góðra \textbf{sínum} \\
\hline
1409&sinn&nheo&Svo er sagt eitthvert sinn að Þórður frétti að Hrafn var riðinn út í Einarshöfn til skips og var einn í reið og ætlaði heim um \textbf{sinn} \\
\hline
1410&sinn&feveo&Hann ætlaði þá ferð sína til umbótar og \textbf{sína} \\
\hline
1411&sinn&fekeþ&En er hann spurði að mikill hluti liðs var frá þeim farinn þá hélt hann sínum her aftur til Sjálands og lagðist í Eyrarsund með allan \textbf{sínum} \\
\hline
1412&sinn&fekeo&Þeir gáfu Þorleifi sinn hlut skips og skildu þeir góðir vinir \textbf{sinn} \\
\hline
1413&sinn&fehfo&En er hann hafði þar á huginn þá minntist hann þess að hina fyrstu tíu vetur konungdóms hans voru honum allir hlutir hagfelldir og farsællegir en síðan voru honum öll ráð sín þunghrærð og torsótt en gagnstaðlegar allar \textbf{sín} \\
\hline
1414&sinn&fekeo&» Akra-Þórir hitti nú nafna sinn og spurði hvort þar mundi staðar nema er nú var komið að hann mundi eigi á líta með \textbf{sinn} \\
\hline
1415&sinn&fekeo&Austur þangað flýði og fjöldi manna úr Þrándheimi fyrir þeim ófriði því að Eysteinn konungur skattgildi Þrændi og setti þar til konungs hund sinn er Saur \textbf{sinn} \\
\hline
1416&sinn&fekeo&Bar Lágálfur hann til bæjar og fór síðan veg sinn og er hann gekk fram eftir Blönduhlíð kom hann á Frostustaði og sunnan undir húsin og að vindglugginum og sá inn í \textbf{sinn} \\
\hline
1417&sinn&feheo&« Vera má að þú hugsir slíkt en nakkvað ætla eg að vér mundum nú nauðga þér meir til sagna ef Þorkell væri eigi hér með sitt fjölmenni því að svo segir mér hugur að þú vitir nakkvað til hvar Þormóður \textbf{sitt} \\
\hline
1418&sinn&feheo&Bjó þá Ögmundur skip sitt og fór út til Íslands um sumarið og hafði aflað mikils fjár í ferð \textbf{sitt} \\
\hline
1419&sinn&fekeo&Hún veik á við Önund að hún vildi kvæna Ólaf frænda sinn og vildi að hann bæði Álfdísar hinnar \textbf{sinn} \\
\hline
1420&sinn&feheo&Þau fóru öll til Íslands og brutu skip sitt við Tjörnes og voru að Auðólfsstöðum hinn fyrsta \textbf{sitt} \\
\hline
1421&sinn&feheo&Býr Þorgeir skip sitt og heldur því til \textbf{sitt} \\
\hline
1422&sinn&feheo&Þeir Sighvatur gáfu honum upp bú sitt og ríki og fór þeim það betur en getið var til \textbf{sitt} \\
\hline
1423&sinn&fekeo&Þorgils lét safna liði um Blönduhlíð en hann reið við tíunda mann út til Hofs að hitta Brodda mág \textbf{sinn} \\
\hline
1424&sinn&fekeþ&Ásmundur hét sonur hans og var hann um alla hluti líkur föður sínum eða nökkvi \textbf{sínum} \\
\hline
1425&sinn&fehfo&Björg og synir hennar fóru á fund Þorgils og biðja hann líta á sín \textbf{sín} \\
\hline
1426&sinn&feveo&Hann Auðun lagði mestan hluta fjár þess er var fyrir móður sína áður hann stigi á skip og var kveðið á þriggja \textbf{sína} björg\\
\hline
1427&sinn&fekfo&Óspakur eggjaði sína menn til varnar og barðist sjálfur \textbf{sína} \\
\hline
1428&sinn&fekfo&En er sendimenn konungs komu til Kveld-Úlfs og sögðu honum sín erindi og það að konungur vill að Kveld-Úlfur komi til hans með alla húskarla \textbf{sína} \\
\hline
1429&sinn&fekeo&Þeir segja honum allt tal þeirra mæðgina og það með að þeir mega eigi bera lengur harm sinn og frýju móður \textbf{sinn} \\
\hline
1430&sinn&fekfþ&Það bar við um dag einn að Ögmundi bar fyrir augu menn þá er gengu í kyrtlum þeim er gervir voru af klæði Ingimundar prests og sagði hann frá vinum sínum Bárði sálu og Pétri glyfsu og Indriða og mælti við \textbf{sínum} \\
\hline
1431&sinn&feveþ&En er hann grunaði að þeir héldu fleirum klæðum hans þá greip hann sinni hendi hvorn þeirra og steyptist utanborðs með alla þá en skútan renndi fram á langt og varð þeim seint að víkja og löng dvöl áður en þeir fengju sína menn \textbf{sinni} \\
\hline
1432&sinn&nheo&Og eitthvert sinn kom hann við Danmörk og fór til Hleiðrar þangað sem haugur Hrólfs konungs kraka var og braut hauginn og tók á braut sverðið Hrólfs \textbf{sinn} \\
\hline
1433&sinn&feveo&Síðan fór hann aftur til Miklagarðs með her þenna og dvaldist þar litla hríð áður hann byrjaði ferð sína út í \textbf{sína} \\
\hline
1434&sinn&feheo&Og er menn voru sofnaðir um nóttina þá vakir Þorgils og íhugar sitt mál og vildi eigi oftar úr leikinum ger \textbf{sitt} \\
\hline
1435&sinn&fekeo&Fer Úlfar þá til Arnkels og segir honum skaða sinn og bað hann \textbf{sinn} \\
\hline
1436&sinn&feheþ&Fannst Þórði það í svörum Ásbjarnar að hann tók á öllu hræddur en kallaði það þó næst sínu skapi að þjóna \textbf{sínu} \\
\hline
1437&sinn&fekfo&Þórir bað sína menn hlífa sér og gæta síns sem \textbf{sína} \\
\hline
1438&sinn&feken&Sigurður sendi snemma um vorið Arnór Tumason stjúpson sinn á fund þeirra Sighvats og Kolbeins Tumasonar og bað þá koma til sín með allan afla þann er þeir fengju því að hann þóttist vita um \textbf{sinn} \\
\hline
1439&sinn&feheo&» Sámur kvaðst ekki nenna starfa þeim að flytja allt sitt úr Austfjörðum og þótti sér eigi veitt í ef eigi væri þetta og sagðist vildu heim búast og bað skipta hestum við sig og var það þegar til \textbf{sitt} \\
\hline
1440&sinn&feveo&Kjartan segir hvað þeir hefðu helst ætlað en kvað þó það sitt erindi til konungs að biðja sér orlofs um sína \textbf{sína} \\
\hline
1441&sinn&feveo&Fór hann þá á brott með Gunnhildi konu sína og börn \textbf{sína} \\
\hline
1442&sinn&feven&Þótti Magnúsi sín eign gefin engum mun síður og ýfðist hugur hans mjög við það og þóttist mishaldinn af við frænda \textbf{sín} \\
\hline
1443&sinn&fekeo&Hann reisti hof mikið hundrað fóta langt og er hann gróf fyrir öndvegissúlum þá fann hann hlut sinn sem honum var fyrir \textbf{sinn} \\
\hline
1444&sinn&fekfo&Hljópu þeir á hesta sína og hleyptu út á Skaftá sem mest máttu þeir og urðu svo hræddir að þeir komu hvergi til bæja og hvergi þorðu þeir að segja \textbf{sína} \\
\hline
1445&sinn&feveo&Haraldur konungur og Gunnhildur leiddu Ólaf til skips og sögðust mundu leggja til með honum hamingju sína með vingan þeirri annarri er þau höfðu til \textbf{sína} \\
\hline
1446&sinn&fekeo&En er því var lokið þá taka þeir bræður tal um það að þeir muni efna til erfis eftir föður sinn því að það var þá tíska í það \textbf{sinn} \\
\hline
1447&sinn&feveþ&Þá var hann á fertugs aldri er hér var komið sögunni og var ei auðsóttur með afla Guðmundar en framkvæmd \textbf{sinni} \\
\hline
1448&sinn&fevfþ&Riðu þeir þá til móts við Þórð bróður sinn austur til Þjórsár og segja honum frá ferðum sínum sem farið \textbf{sínum} \\
\hline
1449&sinn&feveþ&Það þykir mönnum sem Gunnhildur hafi bannað Geir með fjölkynngi sinni til \textbf{sinni} \\
\hline
1450&sinn&fehfo&» Flosi sté undir borð og allir menn hans en lögðu vopn sín upp að \textbf{sín} \\
\hline
1451&sinn&feveþ&Þorsteinn hafði beðið ámæli af konu sinni Þuríði að hann legði hug á Gró fyrir sakir fjölkynngi \textbf{sinni} \\
\hline
1452&sinn&feheþ&» Höskuldur kvaðst þess eigi þræta mundu og segir henni hið sanna og bað þá þessi konu virkta og kvað það nær sínu skapi að hún væri heima þar að \textbf{sínu} \\
\hline
1453&sinn&fekeþ&Hrútur Herjólfsson gaf frelsi þræli sínum þeim er Hrólfur hét og þar með fjárhlut nokkurn og bústað að landamæri þeirra Höskulds og lágu svo nær landamerkin að þeim Hrýtlingum hafði yfir skotist um þetta og höfðu þeir settan lausingjann í land \textbf{sínum} \\
\hline
1454&sinn&nheo&Og annað sinn er þeir tókust til þrífur Ásbjörn til Þórðar svo að hann féll á \textbf{sinn} \\
\hline
1455&sinn&feheþ&En jafnan er þeir skiptu liði sínu þá fylgdi konungi Norðmanna lið en Dagur fór þá í annan stað með sitt lið en Svíar í þriðja stað með sínu \textbf{sínu} \\
\hline
1456&sinn&feveo&Og í því hljóp út Svartur nautamaður og hafði stálhúfu á höfði mikla og ákaflega forna og skjöld fyrir sér en ekki hafði hann höggvopn annað en hann reiddi mykireku sína um \textbf{sína} \\
\hline
1457&sinn&fekeþ&Það segja flestir menn að Þorvarður Spak-Böðvarsson hafi skírður verið af Friðreki biskupi en Gunnlaugur munkur getur þess að sumir menn ætla hann skírðan verið hafa á Englandi og hafa þaðan flutt við til kirkju þeirrar er hann lét gera á bæ \textbf{sínum} \\
\hline
1458&sinn&feheþ&Reið Hafliði heim norður en Þorgils situr í búi sínu með átta tigi vígra \textbf{sínu} \\
\hline
1459&sinn&feveo&Og er þeir koma til Limafjarðar og lágu að Hálsi þá átti Arinbjörn húsþing við lið sitt og sagði mönnum fyrirætlan \textbf{sína} \\
\hline
1460&sinn&nheo&» Nú líður veturinn og fór Geitir um vorið til Hofs að heimta peninga Höllu í annað sinn en Helgi vildi eigi út \textbf{sinn} \\
\hline
1461&sinn&feveo&Hann seldi um vorið jörð sína og kvikfé og fór til skips með Guðríði með allt \textbf{sína} \\
\hline
1462&sinn&feveo&þeir höfðu farið vestur um haf eftir Eysteini og fylgdu honum í land og héldu þegar norður til Þrándheims um gagndaga svo að hann skyldi hafa þriðjung Noregs við bræður \textbf{sína} \\
\hline
1463&sinn&feveo&« Hér er hringur er faðir minn sendi þér og með kveðju sína og bað þess að þú veittir okkur veturvist í vetur eða lengur þó að við \textbf{sína} \\
\hline
1464&sinn&feheo&Þeir gera ráð sitt þó að heldur sé óráð og ætla honum að ríða á millum kirkjugarðs og fannar þeirrar er þar hafði lagt sem leiðin lá í \textbf{sitt} \\
\hline
1465&sinn&fehfo&og vildu heldur fyrirláta óðul sín og frændur og vini en liggja undir þrælkan og ánauðaroki konungs og leituðu mjög til ýmissa \textbf{sín} \\
\hline
1466&sinn&fekfþ&En Sturla reið heim á bæinn og skipaði sínum mönnum til \textbf{sínum} \\
\hline
1467&sinn&fekeo&Og er Þorsteinn kom þar til lands þá veitti hann tilkall um föðurarf sinn við ármennina er sest höfðu í bú \textbf{sinn} \\
\hline
1468&sinn&feheþ&Leituðu þá menn Arnviðar konungs til flótta en sjálfur hann féll á skipi \textbf{sínu} \\
\hline
1469&sinn&nheo&Síðan færði Þjóstólfur upp öxina í annað sinn og hjó í höfuð Þorvaldi og hafði hann þegar \textbf{sinn} \\
\hline
1470&sinn&fekfo&En þegar er lið Þórðar sá af öðrum skipum að hann lagði frá þá hét hver skipstjórnarmaður á sína liðsmenn að leysa sig úr flotanum og flýr nú hvert skipið sem skjótast verður \textbf{sína} \\
\hline
1471&sinn&fehfo&Hún gekk ofan fyrir brekkuna til lækjar þess er þar féll og tók að þvo léreft \textbf{sín} \\
\hline
1472&sinn&fekfo&En þeir hinir vermsku er undan komust tóku hesta sína og drógust austur af skóginum til \textbf{sína} \\
\hline
1473&sinn&fekeo&« Þar kann eg þó eigi af að taka nema það sé til nokkurs gert og vilji Freyr þar láta sinn hlut niður koma er hann vill sitt sæmdarsæti \textbf{sinn} \\
\hline
1474&sinn&fekeo&Hall son sinn níu vetra gamlan hafði hann með \textbf{sinn} \\
\hline
1475&sinn&fekfo&Þessir menn riðu vestur til Ísafjarðar á Nauteyri og tóku þar skip en létu eftir hesta sína og \textbf{sína} \\
\hline
1476&sinn&feheo&lagðist síðan niður í rúm sitt með miklum harmi \textbf{sitt} \\
\hline
1477&sinn&fekeo&Hallfreður fór þegar að finna föður sinn er hann spurði vanmátt \textbf{sinn} \\
\hline
1478&sinn&fekeo&Þar er nú til að taka að Freydís Eiríksdóttir gerði ferð sína heiman úr Görðum og fór til fundar við þá bræður Helga og Finnboga og beiddi þá að þeir færu til Vínlands með farkost sinn og hafa helming gæða \textbf{sinn} við hana\\
\hline
1479&sinn&fehfo&Báru menn þá grjót á skip sín og bjuggust til \textbf{sín} \\
\hline
1480&sinn&fekeþ&Þá gaf Haraldur konungur yfirsókn ríkis þess alls Guttormi syni sínum og setti hann þar höfðingja \textbf{sínum} \\
\hline
1481&sinn&feheo&» Henni kvaðst harðla gott þykja hann að finna en kvað það hugboð sitt að hún sæi eigi síður fyrir hans kosti þó að þau sætu eigi síður heima þar en rækjust annars \textbf{sitt} \\
\hline
1482&sinn&fekeþ&Sögðu þeir af fundi þeirra bræðra og öllum sínum \textbf{sínum} \\
\hline
1483&sinn&feheo&Hann seldi um vorið jörð sína og kvikfé og fór til skips með Guðríði með allt \textbf{sitt} \\
\hline
1484&sinn&feveo&Það er sagt að Vémundur kom að máli við konu sína einnhvern dag og segir að hann mundi fara á brott erinda sinna og kvaðst ekki heim mundu koma í \textbf{sína} \\
\hline
1485&sinn&fekfþ&Bauð hann sínum verkmönnum öllum þegar er dagaði fara til skyndilega og brjóta ofan hofið en færa heim viðinn til \textbf{sínum} \\
\hline
1486&sinn&feveþ&Skal eigi pynda yður til kristni að sinni því að guð mælir svo að hann vill að engi komi nauðigur til \textbf{sinni} \\
\hline
1487&sinn&feheo&Jökuls þáttur Búasonar Jökli þótti nú illt verk \textbf{sitt} \\
\hline
1488&sinn&fekfþ&kvíddi mjög ánauð og hræddist bæði sult og píslir og vænti engrar afturlausnar af frændum \textbf{sínum} \\
\hline
1489&sinn&feveþ&En eg vil eigi ójafnað bjóða Gissuri svo að eigi ráði hann fyrir dóttur sinni sem hann vill því að Gissur er góðs verður frá \textbf{sinni} \\
\hline
1490&sinn&feveþ&Síðan fóru þeir eigi allfáir suður yfir heiði og yfir Breiðafjörð og komu um nátt til Helgafells og gengu þegar inn í skálann og var Hallgerður tekin upp úr hvílu sinni og borin út en Ólafi var haldið og Runólfi \textbf{sinni} \\
\hline
1491&sinn&nheþ&Sveinninn Hörður stóð við stokk og gekk nú hið fyrsta sinni frá stokkinum og til móður sinnar og rasaði að knjám \textbf{sinni} \\
\hline
1492&sinn&feveþ&« Vér munum nú og eigi ófrið bjóða að sinni þótt til þess sé ærin sök og viljum vér reyna enn framar áður en vér ráðum frá með \textbf{sinni} \\
\hline
1493&sinn&fekeþ&Er og svo sagt að þá er kristni kom til Íslands að Þórhallur lét kirkju gera á bæ sínum af þeim viði er Hávarður flutti út \textbf{sínum} \\
\hline
1494&sinn&fevfo&Nú er frá þeim Katli að segja að þeir riðu sem mest máttu þeir til þess er þeir komu heim til Svínafells og sögðu sínar farar eigi \textbf{sínar} \\
\hline
1495&sinn&nheþ&Hann kemur einhverju sinni á fund Þorbergs frænda síns og skorar á hann til nokkurra tillaga við \textbf{sinni} \\
\hline
1496&sinn&fekfþ&og svo Völluna út til Eyvindarár og tók mikið af landnámi Una Garðarssonar og byggði þar frændum sínum og \textbf{sínum} \\
\hline
1497&sinn&fehfþ&« Eigi mun við ljúgast að hann Barði er kominn » og vildu þrífa til vopna sinna og engi þeirra fékk náð sínum \textbf{sínum} \\
\hline
1498&sinn&feheo&hlupu þar í byggðir er þeim þótti sitt færi vera fyrir \textbf{sitt} sakir\\
\hline
1499&sinn&feheþ&Það sama sumar kom Bjarni skipi sínu á Eyrar er faðir hans hafði brott siglt um \textbf{sínu} \\
\hline
1500&sinn&feveo&Og móður sína hitti hann brátt og fagnaði hún honum vel og sagði ójafnað þeirra feðga og bað hann þó hafa við þolinmæði en kvaðst til lítils um fær að ganga þeim í \textbf{sína} \\
\hline
1501&sinn&fekeo&Settist þá í sinn stein hvort þeirra og lifðu svo langan tíma sem guð vildi skipa og entu svo sína \textbf{sinn} \\
\hline
1502&sinn&feheo&En þessi mál lukust svo að um haustið Máritíusmessu sættust þeir Sigurður við biskup og lögðu sitt mál allt á \textbf{sitt} dóm\\
\hline
1503&sinn&fekeo&og ætluðu til hefnda eftir bróður \textbf{sinn} \\
\hline
1504&sinn&feveþ&Og er hún kom á Bólstað segir hún Arnkatli að frændur Vigfúss vildu að hann gerðist fyrirmaður að eftirmáli um víg Vigfúss en þeir hétu allir sinni \textbf{sinni} \\
\hline
1505&sinn&nheo&« Þá hefir þú enn eigi betur en hálfgert mitt erindi og muntu nú vilja fara í annað sinn og færa mér augað Þorleifs það er eftir \textbf{sinn} \\
\hline
1506&sinn&nheo&En í annað sinn riðu þeir til \textbf{sinn} \\
\hline
1507&sinn&fekfo&Þetta haust drap Grímur hersir Öndótt kráku fyrir það er hann náði eigi fénu til handa konungi en Signý kona Öndótts bar á skip allt lausafé þeirra þegar hina sömu nótt og fór með sonu \textbf{sína} \\
\hline
1508&sinn&fekfo&Þá tók jarl til máls og segir frá því að Ólafur Noregskonungur hafði senda menn sína austur þannug til \textbf{sína} \\
\hline
1509&sinn&feheo&harðlífi sitt og óyndi af andláti \textbf{sitt} \\
\hline
1510&sinn&fehfo&Þorgils þakkar honum sín ummæli en kvaðst fúsastur að fara til \textbf{sín} \\
\hline
1511&sinn&feheo&Síðan koma bændur á fund Ólafs og verða hvorir öðrum fegnir og taka þegar samlag \textbf{sitt} \\
\hline
1512&sinn&fekfo&Þorgils reið með heimamenn sína til Rauðamels og leitaði eftir við Halldór og húsfreyju hans að þau réðust til Staðar og væru þar fyrir \textbf{sína} \\
\hline
1513&sinn&fekeþ&Þau heit virtust Einari mikils og hét þar í mót trúnaði \textbf{sínum} \\
\hline
1514&sinn&nheo&Már talar til öðru sinni og hið þriðja sinn og stoðar þó \textbf{sinn} \\
\hline
1515&sinn&feveo&Fari Þrándur annað sumar með ull sína ef hann vill selja láta en ef eg kemst í brott þá þykir mér þess von að eg komi aldrei síðan til \textbf{sína} \\
\hline
1516&sinn&fehfo&En sakir þess að henni var Björn kunnigur áður og þau höfðu elskast sín á millum mjög kærlega þá játaði \textbf{sín} \\
\hline
1517&sinn&fekeþ&Ásdís húsfreyja sendir eftir mönnum og var búið um lík Atla og var hann jarðaður hjá föður \textbf{sínum} \\
\hline
1518&sinn&feheþ&En Skagfirðingar voru á Víðimýri um nóttina en fóru þaðan snemma fimmtadaginn norður yfir Jökulsá og námu stað fyrir sunnan Djúpadal á skriðunni og fylktu þar sínu liði og höfðu framarlega á sétta hundraði \textbf{sínu} \\
\hline
1519&sinn&feveo&Og er þeir komu til strandar skaut konungur á húsþingi og sagði fyrirætlan \textbf{sína} \\
\hline
1520&sinn&fekfo&Björn réð það nú af að hann fer nú til Reykjadals að finna frændur \textbf{sína} \\
\hline
1521&sinn&fehfþ&Síðan kallaði Björn til tals við sig sendimenn og spurði þá eftir erindum \textbf{sínum} \\
\hline
1522&sinn&feheþ&Þá fór Eyjólfur á Helgastaði og máttu þeir eigi sættast því að hvorirtveggju kölluðust allt eiga það er þeir deildu um og urðu engi miðlunarmál með þeim því að hvorigir vildu láta né eitt af sínu máli og varð það síðan að stefnuför og stefndi Eyjólfur Önundi um afneyslu fjárins og kallaði síns neytt \textbf{sínu} \\
\hline
1523&sinn&fekeo&Svo er sagt frá Hávarði að hann selur eignir sínar en þau ráðast norður til Svarfaðardals og upp í dal þann er Oxadalur heitir og reisir þar bústað sinn og bjuggu þar nokkura vetur og kallaði Hávarður þenna bæ á \textbf{sinn} \\
\hline
1524&sinn&fekfo&En er hann spyr þetta heitir hann á menn sína að þeir skulu hefna Áka \textbf{sína} \\
\hline
1525&sinn&feveþ&Refur sótti þegar á konungs fund og sagði honum allan vöxt á þarkomu sinni og bað hann \textbf{sinni} \\
\hline
1526&sinn&nheþ&« Viltu nú fara til Þorleifs öðru \textbf{sinni} \\
\hline
1527&sinn&fehfþ&Á alþingi var fjölmennt og bar Öngull fram mál sín og hrósaði mjög verkum sínum að hann hafði drepið þann skógarmann er gildastur hefði verið á landinu og þóttist eiga fé það sem lagið var til \textbf{sínum} honum\\
\hline
1528&sinn&fevfo&En er Þórður sá hversu alþýðu var snúið af hræðslugeði til mótgangs við hann fyrir ríki Kolbeins svo að hverjum þótti sem láta sjálfan sig eða eignir sínar ef honum gerði nokkurn góðan hlut þá fékk hann sér hesta og reið inn til \textbf{sínar} \\
\hline
1529&sinn&fekfþ&Skýldi svo allsvaldanda guðs miskunn sínum mönnum að því síður fengu þeir biskup þessu sinni nokkuð mein af illvilja og umsát heiðingja að þeir urðu með engu móti varir við þessa aðför og \textbf{sínum} \\
\hline
1530&skip&nheþ&Þorgils hafði verið í Hafursfirði í liði Haralds konungs og stýrði þá skipi því er Þórólfur átti og hann hafði haft í \textbf{skipi} \\
\hline
1531&skip&nheng&« Mjög undrast konungsson skipið og bjóð þú honum að þiggja að þér því að eg veit að okkur verður það að liðsemd mikilli við konung ef Eiríkur er flutningsmaður \textbf{skipið} \\
\hline
1532&skip&nheþ&Konungur spurði hví hann kæmi á hans vald þar sem hann var áður á öðru \textbf{skipi} \\
\hline
1533&skip&nhfþ&En er Sigurður jarl spurði þetta þá fór hann með skipum öllum þeim er hann fékk hið ytra norður til móts við Hákon \textbf{skipum} \\
\hline
1534&skip&nheng&Og sáu varðmenn Knúts konungsskipið og ræddu um sín í milli hvað skipa það mundi vera og gátu þess að vera mundi flutt salt eða síld er þeir sáu fá mennina en lítinn róðurinn en skipið sýndist þeim grátt og bráðlaust og sem skipið mundi skinið af sólu og sáu þeir að skipið var sett \textbf{skipið} \\
\hline
1535&skip&nheo&» Síðan koma þau til eyjarinnar og bar hann út á skip fé það er þar var eftir \textbf{skip} \\
\hline
1536&skip&nhen&Litlu fyrir vetur kom skip til \textbf{skip} sunnan úr Orkneyjum\\
\hline
1537&skip&nhen&Og er hann var albúinn og skip hans lá til byrjar fyrir bryggjunum þá kom þar að honum Suðurmaður \textbf{skip} \\
\hline
1538&skip&nhen&Á einu sumri herjuðu þeir í Danmörk og þá hittu þeir Guðlaug Háleygjakonung og áttu við hann orustu og lauk svo að skip Guðlaugs var hroðið en hann varð \textbf{skip} \\
\hline
1539&skip&nhfn&Fær hann nú til tvö skip og liðfæra menn svo að birgt var og lætur flytja líkin norður til Skaga og sára \textbf{skip} \\
\hline
1540&skip&nheo&Skjálf og hennar menn hljópu á skip og reru í \textbf{skip} \\
\hline
1541&skip&nheþ&Hinu þriðja skipi réð sá maður er Hermundur hét og var Koðránsson og Þorgils bróðir hans og höfðu mikla sveit \textbf{skipi} \\
\hline
1542&skip&nheþg&Kerling hafði ráð fyrir liði þeirra og hún hafði huliðshjálm yfir skipinu meðan þau reru yfir fjörðinn til \textbf{skipinu} \\
\hline
1543&skip&nheþ&Voru þeir Gunnsteinn þá langt komnir er skriður var að skipi \textbf{skipi} \\
\hline
1544&skip&nheþ&Þá lagði Gregoríus að skipi Ívars og áttust þeir þá við langa \textbf{skipi} \\
\hline
1545&skip&nhfn&Þau misseri týndist skip Ásmundar kastanrassa og fóru þar margir íslenskir \textbf{skip} \\
\hline
1546&skip&nhfn&Síðan lögðu hvorir í mót öðrum og laust saman með þeim snarpri sókn og var þar hinn harðasti bardagi og réðst brátt mikið mannfall í hvortveggja liði en þó hafði eigi lengi staðið bardaginn áður en mannfallið hneig í lið konungs og hruðust hans skip \textbf{skip} \\
\hline
1547&skip&nheo&Um daginn gerði á hvasst veður og hljópu þeir Kjartan þá út að festa skip sitt og er þeir höfðu því lokið ganga þeir heim til \textbf{skip} \\
\hline
1548&skip&nheo&Ná þeir brátt uppgöngu á skip Þorgeirs og láta skammt stórra höggva í \textbf{skip} \\
\hline
1549&skip&nheo&Höfðu þeir jafnan skip fyrir landi og tóku af hvers manns eigu eða rekum það er þeim \textbf{skip} \\
\hline
1550&skip&nhfng&Taka nú skipin að leggja sem næst hvert öðru til \textbf{skipin} \\
\hline
1551&skip&nheþ&En er Ari heyrði þessa umræðu þá lét hann þegar bera föt sín af skipi og ræðst þá enn til hirðvistar með konungi og jarli en Ingimundur prestur og aðrir íslenskir menn héldu til Íslands og urðu vel \textbf{skipi} \\
\hline
1552&skip&nheo&Önundi varð komið á skip til þess manns er Þrándur \textbf{skip} \\
\hline
1553&skip&nhfn&Nú renna skip að landi og ganga menn upp til \textbf{skip} \\
\hline
1554&skip&nhfþg&Hallbjörn kveðst ætla til strandar niður er skipin sigla að landi og lést hann mundu segja þeim grein á \textbf{skipunum} \\
\hline
1555&skip&nhfn&Önundur bað þá leggja skip sín milli hamra \textbf{skip} \\
\hline
1556&skip&nheo&Annan dag fór Gautur og hans förunautar að fá eldibranda en Þorgeir var heima og bjó \textbf{skip} \\
\hline
1557&skip&nheo&Gekk þá Þórður aftur á skip sitt og svo gerðu nú margir \textbf{skip} menn\\
\hline
1558&skip&nheþg&Austmennirnir fara frá skipinu eftir þetta og fá sér hesta og ætla til vistar \textbf{skipinu} í Skutilsfjörð\\
\hline
1559&skip&nhfþ&En er þeir spurðu að Ólafur konungur fór með her manns sunnan um Hálogaland þá safna þeir her að sér og bjóða skipum út og fá lið \textbf{skipum} \\
\hline
1560&skip&nheþg&Síðan gengur hún af skipinu og til förunauta \textbf{skipinu} \\
\hline
1561&skip&nhen&Skip hafði staðið uppi í Hrútafirði um veturinn og hafði Þorgils keypta marga viðu til skálagerðar og heim flutta nema eitt hundrað viðar hafði eftir orðið og það eitt fékk Hafliði af sektarfjám \textbf{Skip} \\
\hline
1562&skip&nhfþ&Og þar kom að konungur mælti að stöðva skyldi liðið og hverfa aftur « því að eg sé bragð þeirra að þeir munu þegar við nema er þeir fá sér afla til en flýja undan þar til og teygja oss svo frá skipum \textbf{skipum} \\
\hline
1563&skip&nheþ&Var og á skipi sá maður er Svalur hét og Þúfa kona \textbf{skipi} \\
\hline
1564&skip&nheþ&Fór Þórður þá á skipi inn til \textbf{skipi} \\
\hline
1565&skip&nheþ&Menn fóru af skipi konungs og á kaupskipið og spurðu tíðinda eða hvað íslenskra manna þar væru á \textbf{skipi} \\
\hline
1566&skip&nheog&Hann sér skipið Slóðann og í fimm hundruð \textbf{skipið} \\
\hline
1567&skip&nheo&Og þegar er sár Geirs voru bundin sté Hörður á skip með tólfta mann og fór þegar inn til Brynjudals og kveðst enn vilja reyna \textbf{skip} \\
\hline
1568&skip&nheng&En er þeir voru skammt frá landi komnir fyllti skipið undir þeim og komust við nauð til sama \textbf{skipið} \\
\hline
1569&skip&nheog&Og er skipið var algert þá bjó hún skipið og hafði auð \textbf{skipið} \\
\hline
1570&skip&nheþ&Skildu þeir Þorkell með vináttu en Björn fór vestur til Englands og var fyrir skipi Þorkels því er þangað \textbf{skipi} \\
\hline
1571&skip&nheþ&En hálfum mánuði síðar kom Ormar skipi sínu í Reyðarfjörð og bauð Ketill honum heim en skip hans var upp \textbf{skipi} \\
\hline
1572&skip&nheþ&Þeir voru í stafni á skipi hans en þá er Egill tók skipstjórn þá var Þorfinnur \textbf{skipi} stafnbúi\\
\hline
1573&skip&nheng&En er Þórður ætlaði aftur á sitt skip þá sá hann að skipið var autt af hans mönnum en hann hafði þá eigi liðskost til að sækja \textbf{skipið} \\
\hline
1574&skip&nhfþ&Þá var blásið í bænum og farið að leita mannsins bæði á skipum og \textbf{skipum} \\
\hline
1575&skip&nheþ&og ekki starfa það sem hann átti að skipi að gera til jafnaðar við aðra \textbf{skipi} \\
\hline
1576&skip&nheog&Og er þeir höfðu siglt um stund gekk veður til landsuðurs og austurs og gerði storm mikinn og bar þá norður um Írland og brutu þar skipið í spón við ey eina \textbf{skipið} \\
\hline
1577&skip&nheþ&Þeir tóku Björn og bundu hann á skipi og fóru svo til \textbf{skipi} með Björn\\
\hline
1578&skip&nhfn&Síðan fara þeir og hafa skip úr Haukadal og róa til Lækjaróss og ganga þar á land og til bónda þess er þar bjó á \textbf{skip} \\
\hline
1579&skip&nheo&Þráinn Sigfússon bjó þá skip sitt til Íslands og var nú mjög \textbf{skip} \\
\hline
1580&skip&nheþg&Hélt hann til Borgarfjarðar og kom skipinu skammt frá bæ \textbf{skipinu} \\
\hline
1581&skip&nheog&Og þá er þeir voru mjög búnir reri að þeim maður á báti og festi bátinn við skipið en gekk síðan upp á skipið til \textbf{skipið} við Kolbein\\
\hline
1582&skip&nheþ&Þá spurði Skalla-Grímur hvað fleira væri þeirra manna á skipi er virðingarmenn \textbf{skipi} \\
\hline
1583&skip&nheþg&» Síðan réðu þeir skipinu til hlunns og bjuggu um en færðu varning allan vestur í \textbf{skipinu} \\
\hline
1584&skip&nheo&Síðan gengum vér þann stíg til sjóvar sem hann vísaði oss og fundum þar Norðmanna skip og fór allt eftir því sem hann hafði oss fyrir sagt um vora \textbf{skip} \\
\hline
1585&skip&nheo&En um vorið mælti Þorvaldur að þeir skyldu búa skip sitt og skyldi eftirbátur skipsins og nokkurir menn með fara fyrir vestan landið og kanna þar um \textbf{skip} \\
\hline
1586&skip&nheþ&En er kaupmenn drifu frá skipi hver til síns heima þá mælti Einar til \textbf{skipi} \\
\hline
1587&skip&nheo&Önundur lagði skip sitt á annað borð skipi Þóris \textbf{skip} \\
\hline
1588&skip&nhfo&Voru þeir með Véþormi um veturinn en um vorið bjuggu þeir bræður skip sín til Íslands og ætluðu þeir Ketill að halda \textbf{skip} \\
\hline
1589&skip&nheþ&Helgi bjólan kom skipi sínu fyrir sunnan land og nam Kjalarnes allt á milli Kollafjarðar og Hvalfjarðar og bjó að Esjubergi til \textbf{skipi} \\
\hline
1590&skip&nheo&Síðan bjuggu þeir skip sitt og héldu í haf og er engi frásögn um ferð þeirra fyrr en þeir koma til Vínlands til Leifsbúða og bjuggu þar um skip sitt og sátu um kyrrt þann vetur og veiddu fiska til \textbf{skip} sér\\
\hline
1591&skip&nheo&En Lofti kom njósn um kvöldið og reið hann ofan til sjóvar og fékk sér þar skip og fór hann þá suður yfir fjörð til \textbf{skip} \\
\hline
1592&skip&nheo&Flosi sagði vera ærið gott gömlum og feigum og sté á skip og lét í haf og hefir til þess skips aldrei spurst \textbf{skip} \\
\hline
1593&skip&nheo&En þegar sjór féll undir skip þeirra þá tóku þeir bátinn og reru til skipsins og fluttu það upp í \textbf{skip} \\
\hline
1594&skip&nhfþg&Og einn morgun snemma er þeir lituðust um sáu þeir níu húðkeipa og var veift trjánum af skipunum og lét því líkast í sem í hálmþústum og fer \textbf{skipunum} \\
\hline
1595&skip&nheþ&Eg kom skipi mínu í Leiruvog fyrir neðan Heiði fyrir fám vetrum og átti eg að gjalda hálfa mörk silfurs húskarli Hrafns og hélt eg því fyrir \textbf{skipi} \\
\hline
1596&skip&nheþ&Hann var á móti Haraldi konungi og lá á annað borð skipi \textbf{skipi} \\
\hline
1597&skip&nhfo&Sigvaldi jarl lét skotta við sín skip og lagði ekki til \textbf{skip} \\
\hline
1598&skip&nhen&Oddur hljóp þegar fyrir borð er skip hans kenndi grunns og hver að öðrum hans \textbf{skip} \\
\hline
1599&skip&nheþg&Þeir Sigmundur stökuðu húskörlum Steins og ráku þá frá \textbf{skipinu} \\
\hline
1600&skip&nhfn&Þá sá Ólafur konungur að þeir Erlingur sóttu eftir mjög því að skip konungs voru sett mjög og sollin er þau höfðu flotið á sæ allt sumarið og um haustið og veturinn þar \textbf{skip} \\
\hline
1601&skip&nheog&Þeir snúa nú eftir þeim Sveini og lætur hann eltast að landi og þegar hljóp Sveinn á land upp en þeir Halldór tóku skipið og fóru til \textbf{skipið} \\
\hline
1602&skip&nhfþg&og tóku þá og reru öllum skipunum forstreymis sem mest máttu \textbf{skipunum} \\
\hline
1603&skip&nheþ&Á því sumri kom Grímur skipi sínu á Eyrar í þá höfn er Knarrarsund heitir og var um veturinn með þeim manni er Þorkell \textbf{skipi} \\
\hline
1604&skip&nhfo&Hann hafði tvö skip til félags við hann og lögðu síðan austur fyrir Svíþjóð í hernað og herjuðu um sumarið en voru um veturinn í \textbf{skip} \\
\hline
1605&skip&nheng&En fyrir þá sök að skipið var borðmikið svo sem borg væri en fjöldi manns á og valið hið besta \textbf{skipið} \\
\hline
1606&skjöldur&nkeng&En Björn hélt á skildinum svo að handleggur hans var í mundriðanum og kom höggið á skjöldinn og varð svo mikið að handleggur Bjarnar gekk í sundur og féll skjöldurinn \textbf{skjöldurinn} \\
\hline
1607&skjöldur&nkeog&Í þessu hjó Þorfinnur til Þórðar og kom í skjöldinn og sneið af mikinn mána af \textbf{skjöldinn} \\
\hline
1608&skjöldur&nkeog&Í móti Gunnari gekk Vandill og hjó þegar til hans og kom í \textbf{skjöldinn} \\
\hline
1609&skjöldur&nkeo&» Jökull þrífur eitt spjót og leggur til Finnboga í skjöld hans og gekk í sundur \textbf{skjöld} \\
\hline
1610&skjöldur&nkeog&Gunnbjörn brá sverði og hjó til Jökuls og klauf allan skjöldinn öðrum megin \textbf{skjöldinn} \\
\hline
1611&skjöldur&nkeo&Hrafn átti fyrr að höggva er á hann var skorað og hjó hann í skjöld Gunnlaugs ofanverðan og brast sverðið þegar sundur undir hjöltunum er til var hoggið af miklu \textbf{skjöld} \\
\hline
1612&skjöldur&nkfþg&» En er þeir Þorgrímur komu yfir díkið köstuðu þeir skjöldunum á bak sér og runnu til skipa \textbf{skjöldunum} \\
\hline
1613&skjöldur&nkeo&þá er þeir lögðust til svefns að hver hafði hjálm á höfði og skjöld yfir sér og sverð undir höfði og skyldi leggja hægri hönd á \textbf{skjöld} \\
\hline
1614&sonur&nkfn&Synir \textbf{Synir} \\
\hline
1615&sonur&nkfn&Þá er synir Magnúss voru til konunga teknir komu utan úr Jórsalaheimi og sumir úr Miklagarði þeir menn er farið höfðu út með Skopta Ögmundarsyni og voru þeir hinir frægstu og kunnu margs konar tíðindi að \textbf{synir} \\
\hline
1616&sonur&nkfo&« færa mér sonu Guttorms en dætur hans skulu þar upp fæðast til þess er eg gifti \textbf{sonu} \\
\hline
1617&sonur&nken&Stökk þá Hallgerður til Grjótár og Grani sonur \textbf{sonur} \\
\hline
1618&sonur&nken&Hann var sonur Sighvats hins \textbf{sonur} \\
\hline
1619&sonur&nkfn&Á þessu sama þingi sóttu þeir Þorgestur hinn gamli og synir Þórðar gellis Eirík hinn rauða um víg sona Þorgests er látist höfðu um haustið þá er Eiríkur sótti setstokkana á Breiðabólstað og var þetta þing \textbf{synir} \\
\hline
1620&sonur&nken&Það var sagt eitthvert sumar að Guðmundur son hans hafði fallið í bardaga en það hafði þó logið \textbf{son} \\
\hline
1621&sonur&nken&Íslsag Sturla son Þórðar Gilssonar bjó í Hvammi vel þrjá tigu \textbf{son} \\
\hline
1622&sonur&nken&En það lét hann fylgja að honum þótti Einar best fallinn til að bera tignarnafn í Noregi ef eigi væri jarls við kostur eða sonur hans Eindriði fyrir ættar sakir \textbf{sonur} \\
\hline
1623&sonur&nken&Áskell son Skeggja Árnasonar hafði verið með Sturlu og farið til \textbf{son} um morguninn\\
\hline
1624&sonur&nkfn&Þá voru með Ólafi konungi synir hans Kálfur og \textbf{synir} \\
\hline
1625&sonur&nkfn&Bolli gekk í móti þeim Ólafi og synir Ósvífurs og fagna þeim \textbf{synir} \\
\hline
1626&sonur&nken&Jón sonur Hallkels húks safnaði bóndaliði og fór að þeim og tók Kolbein óða og drap hvert barn af skipi \textbf{sonur} \\
\hline
1627&sonur&nken&Steinólfur var og sonur Ölvis \textbf{sonur} \\
\hline
1628&sonur&nkeo&Þorsteinn hét son \textbf{son} laungetinn\\
\hline
1629&sonur&nken&» Hann kvað það einsætt og réðst til ferðar með þeim en Eyjólfur son hans var farinn upp í \textbf{son} \\
\hline
1630&sonur&nkeo&Þorkell hét son \textbf{son} \\
\hline
1631&sonur&nkeo&Þar með vil eg gefa þér Grím son minn til fylgdar og \textbf{son} \\
\hline
1632&sonur&nkfþ&Þórður bróðir Þorgeirs gekk þar mest í millum manna og kvað Þorgeiri mjög missýnast er hann gekk í mót sonum sínum í \textbf{sonum} \\
\hline
1633&sonur&nkfn&Eftir það tók Snorri bróðir hans mannvirðing í Vatnsfirði og voru hans synir \textbf{synir} \\
\hline
1634&sonur&nken&Var þar hin harðasta atsókn því að Þórir var allreiður og lauk svo að Grímur féll og húskarlar hans tveir en Hergils son hans komst út um laundyr og varð Gunnar var við hann og hljóp eftir og vó hann þar er nú heita \textbf{son} \\
\hline
1635&sonur&nken&« Þú skalt ríða til öndverðs þings og tjalda búðir vorar og með þér Þórhallur son minn því að þú munt best og hóglegast með hann fara er hann er fótlami en vér munum hans mest þurfa á þessu \textbf{son} \\
\hline
1636&sonur&nkfn&Er þeir Eyvindur voru komnir á vestanverða þá kenna þeir að Hrafnkell var í eftirreiðinni og synir hans báðir og marga aðra menn kenndu \textbf{synir} \\
\hline
1637&sonur&nken&En Hallur son hans og þau Ingibjörg lágu þar fyrir utan þilið næst í stafnrekkju og var gluggur kringlóttur á þilinu milli \textbf{son} \\
\hline
1638&sonur&nken&Eftir fall Hákonar tók Sigurður sonur hans ríki og gerðist jarl í \textbf{sonur} \\
\hline
1639&sonur&nkeo&hans son var Önundur \textbf{son} \\
\hline
1640&sonur&nkeo&Melkorka vekur tal við Ólaf son sinn þá er þau finnast að hún vill að hann fari utan að vitja frænda sinna göfugra « því að eg hefi það satt sagt að Mýrkjartan er að vísu faðir minn og er hann konungur \textbf{son} \\
\hline
1641&sonur&nkfo&Hann eignaðist ríki víða um Saxland og setti þar sonu sína til \textbf{sonu} \\
\hline
1642&sonur&nkeo&þeirra son Jón er fyrr átti Þóru dóttur Guðmundar \textbf{son} \\
\hline
1643&sonur&nken&Hrafn og Sturla son hans og tókust þeir í hendur fyrir kirkjudurum á \textbf{son} \\
\hline
1644&sonur&nkfo&Sé eg það gjörla þó að oss veitti það að vilja að vér dræpum Njál eða sonu \textbf{sonu} \\
\hline
1645&sonur&nkeo&Eindriði hét son \textbf{son} \\
\hline
1646&sonur&nken&Þorgrímur hét sonur þeirra og var kenndur við móður sína og var kallaður Hlífarson fyrir því að hún lifði lengur en \textbf{sonur} \\
\hline
1647&sonur&nken&sonur \textbf{sonur} frá Fróðá\\
\hline
1648&sonur&nken-s&Son Þorsteins var Þorkell máni \textbf{Son} \\
\hline
1649&sonur&nkeo&son \textbf{son} \\
\hline
1650&sonur&nken&Þeirra sonur var Hallur faðir \textbf{sonur} \\
\hline
1651&sonur&nkeo&Þá segir Ölvir hnúfa að þar er kominn son Kveld-Úlfs « sem eg sagði yður í sumar að Kveld-Úlfur mundi senda til \textbf{son} \\
\hline
1652&sonur&nken&Álfur sonur Yngvars konungs og Ingjaldur sonur Önundar \textbf{sonur} \\
\hline
1653&sonur&nkfn&synir Breiðár-Skeggja og allir aðrir Íslendingar þeir sem þar voru í \textbf{synir} \\
\hline
1654&sonur&nkfn&synir Hallbjarnar \textbf{synir} úr Laxárdal\\
\hline
1655&sonur&nkeþ&Flosi sendi orð Kol Þorsteinssyni og Glúmi syni Hildis hins \textbf{syni} \\
\hline
1656&sonur&nkeo&Kjalnesinga saga Helgi bjóla son Ketils flatnefs nam Kjalarnes millum Leiruvogs og Botnsár og bjó að Hofi á \textbf{son} \\
\hline
1657&sonur&nken&Þeir Böðvar son hans höfðu þar sex tigu \textbf{son} \\
\hline
1658&sonur&nkeo&Þá var og heim kominn Eyjólfur son \textbf{son} \\
\hline
1659&sonur&nkeo&son Torfa \textbf{son} \\
\hline
1660&sonur&nkfn&Eftir Torf-Einar réðu fyrir löndum synir \textbf{synir} \\
\hline
1661&sonur&nkeo&Þorgils hét son \textbf{son} \\
\hline
1662&sonur&nkfn&Synir Þórdísar fjórir voru á vist með \textbf{Synir} \\
\hline
1663&sonur&nkfo&Ólafur kvað þá þegar skyldu drepa þau Kotkel og konu hans og sonu « er þó ofseinað \textbf{sonu} \\
\hline
1664&sonur&nkfn&» Ólafur þakkar honum boð þetta með mikilli snilld og fögrum orðum en kvaðst þó eigi mundu á hætta hversu synir hans þyldu það þá er Mýrkjartans missti \textbf{synir} \\
\hline
1665&sonur&nkfn&Hún gerði heiman dóttur Orms en synir hans höfðu \textbf{synir} \\
\hline
1666&sonur&nken&Hallfreður son hans var þá með \textbf{son} \\
\hline
1667&sonur&nkfn&En eigi skaltu flýja úr héraðinu án mínu ráði né synir þínir því að vér erum maklegastir Austfirðingar að hræra um \textbf{synir} vandræði\\
\hline
1668&sonur&nken&sonur Ólafs hvíta og Auðar \textbf{sonur} djúpúðgu\\
\hline
1669&sonur&nkeo&son Hávarðar halta og Bjargeyjar \textbf{son} \\
\hline
1670&sonur&nken&Son hennar var henni mjög líkur í \textbf{Son} \\
\hline
1671&sonur&nken&Ásbjörn hét annar son \textbf{son} \\
\hline
1672&sonur&nken&Svo og Magnús konungur sonur minn má og eigi mér þess synja en eg vil við yður vera skeyttur og skyldur til allrar þjónustu þeirrar er því nafni \textbf{sonur} \\
\hline
1673&sonur&nken&Þar var á Helga kona hans og Þormóður fóstri hennar og Þormóður son þeirra og þeir tólf menn er úr byggð \textbf{son} \\
\hline
1674&sonur&nkeo&hans son var Hallur \textbf{son} \\
\hline
1675&sonur&nken&Var þar Ólafur son hans og móðir \textbf{son} \\
\hline
1676&sonur&nkeo&Snorri hét enn son \textbf{son} \\
\hline
1677&sonur&nkfn&Var með honum Einar frændi hans og synir Halldórs \textbf{synir} \\
\hline
1678&sonur&nkeo&Gissur glaði varð skjótastur á fætur og Björn son Ólafs \textbf{son} \\
\hline
1679&sonur&nkeo&Þórð son sinn og Þórð Heinreksson til \textbf{son} \\
\hline
1680&sonur&nkeo&Hann var son Arnórs heynefs Þóroddssonar er numið hafði Hrútafjörð þeim megin til \textbf{son} við Bakka\\
\hline
1681&sonur&nkfn&Tóku synir hans þar við búi og \textbf{synir} \\
\hline
1682&sonur&nkfn&Það hafði mikill siður verið í Noregi að lendra manna synir eða ríkra búanda fóru á herskip og öfluðu sér svo fjár að þeir herjuðu bæði utanlands og \textbf{synir} \\
\hline
1683&sonur&nken&Herjólfur son hans var þá átta \textbf{son} \\
\hline
1684&sonur&nkfo&tók sonu þeirra eða bræður eða aðra náfrændur eða þá menn er þeim voru kærstir og honum þóttu best til \textbf{sonu} \\
\hline
1685&sonur&nkfn&þeirra synir Gunnlaugur \textbf{synir} \\
\hline
1686&sonur&nkfo&« er hann lét eigi aka í skegg sér að hann væri sem aðrir karlmenn og köllum hann nú karl hinn skegglausa en sonu hans taðskegglinga og kveð þú um nokkuð Sigmundur og lát oss njóta þess er þú ert \textbf{sonu} \\
\hline
1687&sonur&nken&Þorfinnur og Þorbrandur sonur \textbf{sonur} \\
\hline
1688&sonur&nkfn&Nú þróast synir \textbf{synir} \\
\hline
1689&sonur&nkfn&Synir hans voru þeir Kolur og Óttar og \textbf{Synir} \\
\hline
1690&sonur&nken&En er Ólafur konungur Tryggvason kom á Hringaríki að boða kristni þá lét skírast Sigurður sýr og Ásta kona hans og Ólafur sonur hennar og gerði Ólafur Tryggvason guðsifjar við Ólaf \textbf{sonur} \\
\hline
1691&sonur&nken&Hann hefir her af öllum þeim ríkjum er áður hafði hann og kemur til liðs við þá Högni konungur og Hildir sonur hans er réðu fyrir \textbf{sonur} \\
\hline
1692&sonur&nkfn&Þóroddur sótti Snorra goða að eftirmáli um víg Þórissona og lét Snorri búa mál til Þórsnessþings en synir Þorláks á Eyri veittu Breiðvíkingum að málum \textbf{synir} \\
\hline
1693&sonur&nkeo&er Haraldur konungur hafði haft réðu það að hleypiskip var gert og sent norður til Þrándheims að segja fall Haralds konungs og það með að Þrændir skyldu taka til konungs son Haralds \textbf{son} \\
\hline
1694&sonur&nken-s&Son þeirra var \textbf{Son} \\
\hline
1695&sonur&nkfn&Þorbjörn Vífilsson og synir Þorbrands úr Álftafirði en Þorgesti veittu synir Þórðar gellis og Þorgeir úr Hítardal og Áslákur úr Langadal og Illugi son \textbf{synir} \\
\hline
1696&sonur&nken&Þorleifur beiskaldi og Ari hinn sterki Þorgilsson því að Magnús sonur Páls átti Hallfríði systur \textbf{sonur} \\
\hline
1697&sonur&nkfn&Njáll dvaldist skamma stund heima áður hann reið austur til Svínafells og synir hans og Höskuldur og vekur bónorðið við Flosa en hann kveðst efna mundu öll mál við \textbf{synir} \\
\hline
1698&sonur&nken&Skyldi og fara til konungs Skjálgur sonur \textbf{sonur} og vera með honum\\
\hline
1699&sonur&nken&Kyrpinga-Ormur var sonur Sveins Sveinssonar \textbf{sonur} úr Gerði\\
\hline
1700&sonur&nkfn&En er þetta spurði Gunnhildur og synir \textbf{synir} \\
\hline
1701&sonur&nkeo&Þórður hét son \textbf{son} \\
\hline
1702&sonur&nken&Hálfdan háleggur sonur Haralds hárfagra fór á hendur Torf-Einari og rak hann á brott úr \textbf{sonur} \\
\hline
1703&sonur&nkeo&hans son var \textbf{son} \\
\hline
1704&sonur&nkfo&En Vísbur lét hana eina og fékk annarrar konu en hún fór til föður síns með sonu \textbf{sonu} \\
\hline
1705&sonur&nken&Helgi hét annar son \textbf{son} \\
\hline
1706&sonur&nkeo&« Það er hugarboð mitt Syrpa að Urðarköttur sé ekki \textbf{son} son\\
\hline
1707&sonur&nkeo&hans son var \textbf{son} \\
\hline
1708&sonur&nken&Eftir fundinn var Þorgeiri það sagt að Höskuldur sonur hans var mjög sár og báðu menn hann skiljast við mál þessi og vera eigi í móti sonum \textbf{sonur} \\
\hline
1709&sonur&nkeo&Jón var son þeirra er mestur höfðingi og vinsælastur hefir verið á \textbf{son} \\
\hline
1710&sonur&nkfn&Voru þeir synir Ósvífurs óðastir á þetta mál en Bolli svafði \textbf{synir} \\
\hline
1711&sonur&nkfþ&Þetta spurði Hrútur og líkar illa og sonum \textbf{sonum} \\
\hline
1712&sonur&nken&Þórarinn son Jósteins var knáastur maður annar en \textbf{son} \\
\hline
1713&sonur&nken&Þá er þeir Sveinn og Hákon réðu Noregi gerðu þeir sátt við Erling Skjálgsson og var bundið með því að Áslákur sonur Erlings fékk Gunnhildar dóttur Sveins \textbf{sonur} \\
\hline
1714&sonur&nkfn&Þorkell og Eyjólfur hétu synir \textbf{synir} \\
\hline
1715&sonur&nkeo&Nú hefir minn son orðið fyrir áverka og þykir mér hann gildur maður fyrir sér og mun eg þó eigi sjá þann hlut til handa honum eða mér að gera almúginum vandræði í heldur mun eg bíða og leita mér \textbf{son} og fara heim að sinni\\
\hline
1716&sonur&nkfn&synir \textbf{synir} \\
\hline
1717&sonur&nkfn&En er við urðu varir Knúts konungs menn þá drógu þeir her saman og urðu brátt fjölmennir svo að synir Aðalráðs konungs höfðu ekki liðsafla við og sáu þann sinn kost helst að halda í brott og aftur vestur til \textbf{synir} \\
\hline
1718&sonur&nkfn&er áttu synir Brynjólfs hins \textbf{synir} \\
\hline
1719&sonur&nken&Sonur þeirra hét Helgi en dóttir \textbf{Sonur} \\
\hline
1720&sonur&nken&Eindriði hét sonur \textbf{sonur} \\
\hline
1721&sonur&nkfn&Nú er frá því sagt að þeir synir Þorgauts rísa upp allir og fara að slá á Gullteig og ræddu um það að nú mundi vel vita og nú mundi sleginn verða Gullteigur hinn sama \textbf{synir} \\
\hline
1722&sonur&nken&Ólafur son Haralds konungs fór og með honum úr \textbf{son} \\
\hline
1723&sonur&nken&Þar kom til hið fyrsta haust Snorri son Karlsefnis og var hann þá þrívetur er þeir fóru á \textbf{son} \\
\hline
1724&sonur&nkeo&en sendi son sinn til Þverár og bað segja Glúmi fyrirætlan þeirra Esphælinga « en ríð til móts við mig síðan \textbf{son} \\
\hline
1725&sonur&nkfo&Þessi Auðun vildi eigi utan flytja sonu Ósvífs hins spaka eftir víg Kjartans Ólafssonar sem segir í Laxdæla sögu og varð það þó síðar en \textbf{sonu} \\
\hline
1726&sonur&nkeo&Helgi var son Eyvindar Austmanns og Raförtu dóttur Kjarvals \textbf{son} \\
\hline
1727&sonur&nkfn&Ásmundur skegglaus og Ásbjörn voru synir Ófeigs grettis en dætur hans voru þær \textbf{synir} \\
\hline
1728&sonur&nkfþ&Sýnist mér það ráð meðan við erum á lífi og svo nær staddir deilu þeirra að við tökum mál þetta undir okkur og setjum niður en látum eigi þá Tungu-Odd og Einar etja saman sonum okkrum sem \textbf{sonum} \\
\hline
1729&sonur&nkeo&Þorleifur Hrómundarfóstri var son \textbf{son} \\
\hline
1730&sonur&nken-s&Son Ívars hét \textbf{Son} \\
\hline
1731&sonur&nkfo&Nú mun eg gera yður þann kost að fylgja yður þar til þér sjáið Ref og sonu hans og þá tólf menn sem honum hafa fylgt í sumar enda skal eg eignast eyri silfurs af hverjum yðrum og einn kostgrip af skipi yðru enda skuluð þér skyldir að flytja mig til \textbf{sonu} \\
\hline
1732&sonur&nken&« að Skarphéðinn fari til þín og Höskuldur sonur minn og munu þeir leggja sitt líf við þitt \textbf{sonur} \\
\hline
1733&sonur&nkeþ&Hann fer á fund Ólafs og bauð Halldóri syni hans til \textbf{syni} \\
\hline
1734&sonur&nken&» Hallur sonur Þorgeirs læknis Steinssonar var hirðmaður Inga konungs og var viðstaddur þessi \textbf{sonur} \\
\hline
1735&sonur&nken&Þorbjörn var í haug lagður en Hallsteinn sonur hans var \textbf{sonur} \\
\hline
1736&spjót&nheo&Hann spratt upp skjótt og hljóp á bak og tekur upp spjót og reiðir upp spjótshalann og lýstur til \textbf{spjót} \\
\hline
1737&spjót&nhen&Öngvan hlut rak upp af fé Þorvalds nema eitt spjót \textbf{spjót} \\
\hline
1738&spjót&nheo&Tveir húskarlar Snorra sáu að hann hljóp út reiður með spjót \textbf{spjót} \\
\hline
1739&sumar&nheog&En Gellir bjóst til ferðar og fór um sumarið til Íslands og hafði með sér orðsendingar þær þangað er hann flutti fram annað sumar á \textbf{sumarið} \\
\hline
1740&sumar&nheog&Um sumarið eftir riðu þeir Hrafn og Þorvaldur með fjölmenni miklu til \textbf{sumarið} \\
\hline
1741&sumar&nheo&Bjó hann skip sitt til Íslands og kom út næsta sumar eftir brúðkaupið í Reyðarfjörð og hafði selt Austmönnum \textbf{sumar} \\
\hline
1742&sumar&nheþ&Að áliðnu sumri fór Þorbjörn öngull með alskipaða skútu til Drangeyjar en þeir gengu fram á \textbf{sumri} \\
\hline
1743&sumar&nheog&En um sumarið vill hann fara til \textbf{sumarið} að hitta Þórodd\\
\hline
1744&sumar&nheog&Og um sumarið annað eftir á alþingi mælti Illugi svarti til \textbf{sumarið} að Lögbergi\\
\hline
1745&sumar&nheog&Hann fór til \textbf{sumarið} of sumarið\\
\hline
1746&sumar&nheo&Lýsi eg nú til sóknar í sumar og til sektar fullrar á hönd Glúmi \textbf{sumar} \\
\hline
1747&sumar&nheo&Og ætlaði eg það eigi að þú mundir nú reka mig á brottu þar sem eg hefi unnið hér til sprengs í sumar og vonað til þess að þú mundir mér nokkura forstöðu veita en þann veg farið þér þó að þér látið \textbf{sumar} \\
\hline
1748&sumar&nheog&Þá fór Erlendur í fjárbón of sumarið og rifnuðu aftur sár hans og urðu seint \textbf{sumarið} \\
\hline
1749&sumar&nhen&Það sumar fór hinn heilagi Þorlákur biskup fyrsta sinn of \textbf{sumar} \\
\hline
1750&sumar&nheog&Og er það sumra manna sögn að þessi Þorgils kæmi til Íslands fyrir Fróðárundur um \textbf{sumarið} \\
\hline
1751&sumar&nheo&Þetta sumar fór Guðmundur prestur hinn góði til þings en af þingi buðu honum heim Sunnlendingar og Austfirðingar og fór hann af þinginu suður í \textbf{sumar} \\
\hline
1752&sumar&nheog&Og er á líður sumarið og haustar koma aftur draumar hans og þungar \textbf{sumarið} \\
\hline
1753&sumar&nheog&Þegar eftir jólin byrjaði Ólafur konungur ferð sína til Upplanda því að hann hafði fjölmenni mikið en tekjur norðan úr landi höfðu engar til hans komið þá um haustið því að leiðangur hafði úti verið um sumarið og hafði þar konungur allan kostnað til \textbf{sumarið} \\
\hline
1754&sumar&nheog&Og nú um sumarið fjölmenntu hvorirtveggju til alþings eftir föngum þá er þar kom og riðu menn á þing hinn næsta dag fyrir Jóns messu \textbf{sumarið} \\
\hline
1755&sumar&nheog&sögðu að Þorsteini skyldi sá bestur að þar félli hvert mál sem komið var og sögðu hitt maklegra að hann gyldi sjálfan sig fyrir þá svívirðing er þeir Illugi mágur hans höfðu gert til þeirra hið fyrra \textbf{sumarið} \\
\hline
1756&sumar&nheog&lét það og fylgja að hann sjálfur mundi koma þegar um sumarið til Danmerkur með liði sínu og lét þar fylgja ályktarorð að hann skyldi eignast allt Danaveldi svo sem stóðu til einkamál og svardagar eða falla sjálfur í orustu með her \textbf{sumarið} \\
\hline
1757&sumar&nheþ&Á einhverju sumri fór hann með her sinn til \textbf{sumri} \\
\hline
1758&sumar&nhfþ&Hann vann til fjár sér á sumrum en á vetrum var hann vistlaus og fór þá með kaupvarning \textbf{sumrum} \\
\hline
1759&sumar&nhfo&En þó verða þessi málalok að í sætt var slegið og skulu þar gjaldast þrír tigir hundraða fyrir víg Hneitis en níu hundruð fyrir áverka við Má og sekt Ólafs slík að hann skal leita við utanför þrjú sumur og varða eigi bjargir \textbf{sumur} \\
\hline
1760&sumar&nheog&Nú er þar til að taka að Þórði batnar sára sinna og ríður þaðan inn til Hrafnagils og smíðaði þar skála um \textbf{sumarið} \\
\hline
1761&sumar&nheo&Eg þykist þess kenna í bréfi hans að hann mun ætla utan í sumar ef hann er eigi frá kosinn því að hann bað mig skjótt að kveða hvort eg vildi kjósa hann til eða \textbf{sumar} \\
\hline
1762&sumar&nheþ&Á einu sumri herjuðu þeir í Danmörk og þá hittu þeir Guðlaug Háleygjakonung og áttu við hann orustu og lauk svo að skip Guðlaugs var hroðið en hann varð \textbf{sumri} \\
\hline
1763&sumar&nhen&Það var sagt eitthvert sumar að Guðmundur son hans hafði fallið í bardaga en það hafði þó logið \textbf{sumar} \\
\hline
1764&sumar&nheo&Þorbjörn Þjóðreksson reið hvert sumar til þings með menn \textbf{sumar} \\
\hline
1765&sumar&nheo&« Málin við Hrafn ef hann fær Helgu hinnar vænu að veturnóttum og var eg hjá í sumar á alþingi er það \textbf{sumar} \\
\hline
1766&sumar&nheþ&En að sumri búast þeir báðir til þingreiðar Sturla Langavatnsdal en Einar \textbf{sumri} \\
\hline
1767&sumar&nheo&Grænlands að boða þar kristni og fór hann það sumar til \textbf{sumar} \\
\hline
1768&sumar&nheo&Það sumar eftir Haugsnessfund og fall Brands var friður á \textbf{sumar} \\
\hline
1769&sumar&nheþ&Var það allt á einu sumri er Eiríkur konungur fór af Hörðalandi með her sinn austur í Vík til bardaga við bræður sína og áður höfðu þeir deilt á Gulaþingi Egill og Berg-Önundur og þessi tíðindi er nú var \textbf{sumri} \\
\hline
1770&sumar&nheog&» Eftir það fóru þeir biskup heim til Lækjamóts og dvöldust þar um \textbf{sumarið} \\
\hline
1771&sumar&nheog&Um sumarið eftir fóru þeir til Íslands og urðu \textbf{sumarið} \\
\hline
1772&sumar&nheo&Mig minnir að hann lýsti til fjár á hendur yður í sumar á alþingi og mun svo ætla að gera á hendur yður stelafé að þér finnið eigi fyrr en hann hefir sekta yður og mun þá ætla sér landið það er þið búið á og mun hann nýta að eiga land jafnt fyrir vestan heiði sem fyrir austan eða \textbf{sumar} \\
\hline
1773&sumar&nheog&Flutti Þórður það mjög og var brúðlaup þeirra að Sauðafelli um \textbf{sumarið} \\
\hline
1774&sumar&nheo&Nú er Þórhallur þóttist spyrja kaup þeirra Sveins og Þorsteins þá hitti hann þá Þorleif beiskalda og Einar Þorgilsson eitt sumar á þingi og segir þeim svo að hann þóttist við mikil vandræði kominn \textbf{sumar} \\
\hline
1775&sumar&nheog&Þórólfur fór ferðar sinnar um sumarið og greiddist vel um hafið og komu utan að \textbf{sumarið} \\
\hline
1776&sumar&nheog&Þá er Snorri spurði að Sturla hafði lagið undir sig hérað allt fór hann brott af Suðurnesjum og suður til búa sinna og þaðan austur í Skál til Orms Svínfellings og var þar um \textbf{sumarið} \\
\hline
1777&sumar&nheo&En er þetta var allt ráðið þá bjuggu þeir skip sín til hafs og fóru þeir allir í skip með Eysteini hvíta og fóru þetta sumar til \textbf{sumar} \\
\hline
1778&sumar&nheog&Þá er Ásbjörn var riðinn til þings um sumarið hafði Skíði tekið í brott meyna með ráði Þorgerðar móður \textbf{sumarið} \\
\hline
1779&sumar&nheog&Um sumarið eftir á þingi gerði biskup gerð þessa við ráð Páls biskups og Sæmundar úr Odda tólf hundruð vaðmála á hendur \textbf{sumarið} \\
\hline
1780&sumar&nheog&Bjó þá Ögmundur skip sitt og fór út til Íslands um sumarið og hafði aflað mikils fjár í ferð \textbf{sumarið} \\
\hline
1781&sumar&nhen&En þó lét konungur leyft honum verða það sumar út að fara til \textbf{sumar} \\
\hline
1782&sveinn&nkfng&Og þennan vetur er Gróa er þar voru sveinarnir Helgi og Grímur heima og var Gróa vel til \textbf{sveinarnir} \\
\hline
1783&sveinn&nkeo&Kveðst hún hyggja það svein vera « og mun mörgum þykja mikils um hann vert sakir framkvæmdar sinnar en ekki kæmi mér það á óvart þó að eigi stæði hans hagur með hinum mesta blóma áður \textbf{svein} \\
\hline
1784&sveinn&nkeng&» Nú settist Björn niður en sveinninn fór að taka hrossin og vildi víkja og mátti eigi því að þá hafði tekist fundur \textbf{sveinninn} \\
\hline
1785&sveinn&nken-s&Lagði Sveinn þar á hendur sínar og sór trúnaðareiða Magnúsi \textbf{Sveinn} \\
\hline
1786&sveinn&nken-s&» Svo komu fortölur hans að Sveinn játaði að fylgja þessu \textbf{Sveinn} \\
\hline
1787&sveinn&nkeþg&Viltu árna sveininum \textbf{sveininum} \\
\hline
1788&sveinn&nken&Þeim varð barns auðið og var það sveinn og hét Jón og lifði litla \textbf{sveinn} \\
\hline
1789&sveinn&nkeng&En þá er sveinninn var tvævetur þá var hann almæltur og rann einn saman sem fjögurra vetra gömul \textbf{sveinninn} \\
\hline
1790&sveinn&nkfng&Þangað fara sveinarnir til hans og biðja hann búðarrúms þar en Hallbjörn var góður viðtakna og kveðst enn hverjum manni veitt hafa búðarrúm hér til er hann hafði \textbf{sveinarnir} \\
\hline
1791&sveinn&nkeng&Og í því hljóp sveinn Knúts úr koluskugga og hjó á öxl Jóni mikið sár en Jón slæmdi öxi á bak sér og varð sveinninn sár \textbf{sveinninn} \\
\hline
1792&sveinn&nken&Þá spurði Gestur Syrpu hvað sveinn þeirra skyldi \textbf{sveinn} \\
\hline
1793&sveinn&nken-s&Þá spurði hann þau tíðindi að Sveinn konungur var kominn til Sjálands og þá var kominn til hans her sjá allur er flúið hafði úr orustu og mikið lið annað og hafði hann ógrynni hers \textbf{Sveinn} \\
\hline
1794&sveinn&nken&Hann sest niður hjá henni og taka þau tal sín á milli og í því kemur sveinn frá selinu og biður Jófríði taka ofan klyfjar með \textbf{sveinn} \\
\hline
1795&sveinn&nken-s&Sveinn son hans var þar með \textbf{Sveinn} \\
\hline
1796&sveinn&nkeo&Þá fór Álfheiður húsfreyja hans með barni og fæddi hún á þeim misserum svein og var kallaður eftir föður sínum \textbf{svein} \\
\hline
1797&sveinn&nkfn&En þeir segja að sveinar tveir hafi komið í flokkinn þeirra og segja að þetta kemur mjög að þeim óvörum og kváðust eigi deili á þeim \textbf{sveinar} \\
\hline
1798&sveinn&nken-s&Og eftir dagverð drottinsdag fór Sturla heiman og Sveinn son hans til \textbf{Sveinn} \\
\hline
1799&sverð&nheog&Og í því hjó Hjarrandi til Helga en hann brá við skildinum og hljóp af sverðið í andlit honum og kom á tanngarðinn og af vörina \textbf{sverðið} \\
\hline
1800&sverð&nheþ&Lagði Óttar sverði til Sokka neðan undir brynjuna og svo upp í kviðinn að það varð þegar \textbf{sverði} bani\\
\hline
1801&sverð&nhfn&Og nú setti hann sjálfur þau ráð til að hann mun brjóta fatakistu Þorbergs og taka á brott feld hans og kyrtil og sverð en ganga síðan fyrir hann og heitast við hann sjálfan og taka meira ef hann vildi eigi þetta gefa honum en fara síðan til Skútu og biðja hann ásjá og sagði hann eigi gruna mundu ef hann færi þann veg \textbf{sverð} \\
\hline
1802&sverð&nheþg&Kári slæmdi til þessa manns sverðinu og hjó hann í sundur í \textbf{sverðinu} \\
\hline
1803&sverð&nheog&Hann hafði í hendi sverðið Ættartanga og gekk síðan að \textbf{sverðið} \\
\hline
1804&sverð&nhfn&» Hann tók fyrst sverð þeirra og bar upp að \textbf{sverð} \\
\hline
1805&sverð&nheþg&Og eitt högg höggur Þóroddur til Þorbjarnar og af fótinn í ristarliðnum og ei berst hann að síður og leggur fram sverðinu í kvið Þóroddi og fellur hann og liggja úti \textbf{sverðinu} \\
\hline
1806&sverð&nhen&Það sverð hefir best komið til \textbf{sverð} \\
\hline
1807&sverð&nheog&Hafði Gísli þá í hendi sverðið Grásíðu en Kolur \textbf{sverðið} \\
\hline
1808&sverð&nheog&Þorgils gekk einn morgun til búðar Bjarna og tók sverðið Jarðhússnaut í hönd \textbf{sverðið} \\
\hline
1809&sverð&nheng&Hjó Kolur í höfuð Gísla svo að í heila stóð en sverðið kom í höfuð Koli og beit ekki en þó var svo fast til höggvið að hausinn rifnaði en sverðið brast \textbf{sverðið} \\
\hline
1810&sverð&nheþ&Karl brá þá sverði og mælti til \textbf{sverði} \\
\hline
1811&sverð&nheng&Jökull höggur þá með báðum höndum til Gunnars en hann veik sér undan högginu svo að hann náði honum ekki en sverðið kom í stóran \textbf{sverðið} \\
\hline
1812&sök&nvfo&» Sæmundur kvaðst eigi mundu hætta lífi sínu fyrir \textbf{sakir} sakir\\
\hline
1813&sök&nvfo&þriðji Guðmundur grís er fleira veitti fyrir guðs sakir en flestir menn aðrir en gerði síðan eftir \textbf{sakir} boðorðum\\
\hline
1814&sök&nvfo&Skegg-Ávaldi átti búð saman og Hermundur son hans og er Galti Óttarsson var genginn erinda sinna mætti hann Hermundi en hann minntist á sakir þær er Hallfreður hafði gert við þá og hljóp að Galta og drap hann og fór síðan í búð til föður \textbf{sakir} \\
\hline
1815&sök&nvfo&Björn segir þá og tekur til að upphafi um viðskipti þeirra Þórðar og sakir þær er hann þóttist eiga við Þórð \textbf{sakir} \\
\hline
1816&sök&nveþ&« Seint leiðist þér að biðja griða til handa Gretti en grunar mig að þú hafir þar eigi gott að \textbf{sök} \\
\hline
1817&sök&nveo&Og mun hans reiði á liggja og muntu hana hafa ef þú vilt svo margs manns blóði út hella um þessa sök en líkast ef þú lætur fyrirfarast þetta á þessari hátíðinni um friðinn að guð og hinn heilagi Jóhannes sjái þér hlut til handa um þetta \textbf{sök} \\
\hline
1818&sök&nvfo&Hefi eg hvern drottinsdag beðið yður til fyrir guðs sakir en þér vilduð því aldrei \textbf{sakir} \\
\hline
1819&sök&nveþ&« Það þykir mönnum mikil vandræði ef hraustir menn skulu drepast niður án nokkurri sök og býðst eg til að gera í milli \textbf{sök} \\
\hline
1820&sök&nvfo&« sakir nauðsynja \textbf{sakir} \\
\hline
1821&sök&nvfþ&Illa líkaði Ólöfu við Þórhall er hann hafði sagt til Þórðar og var við sjálft búið að hún mundi skilja við hann fyrir þessum \textbf{sökum} \\
\hline
1822&sök&nvfo&kallar honum best fallið að hafa slíka menn fyrir sakir ofsa og \textbf{sakir} \\
\hline
1823&sök&nveþ&Lýsi eg handseldri sök Þorgeirs \textbf{sök} \\
\hline
1824&sök&nveo&Báru þeir svo skapaðan fram kviðinn í fimmtardóm yfir höfði þeim manni er Mörður hafði sök sína fram \textbf{sök} \\
\hline
1825&sök&nveo&Þann kallar hann svo er leiður er friðurinn svo að hann keppist til smárra hluta og fær þó eigi en lætur fyrir þá sök farsællega hluti \textbf{sök} \\
\hline
1826&sök&nvfo&kveðst því hafa þangað sótt erfiðlega um langan veg að hann vænti fyrir frændsemis sakir þar \textbf{sakir} ásjá\\
\hline
1827&sök&nvfo&hlupu þar í byggðir er þeim þótti sitt færi vera fyrir \textbf{sakir} sakir\\
\hline
1828&sök&nveo&« Ekki höfum vér bræður fyrir þá sök haft liðsafnað að vér munum ófrið bjóða yður konungur heldur ber hitt til konungur að vér viljum yður fyrst bjóða vora þjónustu en ef þér neitið og ætlið Þorbergi nokkura afarkosti þá munum vér fara allir með lið það er vér höfum á fund Knúts hins \textbf{sök} \\
\hline
1829&sök&nvfo&Konungur beiddi að hann réðist til þessa vanda fyrir guðs sakir og bænar hans « og mun eg senda þig til Danmerkur á fund Össurar erkibiskups í Lund með mínum bréfum og \textbf{sakir} \\
\hline
1830&sök&nveo&« Þar liggur þú Magnús konungur og gefur meira gaum að drepa frænda skálds míns fyrir litla sök en að fá fagran sigur á \textbf{sök} \\
\hline
1831&sök&nvfo&En þær sakir sem gerst hafa millum vor og annarra leikmanna skiljum vér undan \textbf{sakir} dómi\\
\hline
1832&sök&nveo&Þá segir Sigurður drottningu hverrar ættar Ólafur var og fyrir hverja sök hann var þar \textbf{sök} \\
\hline
1833&sök&nvfo&Kálfur Árnason beiddist þess af konungi að hann gifti honum konu þá er Ölvir hafði átt og fyrir vináttu sakir veitti konungur honum það og þar með eignir þær allar er Ölvir hafði \textbf{sakir} \\
\hline
1834&sök&nvfo&» Og við fortölur Guðrúnar miklaði Bolli fyrir sér fjandskap allan á hendur Kjartani og sakir og vopnaðist síðan skjótt og urðu níu \textbf{sakir} \\
\hline
1835&sök&nvfo&undir hann mart lið fyrir ofríkis sakir en margir guldu \textbf{sakir} \\
\hline
1836&sök&nvfo&Það var siðvenja þeirra Gunnars og Njáls að sinn vetur þá hvor þeirra heimboð að öðrum og veturgrið fyrir vináttu \textbf{sakir} \\
\hline
1837&sök&nvfo&« Hafið þér nú ríki til þess fyrir liðfjölda sakir en aldrei hef eg fyrr rannsakaður verið sem \textbf{sakir} \\
\hline
1838&sök&nvfo&Ögmundur reið þegar að sjá yfir efni hans og vildi hann kalla það að Höskuldur mætti eigi annast ómagann fyrir fjár \textbf{sakir} \\
\hline
1839&sök&nveo&» Í annað sinn nefndi Gissur sér votta og lýsti sök á hönd Gunnari Hámundarsyni um það er hann særði Þorgeir Otkelsson holundarsári því er að ben gerðist en Þorgeir fékk bana af á þeim vettvangi er Gunnar hljóp til Þorgeirs lögmætu frumhlaupi \textbf{sök} \\
\hline
1840&tíðindi&nhfn&Og er þessi tíðindi spurðust til Ísafjarðar þá tekur Sigríður það ráð og Þórálfur frændi hennar að kveðja til bændur og láta virða Sigríði allt sitt fé af \textbf{tíðindi} \\
\hline
1841&tíðindi&nhfn&Þórð Einum vetri eftir lát Snorra Sturlusonar hófust þeir atburðir er mörg tíðindi gerðust af \textbf{tíðindi} \\
\hline
1842&tíðindi&nhfo&Eftir það hljóp Brandur í búð til Sighvats og segir honum tíðindi \textbf{tíðindi} \\
\hline
1843&tíðindi&nhfþ&Þórir tók ekki mjög á þessum tíðindum og bað þó Grím fara til sín « en ekki vil eg taka við \textbf{tíðindum} \\
\hline
1844&tíðindi&nhfo&Hann kemur til Laugarhúsa og hittir Bjarna bróður sinn og segir honum þessi tíðindi og biður að hann muni nokkurn hlut að eiga um eftirmál \textbf{tíðindi} \\
\hline
1845&tíðindi&nhfo&spyrja hann tíðinda en hann segir þau tíðindi sem gerst höfðu um \textbf{tíðindi} \\
\hline
1846&tíðindi&nhfþ&Og um morguninn eftir tók hann hest sinn og reið upp til Hofs og var þar við honum vel tekið og var spurður að tíðindum en hann sagði að maður braut fót \textbf{tíðindum} \\
\hline
1847&tíðindi&nhfn&En er njósnarmenn komu aftur til borgarinnar þá kunnu þeir segja þau tíðindi að höfðingi Væringja væri sjúkur og fyrir þá sök var engi atsókn til \textbf{tíðindi} \\
\hline
1848&tíðindi&nhfþ&Þessu næst er að segja frá þeim tíðindum að Hjörgunnur kona Hjörvarðar jarls tók sótt hættlega og þarf þar eigi að gera mikinn orðahjaldur að þessi sótt leiðir Hjörgunni til \textbf{tíðindum} \\
\hline
1849&tíðindi&nhfþ&segja þá konungi glögglega allan atburð sinnar þarkomu og frá þeim tíðindum er gerðust á \textbf{tíðindum} \\
\hline
1850&tíðindi&nhfn&« Hér eru orðin hörmuleg tíðindi eða hvað er nú til \textbf{tíðindi} \\
\hline
1851&tíðindi&nhfþ&Hann reið þegar til þings og varð mönnum dátt um það því að hann kunni vel að segja frá \textbf{tíðindum} \\
\hline
1852&tíðindi&nhfo&Enginn þinna manna hefur mér þessi tíðindi \textbf{tíðindi} \\
\hline
1853&tíðindi&nhfn&Og er menn komu á fund Sturlu og sögðu honum \textbf{tíðindi} \\
\hline
1854&tíðindi&nhfþ&spurði að ferðum Þorgils og að tíðindum úr \textbf{tíðindum} \\
\hline
1855&tíðindi&nhfn&Og er þau tíðindi komu til Noregs var Þorlákur biskup kominn frá vígslu til skips og fór það sumar til Íslands og sagði þessi tíðindi \textbf{tíðindi} \\
\hline
1856&vetur&nkeog&En um veturinn var það siður Arnkels að flytja heyið af Örlygsstöðum um nætur er nýlýsi voru því að þrælarnir unnu alla \textbf{veturinn} \\
\hline
1857&vetur&nkeog&Um veturinn er það sagt að maður sá kemur til hirðarinnar er Hárekur \textbf{veturinn} \\
\hline
1858&vetur&nkeog&» Þeir skilja nú við þetta og reið Karl heim til Upsa og sat þar um veturinn með \textbf{veturinn} \\
\hline
1859&vetur&nken&Þenna vetur lét Órækja son Snorra Sturlusonar drepa Klæng Bjarnarson í Reykjaholti annan dag jóla og fóru síðan að Gissuri í \textbf{vetur} \\
\hline
1860&vetur&nkeog&» Þorvaldur gekk til sætis og er hann þar um \textbf{veturinn} \\
\hline
1861&vetur&nkeog&Um veturinn kom Össur að máli við biskup að hann ætti þangað févon eftir Arnbjörn frænda sinn og beiddi biskup þar gera greiða á bæði fyrir sína hönd og annarra \textbf{veturinn} \\
\hline
1862&vetur&nkeo&Þeir voru átján vetra gamlir og höfðu einn vetur utan \textbf{vetur} \\
\hline
1863&vetur&nkeo&Þann vetur fór Björn til hirðar Eiríks jarls og var með \textbf{vetur} \\
\hline
1864&vetur&nkeog&Um sumarið eftir fór Jón Brandsson norður til Þingeyra til gildis og við honum Guðmundur Arason því að Ingimundur prestur vildi að hann færi á Háls til vistar til Þorvarðs og var hann þar of \textbf{veturinn} \\
\hline
1865&vetur&nken&þá Knútur sonur Sveins annar sjö vetur og er \textbf{vetur} \\
\hline
1866&vetur&nken&Hann hafði þá utan verið fimmtán vetur í einu og orðið gott til fjár bæði og \textbf{vetur} \\
\hline
1867&vetur&nkeo&Þrándur spurði nú lát föður síns og bjóst þegar af Suðureyjum og Önundur tréfótur með honum en þeir Ófeigur grettir og Þormóður skafti fóru út til Íslands með skuldalið sitt og komu út á Eyrum fyrir sunnan landið og voru hinn fyrsta vetur með Þorbirni \textbf{vetur} \\
\hline
1868&vetur&nkeo&» Ingimundur þakkar boðið og kvaðst mundu vera hjá honum í vetur « en þar sem eg hefi breytt ráðahag mínum til þessar ferðar þá mun eg þangað á leita sem mér var á vísað til \textbf{vetur} \\
\hline
1869&vetur&nkeo&En þá er hann hafði verið biskup tuttugu og fjóra vetur sem faðir hans þá var Jón vígður til \textbf{vetur} \\
\hline
1870&vetur&nkeog&» Nú líður veturinn og fór Geitir um vorið til Hofs að heimta peninga Höllu í annað sinn en Helgi vildi eigi út \textbf{veturinn} \\
\hline
1871&vetur&nkeþ&Hann andaðist ellefu hundruð og tíu og einum vetri eftir burð Krists en fimm vetrum fyrr en Hvamms-Sturli væri \textbf{vetri} \\
\hline
1872&vetur&nkeo&» eftir fall Haralds konungs en síðan réð hann landi tvo vetur með Ólafi bróður \textbf{vetur} \\
\hline
1873&vetur&nkeo&Í þenna tíma réðst Sighvatur Sturluson norður til Eyjafjarðar og var hinn fyrsta vetur á Möðruvöllum í Hörgárdal með Sigurði mági sínum \textbf{vetur} \\
\hline
1874&vetur&nkfþ&Glámur kvað sér vel hent að geyma \textbf{vetrum} á vetrum\\
\hline
1875&vetur&nken&Tvo vetur fulla var hann þar og það hins þriðja sem hann \textbf{vetur} \\
\hline
1876&vetur&nkeo&Þetta sumar er kaup þessi voru réðust þau Sigurður og Þuríður til Hóla og voru þar tvo \textbf{vetur} \\
\hline
1877&vetur&nkeog&Eitt sumar kom af Íslandi Oddur sonur Ófeigs Skíðasonar og kom norður við Finnmörk og var þar of \textbf{veturinn} \\
\hline
1878&vetur&nkeþ&Það var einum vetri eftir fall Ólafs konungs hins \textbf{vetri} \\
\hline
1879&vetur&nkeog&Um veturinn eftir Örlygsstaðafund voru þeir með Skúla hertoga í Niðarósi Snorri Sturluson og Órækja son hans og Þorleifur Þórðarson en Þórður kakali var í Björgvin með Hákoni \textbf{veturinn} \\
\hline
1880&vetur&nkeog&Björn lá í sárum um sumarið og um veturinn eftir var hann í Garðaríki og hafði hann þá utan verið þrjá vetur og eftir það fór hann til \textbf{veturinn} \\
\hline
1881&vetur&nkeog&Nú komu þeir til Grænlands og eru með Eiríki rauða um \textbf{veturinn} \\
\hline
1882&vetur&nkeþg&Nú líður fram vetrinum og þegar á bak jólum býr jarl ferð sína og hefir sex tigu \textbf{vetrinum} \\
\hline
1883&vetur&nkeo&Vinátta var þar mikil í millum þeirra bræðra og Vigfúss og höfðu sinn vetur hvorir jólaveislu með öðrum og skulu þeir bræður nú búast við \textbf{vetur} \\
\hline
1884&vetur&nken&En Guðmundur fór inn í Saurbæ í Eyjafjörð til Ólafs Þorsteinssonar og var hann þar þessa tvo vetur er fóstri hans var að \textbf{vetur} \\
\hline
1885&vetur&nkeog&Fór Þorsteinn vistafari til Styrkárs á Gimsar og var þar um \textbf{veturinn} \\
\hline
1886&vetur&nkeog&Hafði hann þá um haustið vinaboð mikið og enn jólaboð um veturinn og bauð þá enn til sín mörgum \textbf{veturinn} \\
\hline
1887&vetur&nkeo&Ingunn kona hans fæddi barn um vorið þá er þau höfðu þar verið hinn fyrsta vetur og hét sveinn sá \textbf{vetur} \\
\hline
1888&vetur&nkeo&Þá var liðið frá falli hins heilaga Ólafs konungs sex vetur hins tíunda tigar og hundrað tólfrætt en frá brennunni í Hítardal er mest tíðindi höfðu þá önnur orðið hér á \textbf{vetur} \\
\hline
1889&vetur&nkfþ&Fám vetrum síðar fékk hann Ásgerðar dóttur Asks hins \textbf{vetrum} \\
\hline
1890&vetur&nkeog&Síðan sendi Þórarinn Þorkeli orð að eigi mundi að svo búnu auðsótt til Hofs og líður nú enn \textbf{veturinn} \\
\hline
1891&vetur&nkeog&Var Sveinbjörn þar um veturinn en Einar var á Eyri með Guðrúnu föðursystur þeirra en Grímur og Krákur í \textbf{veturinn} \\
\hline
1892&vetur&nkeog&Nú siglir Karlsefni í haf og kom skipi sínu fyrir norðan land í Skagafjörð og var þar upp sett skip \textbf{veturinn} um veturinn\\
\hline
1893&vetur&nkeo&Rak konungur af sér þann vetur \textbf{vetur} og úthlaupsmenn\\
\hline
1894&vetur&nken&Þenna vetur sat Ólafur konungur austur í Sarpsborg og það spurðist austan að konungs var ekki norður \textbf{vetur} \\
\hline
1895&vetur&nkeþ&Á þeim sama vetri fékk sótt Gestur Oddleifsson og er að honum leið sóttin þá kallaði hann til sín Þórð lága son sinn og \textbf{vetri} \\
\hline
1896&vetur&nkeo&Hann hafði þá sjö vetur hins sjöunda \textbf{vetur} \\
\hline
1897&vetur&nkeo&Þeir bræður voru þann vetur í hans sveit og sátu næst þeim \textbf{vetur} \\
\hline
1898&vetur&nkeo&Þann vetur áður en Óttar kom til Ólafs konungs hafði andast Ólafur \textbf{vetur} \\
\hline
1899&vetur&nkfo&Herjólfur son hans var þá átta \textbf{vetra} \\
\hline
1900&vetur&nkeog&Um veturinn eftir geisladag stefndu Þorvaldssynir þeim mönnum til sín er þeim þóttu röskvastir og fóru vestan með fimm tigu \textbf{veturinn} \\
\hline
1901&vetur&nkeo&Skalla-Grímur bauð Óleifi heim til sín til vistar og liði hans öllu en Óleifur þekktist það og var hann með Skalla-Grími hinn fyrsta vetur er Óleifur var á \textbf{vetur} \\
\hline
1902&vetur&nkeþ&Á þeim vetri vænti hvorutveggi sér liðs norðan úr dalnum af sínum \textbf{vetri} \\
\hline
1903&vetur&nkeog&Líður af veturinn svo að ekki bar til \textbf{veturinn} \\
\hline
1904&vetur&nkeog&Hann var hinn mesti vin Kára og var Kári með honum um \textbf{veturinn} \\
\hline
1905&vetur&nken&Þá sagði lögsögumaður að engi skyldi lengur í sekt vera en tuttugu vetur alls þó að hann gerði útlegðarverk í þeim tímum « en fyrr mun eg öngvan úr sekt \textbf{vetur} \\
\hline
1906&vetur&nkeog&Margt varð til greina um veturinn með konungsmönnum og \textbf{veturinn} \\
\hline
1907&vetur&nkeo&Sá var hinn þrettándi vetur konungsdóms \textbf{vetur} \\
\hline
1908&vinur&nken&væri sannur vinur Ólafs konungs en sumum þótti ekki trúlegt og kváðu hann ráða mundu því við Svíakonung að hann héldi orð sín og sáttmál þeirra Ólafs \textbf{vinur} digra\\
\hline
1909&vinur&nken&Hann var vinur \textbf{vinur} \\
\hline
1910&vinur&nken&Ásbjörn kvað hann eigi mundi svo hjá sitja málunum að eiga ekki við Þórð en vera vinur \textbf{vinur} \\
\hline
1911&vinur&nkfn&Látum eigi aðra eiga að því að hlæja að vér leggjum slíkt til deilu þar er til móts eru vinir og \textbf{vinir} \\
\hline
1912&vinur&nkfþ&Kærði biskup þá fyrir Brodda ójafnað þann er Þorgils hefði veitt honum og vinum \textbf{vinum} \\
\hline
1913&vinur&nkfþ&bauð til sín vinum sínum og \textbf{vinum} \\
\hline
1914&vinur&nkfn&Þeir menn höfðu verið hjá viðurtali þeirra Gísla sem voru vinir Bjarnar í Hítardal og sögðu honum innilega \textbf{vinir} \\
\hline
1915&vinur&nken&Tjörvi var vinur þeirra Brodd-Helga og Geitis en hann var horfinn þann dag allan er Austmaðurinn var \textbf{vinur} \\
\hline
1916&vinur&nkeo&Einar Skúlason var með þeim bræðrum Sigurði og Eysteini og var Eysteinn konungur mikill vin \textbf{vin} \\
\hline
1917&vinur&nkfþ&Um haustið að veturnóttum bauð Ólafur til sín vinum \textbf{vinum} \\
\hline
1918&vinur&nkfn&þá er konungur var klæddur var hann fámálugur og ókátur og hræddust vinir hans að þá mundi enn að honum komið \textbf{vinir} \\
\hline
1919&vinur&nkfn&Aldregi heyrði eg hann leita þess ráðs að vinir hans skyldu kenna honum að \textbf{vinir} \\
\hline
1920&vinur&nken&Fannst konungi það í ræðum hans að hann misjafnaði mjög frásögu um jarlana og var vinur mikill Þorfinns en lagði þungt til Einars \textbf{vinur} \\
\hline
1921&vinur&nkeo&Hann var vin \textbf{vin} \\
\hline
1922&vinur&nkfn&Eftir þetta gengu að beggja vinir og báru sáttmál milli \textbf{vinir} \\
\hline
1923&vinur&nkfn&Skildust þeir vinir og fór Egill ferðar sinnar og kom aftan dags til hirðar jarlsins Arnviðar og fékk þar allgóðar \textbf{vinir} \\
\hline
1924&vinur&nkfn&Og með því að það var konungs boð þá sá hún það að ráði og með henni vinir hennar að heitast Þórólfi ef það væri föður hennar eigi í móti \textbf{vinir} \\
\hline
1925&vinur&nkfþ&Hún gengur fram og biður Þorkel bónda sinn stöðvast « vil eg að þú gerir honum ekki grand nema þú viljir að við skiljum okkart félag upp frá þessum degi ef þú gerir honum nokkurt mein því að Gunnar var mér sendur af vinum mínum til halds og \textbf{vinum} \\
\hline
1926&vinur&nkfn&Þeir voru þingmenn og vinir Þorgeirs \textbf{vinir} \\
\hline
1927&vinur&nkfn&Þeir gáfu Þorleifi sinn hlut skips og skildu þeir góðir vinir \textbf{vinir} \\
\hline
1928&vinur&nkfo&Bíð mín hér meðan eg hitti vini \textbf{vini} \\
\hline
1929&vinur&nkfo&fyrst austur í Svíaveldi og gera þá ráð sitt hvert hann ætlar eða sneri þaðan af en bað svo vini sína til ætla að hann mundi enn ætla til landsins að leita og aftur til ríkis síns ef guð léði honum \textbf{vini} \\
\hline
1930&vinur&nkfo&Hefi eg og látið allar mínar eigur og frændur og vini er eg átti í Noregi og fylgt yður en allir lendir menn yðrir skildust við yður og er það maklegt því að þú hefir marga hluti til mín stórvel \textbf{vini} \\
\hline
1931&vinur&nkfþ&Gerði konungur þá bert fyrir vinum sínum að sú var ætlan hans að fara þá úr landi í \textbf{vinum} \\
\hline
1932&vinur&nkfþ&Þá stefndi hann til sín vinum sínum og hafði nær níu tigum manna og alla vel \textbf{vinum} \\
\hline
1933&vinur&nkfn&Þórólfur bægifótur og margir aðrir þingmenn \textbf{vinir} og vinir\\
\hline
1934&vinur&nkeo&Markús var þingmaður Sæmundar \textbf{vin} og vin\\
\hline
1935&vinur&nkeo&« Let eg þig þess og svo hvern annan minn vin að drepa prestinn því að Þorvarður bróðir minn hefir fyrrum grimmlega hefnt smærri meingerða en eg get að honum þyki \textbf{vin} \\
\hline
1936&vopn&nhen&Hann var eigi sár því að eigi festi vopn á kyrtli \textbf{vopn} \\
\hline
1937&vopn&nhfo&» Þórhaddur bað hann að sækja en Þorsteinn kvaðst spara menn sína til þess að ganga á vopn \textbf{vopn} \\
\hline
1938&vopn&nhfo&« Nú munum vér herma orð yður að þar skal meira fyrir verða að hefna Ketilbjarnar en að vér göngum á land upp undir vopn \textbf{vopn} \\
\hline
1939&vopn&nhfo&Þá sendi hann annað sinni menn til Norðmanna og bað þá í brott fara og hafa vopn \textbf{vopn} \\
\hline
1940&vopn&nhen&Það lofaði hann þeim en fékk vopn þeirra sínum mönnum en þeir ganga í \textbf{vopn} \\
\hline
1941&vopn&nhfn&Eftir um daginn tóku þeir Þóroddur öll klæði sín og vopn og lögðu sér til \textbf{vopn} \\
\hline
1942&vopn&nhfn&Aron Hjörleifsson spurði Eyjólf Kársson hvar væru vopn \textbf{vopn} \\
\hline
1943&vor&nheog&En um vorið búa þeir mágar skip það er Austmenn höfðu átt til \textbf{vorið} \\
\hline
1944&vor&nheog&Um vorið nálega kyndilmessu þá lagði Magnús konungur brott á náttarþeli og lögðu út tjölduðum skipum og ljós undir og hélt út til \textbf{vorið} \\
\hline
1945&vor&nheog&Og um vorið eftir víg Orms réðst Björn á Breiðabólstað og tók við búi því er Ormur hafði átt og \textbf{vorið} \\
\hline
1946&vor&nheo&Kemur til þess ófræði vor og það annað að vér viljum eigi setja á bækur vitnislausar \textbf{vor} \\
\hline
1947&vor&nhen&En til vinstri handar frá minni fylking skal vera það lið er Svíakonungur fékk oss og allt það lið er til vor kom í \textbf{vor} \\
\hline
1948&vor&nheog&Um vorið bað Halli konung orlofs að fara til \textbf{vorið} í kaupferð\\
\hline
1949&vor&nheog&En um vorið fór hann að heimta saman fé Austmanna og sýndist engum ráð að halda fyrir honum réttri skuld og heimti hann hverja alin þá er honum bar að \textbf{vorið} \\
\hline
1950&vor&feveo&« Bitu hann enn ráðin Haralds konungs en brátt mun eftir verða ætt vora ef Haraldur konungur skal einn fyrir \textbf{vora} \\
\hline
1951&vor&nheo&» Skafti sagði að hann ætlaði að það mál væri vel komið þótt þeir gerðu um « því að málaefni vor eru brýn og góð en þeir eru svo vitrir að þeir munu sjá kunna hversu þungs þú ert af \textbf{vor} \\
\hline
1952&vor&nheo&« Sætt þeirri er konungur gerði mun eg hlíta um mál vor Brúsa en það er til þín kemur af skaltu einn \textbf{vor} \\
\hline
1953&vor&nheo&Er mér sagt að konungur mun koma norður í vor en ef hann kemur eigi þá mun eg ekki letja að við förum á \textbf{vor} fund\\
\hline
1954&vor&nheþ&« Akra-Þórir gaf mér hafra þessa á vori til liðs sér er þú hafðir stefnt honum en nú var markað fyrir féránsdóma og á eg \textbf{vori} \\
\hline
1955&vor&nheo&Eitthvert vor átti Þorkell hinn auðgi Þórðarson Víkingssonar för suður til Þórsnessþings og fylgdu honum \textbf{vor} \\
\hline
1956&vor&nheog&Um vorið á páskum sendi Þórður Nikulás Oddsson og Almar Þorkelsson norður til Öxarfjarðar í liðsafnað og stefndi liðinu norðan í \textbf{vorið} \\
\hline
1957&vor&nheog&Um vorið eftir sendir biskup bréf sín til staðarins og bað læsa kirkju og kallaði hana saurgaða bæði af manndrápi og grefti bannsettra \textbf{vorið} \\
\hline
1958&vor&nheog&En eftir um vorið sendi Danakonungur Eirík norður í Noreg og gaf honum jarldóm og þar með Vingulmörk og Raumaríki til yfirsóknar með þeim hætti sem fyrr höfðu þar haft \textbf{vorið} \\
\hline
1959&vor&nheog&Um vorið eftir fór Hrafn suður til Borgarfjarðar í Reykjaholt til sáttarfundar þess er þeir Þorvaldur og Hrafn höfðu ákveðið með \textbf{vorið} \\
\hline
1960&vor&nheog&Annað sumar eftir fór Björn biskup norður til Þverár að vígja til ábóta Björn bróður sinn og í þeirri för biskupaði hann Guðmund Arason á Möðruvöllum og var það um vorið eftir \textbf{vorið} \\
\hline
1961&vor&nheog&Um vorið bað Gunnlaugur konunginn sér orlofs til \textbf{vorið} \\
\hline
1962&vor&nheog&En of vorið fóru þeir út þangað Saxólfur Fornason er bjó í Myrkárdal og Urða-Steinn og Arnþrúðarsynir og tóku upp eigu Þorsteins alla og höfðu til Guðmundar og varð engi forstaða \textbf{vorið} \\
\hline
1963&vor&nheog&Þau Bergdís og Þórir fóru um vorið úr Grímsey og vestur yfir heiði til \textbf{vorið} \\
\hline
1964&vor&nheog&Þá réðst þann veg Guðmundur frændi hans um langaföstu um \textbf{vorið} \\
\hline
1965&vor&nheþ&Ætlaði hann að ferjan skyldi ganga til Stranda að vori til \textbf{vori} \\
\hline
1966&vor&nheog&Fór Kolbeinn norður til föður síns og var þar um veturinn en um vorið tóku þeir feðgar heimildum á Grenjaðarstöðum af Jóni Eyjólfssyni í Möðrufelli og gerði Kolbeinn þar bú \textbf{vorið} \\
\hline
1967&vor&nheog&Um vorið segir Þorgils Ólafi að hann vill fara kaupferð um sumarið og leita svo undan ójafnaði \textbf{vorið} \\
\hline
1968&vor&nheog&Um vorið dregur Kolbeinn lið saman um öll héruð til \textbf{vorið} \\
\hline
1969&vor&nheo&« og skaltu þó heita oss áður að koma aftur til vor það fljótasta sem þú getur því að vér viljum þín ekki missa sakir íþrótta \textbf{vor} \\
\hline
1970&vor&nheog&Um vorið fóru menn til \textbf{vorið} \\
\hline
1971&víg&nhen&Það hygg eg að hvert sinn er þú ríður norður og norðan komi þér í hug víg Þorkels háks en frændur hans sitja hér í hverju húsi og mun þér ótti af því \textbf{víg} \\
\hline
1972&víg&nhen&Mun eg senda mann til Reykja að segja Skeggja víg \textbf{víg} \\
\hline
1973&víg&nhen&Þá var þar Teitur son Gissurar biskups og þá urðu þar þau tíðindi að íslenskur maður sá er Gísl hét vó víg og var sá maður er veginn var hirðmaður Magnúss \textbf{víg} \\
\hline
1974&víg&nheo&« að víg Einars komi fyrir víg Galta og þar með heimsókn við Kolfinnu fyrir mannamun en fyrir Gríssvísur skal Hallfreður gefa Grísi grip einn \textbf{víg} \\
\hline
1975&víg&nheo&Guðmundur Ásbjarnarson var með honum eftir víg þeirra feðga Kálfs og \textbf{víg} \\
\hline
1976&víg&nheo&« að fyrir víg Össurar geri eg tvö hundruð silfurs en hið þriðja skal falla niður fyrir fjörráð við Þórð og allan fjandskap en menn Össurar allir óhelgir fyrir tilför við \textbf{víg} \\
\hline
1977&víg&nheo&» Síðan stefndu þeir Finnboga um víg \textbf{víg} \\
\hline
1978&víg&nhen&« hversu Gunnari fór eftir víg Sigmundar frænda \textbf{víg} \\
\hline
1979&víg&nhen&víg \textbf{víg} og gripatakið\\
\hline
1980&vísa&sfg3eþ&En Arnkell veik því af sér og kvað það koma til Kjalleklinga frænda hans og vísaði hann þessu máli helst á \textbf{vísaði} \\
\hline
1981&vísa&nven&Vísa þessi kom upp í Miðfirði er kveðin var til \textbf{Vísa} \\
\hline
1982&þing&nheþ&En á þingi um sumarið lýsa þeir Gissur sekt \textbf{þingi} að Lögbergi\\
\hline
1983&þing&nheþ&« Ef þeir vita nú það þegar að rangt hafi verið til búið málið þá mega þeir svo bjarga sökinni að senda þegar mann heim af þingi og stefna heiman til þings en kveðja búa á þingi og er þá rétt sótt \textbf{þingi} \\
\hline
1984&þing&nheog&að Pétursmessudagsmorgun um þingið gengu flokkarnir allir til kirkju um messu um guðspjall og stóðu með vopnum fyrir framan kirkjuna og stóðu sínum megin \textbf{þingið} hvorir\\
\hline
1985&þing&nheþ&Og er það mitt ráð að hver maður ríði heim af þingi og sjái um bú sitt í sumar meðan töður manna eru \textbf{þingi} \\
\hline
1986&þing&nhen&Og eitt sinn er þing er fjölmennt þá fer Halli til \textbf{þing} með höfðingjanum\\
\hline
1987&þing&nheþ&Þórður Andrésson reið til þings með Gissurarsonum og veitti þeim að öllum málum á því þingi og Andrés faðir \textbf{þingi} \\
\hline
1988&þing&nheþg&En höfðingjar munu leita annars ráðs en þið bræður séuð höggnir hér niður á þinginu \textbf{þinginu} \\
\hline
1989&þing&nhfþ&Oftlega var það siður hans að hafa kofra á höfði og jafnan á þingum en af því að hann var maður ekki nafnfrægur þá gáfu þingmenn honum það nafn er við hann festist að þeir kölluðu hann \textbf{þingum} \\
\hline
1990&þing&nheþ&Þorleifur beiskaldi tók við eftirmáli um víg Þórhalls og sótti þá á þingi og var Bergþór sekur \textbf{þingi} \\
\hline
1991&þing&nheþ&Báðu þeir Helga af Laugabóli taka við honum og annast hann þar til að Vermundur kæmi heim af \textbf{þingi} \\
\hline
1992&þing&nheþg&Varð Þorvaldur á þinginu sekur skógarmaður og sekt fé hans allt og \textbf{þinginu} \\
\hline
1993&þing&nheþ&Greip gjaldkerinn staðarins þegar Þorstein og spurði fyrir hverja sök hann gerði slíkt óhæfuverk þar á heilögu \textbf{þingi} \\
\hline
1994&þing&nheþg&Og á þinginu gekk hann einn dag til Einars Þveræings og heimti hann á tal við sig og sagði \textbf{þinginu} \\
\hline
1995&þing&nheþg&Þá létu þeir stefna þing fjölmennt og á þinginu talaði Sigurður jarl af hendi Hákonar og bauð bóndum hann til \textbf{þinginu} \\
\hline
1996&þing&nheþ&Og þegar af þingi ríður hann til Víðidals og kaupir landið að \textbf{þingi} \\
\hline
1997&þing&nheog&Eiríkur bjó um þingið skip sitt til hafs í Eiríksvogi í Öxnaey og veittu þeir Eiríki Þorbjörn Vífilsson og Víga-Styr og synir Þorbrands úr Álftafirði og Eyjólfur Æsuson úr \textbf{þingið} \\
\hline
1998&þing&nheþ&Svo segja menn að Hrútur væri svo á þingi eitt sumar að fjórtán synir hans væru með \textbf{þingi} \\
\hline
1999&þing&nheþg&Þetta sumar fór Guðmundur prestur hinn góði til þings en af þingi buðu honum heim Sunnlendingar og Austfirðingar og fór hann af þinginu suður í \textbf{þinginu} \\
\hline
2000&þing&nheog&Eftir þingið söfnuðu þeir báðir liði til féránsdóma og þá fór Einar í Hvamm með hálft fjórða hundrað manna en hann lét eftir í Saurbæ Hrólf Gunnólfsson við hundrað \textbf{þingið} \\
\hline
2001&þing&nheþg&Beiddi hann þess alla menn er á voru \textbf{þinginu} \\
\hline
\end{longtable}
\end{document}
